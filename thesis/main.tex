\documentclass[11pt]{article}
\usepackage[utf8]{inputenc}
\usepackage{graphicx}
\usepackage{geometry}
\usepackage{parskip}
\usepackage{subcaption}
\usepackage{wrapfig}
\usepackage{acronym}
\usepackage[natbib=true]{biblatex}
\usepackage{amsmath}
\usepackage{dirtytalk}
\usepackage{hyperref}
\usepackage{mathtools}
\usepackage{amsfonts}
\usepackage{csquotes}
\usepackage{pgffor}
% bold math symbols
\usepackage{bm}
\addbibresource{literature.bib}
\renewcommand{\baselinestretch}{1.5}
\newcounter{savepage}

\DeclarePairedDelimiterX{\infdivx}[2]{(}{)}{%
    #1\;\delimsize\|\;#2%
}
\newcommand{\kldiv}{D_{KL}\infdivx}
\DeclarePairedDelimiter{\norm}{\lVert}{\rVert}

\begin{document}

    \begin{titlepage}
        \centering
        \includegraphics[width=0.25\textwidth]{rublogo.png}\par
        {\scshape\huge\bfseries Semantic Representations in Variational Autoencoders as a Model of the Visual System \par}
        {\scshape\large Schriftliche Prüfungsarbeit für die Master-Prüfung des Studiengangs Angewandte Informatik an der Ruhr-Universität Bochum\par}
        \vspace{1em}
        vorgelegt von\par
        \vspace{2em}
        Leonard Papenmeier\par 108017257755\par
        \vspace{2em}
        01.01.1980\par

        \vfill
        Prof. Dr. Laurenz Wiskott\par
        M.Sc. Zahra Fayyaz


    \end{titlepage}
    \pagenumbering{Roman}
    \tableofcontents
    \newpage
    \setcounter{savepage}{\arabic{page}}
    \pagenumbering{arabic}


    \section{Introduction}\label{sec:introduction}


    \section{Theoretical Background}\label{sec:theoretical-background}
    \subsection{Primary Visual Cortex}\label{subsec:primary-visual-cortex}

\subsection{Visual Object Perception}\label{subsec:visual-object-perception}

\begin{wrapfigure}[12]{r}{0.3\textwidth}
    \begin{center}
        \includegraphics[width=0.28\textwidth]{images/rubens_sketches.jpg}
    \end{center}
    \caption[Copies of line drawings]{\say{Copies of line drawings.} taken from \citet{rubens1971associative}}
    \label{fig:copies_line_drawings}
\end{wrapfigure}
Recognizing an object as what it is is different from the ability of seeing an object or making a copy of it.
\citet{rubens1971associative} report the case of a 47-year old man who, on March 5--1969, \say{was found unconscious with vomitus on his face and bathrobe}.
Only after \say{his breathing became irregularly}, he was taken to a hospital where a low blood pressure was diagnosed.
The man showed an inability to recognize objects and in cases where he was unable to recognize an object, he also could not describe its use.
When given the category of an object, \say{identification improved very slightly}.
He claimed to recognize the item after being told the name.
In such cases, he was able to \say{point out various parts of the previously unrecognized item}.
When shown sketches of items, he was generally unable to recognize the items.
However, he was able to name geometric shapes such as circles or squares present in the sketch.
Even though the man did not recognize the objects, he was able to make copies of them (see Figure~\ref{fig:copies_line_drawings}).
\citet{rubens1971associative} report, that the Patient \say{was unable to identify any [items] before copying}.
However, he was able to contain some of the objects categories after copying them.

The example presented above shows that the ability to reproduce an object is different from the ability to \textit{perceive} it.

For monkeys, the \ac{IT} is assumed to be the brain region being crucial for object perception~\citep[pp. 1070, 1071]{squire2012fundamental}.
Bilateral lesions of the \ac{IT} in monkeys affect their ability to \say{distinguish between different visual patterns or objects, and in retaining previously acquired visual discriminations}~\citep[p. 1070]{squire2012fundamental}.
They are no longer able to generalize from tasks learned in one half of the visual space to the other half, presumably because the invariance of representations is lost~\citep[p. 1070]{squire2012fundamental}.
\citet[p. 1071]{squire2012fundamental} explicitly point out \say{the crucial role of the inferior temporal cortex during object perception and recognition}.

\subsection{Variational Autoencoders}\label{subsec:variational-autoencoders}

Since \acfp{VAE} are a specialization of the autoencoder, the traditional autoencoder is introduced first.

\subsubsection{Autoencoders}

Autoencoders are neural networks trained to reconstruct their input~\citep[p. 499]{Goodfellow-et-al-2016}.
For autoencoders, it is common to speak of an \textit{encoder}-part and a \textit{decoder}-part.
The encoder $f: \mathbb{R}^n \mapsto \mathbb{R}^m$ transforms an input $\mathbf{x}$ to a hidden representation $\mathbf{r} = f(\mathbf{x})$.
Usually $m < n$, i.e.\ the encoder transforms the input to a lower-dimensional representation.
This can be beneficial for dimensionality reduction or feature learning~\citep[p. 499]{Goodfellow-et-al-2016}.
The decoder $g: \mathbb{R}^m \mapsto \mathbb{R}^n$ transforms the hidden representation back into the original feature space.
Usually, one wants the reconstruction $\tilde{x}$ to be close to the original feature $x$ ($\tilde{x} \approx x$).
In order to achieve this, the autoencoder is usually trained by minimizing $\mathcal{L}(\mathbf{x}, g(f(\mathbf{x})))$ with
\begin{align}
    \mathcal{L}: \mathbb{R}^n \times \mathbb{R}^n \mapsto \mathbb{R}
\end{align}
One common choice for $\mathcal{L}$ is the \ac{MSE} which is defined as
\begin{align}
    \mathcal{L}(\mathbf{x}, \mathbf{y}) = \frac{1}{n}\sum (\mathbf{x}_i - \mathbf{y}_i)^2
\end{align}~\citep[p. 106]{Goodfellow-et-al-2016}.

\subsubsection{Variational Autoencoders}

\subsection{Visual Features in Neural Networks}\label{subsec:visual_features_in_neural_networks}
\begin{itemize}
    \item~\cite{krizhevsky2012imagenet} report Gabor wavelets in \acp{CNN} trained on image classification
\end{itemize}

\subsection{Semantic Representations}\label{subsec:semantic-representations}

\subsubsection{Supervised Models}
\begin{itemize}
    \item \citet{khaligh2014deep} found evidence for that supervised models may explain \ac{IT} cortical representation.
\end{itemize}

\subsubsection{Unsupervised Models}
\begin{itemize}
    \item \citet{han2019variational} found no evidence for or against Gabor wavelets in \acp{VAE} due to too small kernel size
    \item \citet{khaligh2014deep} found evidence against the assumption that unsupervised models might explain \ac{IT} cortical representation, however not explicitly for \acp{VAE}
\end{itemize}


    \section{Methods}\label{sec:methods}
    The following Sections describe the methods used in the course of this thesis.

\subsection{Research Questions}\label{subsec:research-questions}

The studies in Section~\ref{sec:results} are guided by the following research questions:

\begin{table}[H]
    \begin{tabularx}{\textwidth}{llX}
        \toprule \\
        \multicolumn{2}{l}{Number} & Question \\
        \midrule \\
        RQ1 &    & Are \acp{VAE} or \acp{VLAE} related to the visual cortex in terms of \ldots                                      \\
        & a) & \hspace{1cm} \ldots the emergence of Gabor wavelets?                                                             \\
        & b) & \hspace{1cm} \ldots sparse coding?                                                                               \\
        RQ2 &    & Do \acp{VAE} or \acp{VLAE} fulfil the requirements of latent space disentanglement or latent space separability? \\
        RQ3 &    & Can \acp{VAE} or \acp{VLAE} represent both continuous and categorical factors of variation in the latent space?  \\
        RQ4 &    & How do \acp{VAE} and \acp{VLAE} represent lower factors of variation in the latent space?                        \\
        RQ5 &    & Do \acp{VAE} or \acp{VLAE} learn independent factors of variation independently in the latent space?             \\
        RQ6 &    & Are the latent spaces of \acp{VLAE} independent in terms of generated images?                                    \\
        RQ7 &    & Do \acp{VAE}/\acp{VLAE}-generated images resemble the data distribution?                                         \\
        RQ8 &    & Is the discriminative loss superior to the pixel-wise loss in terms of the previous research questions?          \\
        \bottomrule
    \end{tabularx}
    \caption{Research Questions}
    \label{tbl:research_questions}
\end{table}

\subsection{Implementation Details}\label{subsec:implementation-details}

All models are implemented with Keras\footnote{\href{https://keras.io/}{https://keras.io/}, last access: 07/01/2020} in Version 2.2.4 using the TensorFlow\footnote{\href{https://www.tensorflow.org/}{https://www.tensorflow.org/}, last access: 07/01/2020} backend in Version 1.15..
The models are trained on Tesla V100-DGXS GPUs with 16GB of RAM.
The model code can be found under \href{https://github.com/LeoIV/master-thesis-leonard}{https://github.com/LeoIV/master-thesis-leonard}.

\subsection{Datasets}\label{subsec:datasets}

Five different datasets were used to train the models.
Four of the datasets contain images of different sizes, the fifth dataset provides additional labels for one of the datasets.
The images were resized to match the models' expected input sizes using Lanczos interpolation~\citep[pp. 223, ff]{burger2009principles}.

\subsubsection{CelebA}\label{subsubsec:celeba_dataset}

\begin{wrapfigure}[14]{R}{0.3\textwidth}
    \begin{center}
        \includegraphics[width=0.28\textwidth]{images/celeba_sample_63.jpg}
    \end{center}
    \caption[CelebA dataset sample image]{A sample image from the CelebA dataset.}
    \label{fig:celeba_sample}
\end{wrapfigure}

The \textit{CelebA} dataset~\citep{liu2015faceattributes} consists of 202,599 RGB images of size 178 x 218 pixels representing celebrities, as well as 40 binary attributes.
The images belong to 10.177 unique identities\footnote{The identities are not revealed.} as well as five \say{landmark annotations}.
They are aligned and cropped resulting in images of same size always showing only one face (see Figure~\ref{fig:celeba_sample} for an example).
The landmark annotations give the positions of facial attributes in the image; the left and right eye, the nose, and the left and right corner of the mouth.
The binary attributes indicate if the image has certain characteristics, for example if the person wears eyeglasses, has black hair, is smiling\footnote{See \href{https://www.kaggle.com/jessicali9530/celeba-dataset\#list\_attr\_celeba.csv}{https://www.kaggle.com/jessicali9530/celeba-dataset\#list\_attr\_celeba.csv} for a complete list of the attributes, login required. Last access: 12/02/2020.}.

\subsubsection{ImageNet}\label{ssec:imagenet}

ImageNet\footnote{\href{http://image-net.org/}{http://image-net.org/}, last access: 12/02/2020.} is a large-scale \say{image database organized according to the WordNet hierarchy}~\citep{imagenet_cvpr09}.
It contains of over 14 million images as of February 2020.
According to WordNet\footnote{See \href{https://wordnet.princeton.edu/}{https://wordnet.princeton.edu/}, last access: 12/02/2020.}, the images are subdivided into groups called \say{synsets}~\citep{imagenet_cvpr09} on different levels of granularity.
For example, the group \textit{woman, adult female} is subordinated to \textit{person, individual, someone, somebody, mortal, soul} and is further subdivided into groups like \textit{old woman} or \textit{lady} \footnote{ImageNet 2011 Fall Release, \href{http://image-net.org/explore}{http://image-net.org/explore}, last access: 12/02/2020.}

A smaller version of ImageNet, commonly called \textit{ILSVRC2012} has been used for \ac{ILSVRC2017}~\citep{ILSVRC15}, consisting of approximately 1,3 million images from 1000 different classes, that were selected, such that \say{there is no overlap between synsets: for any synsets $i$ and $j$, $i$ is not an ancestor of $j$ in the ImageNet hierarchy}~\citep{imagenet_cvpr09}.

This curated version is commonly used as a baseline~\citep{krizhevsky2012imagenet,Szegedy_2015_CVPR}\footnote{See \href{https://paperswithcode.com/sota/image-classification-on-imagenet}{https://paperswithcode.com/sota/image-classification-on-imagenet} for an overview of models on ImageNet. Last access: 07/16/2020}.

\subsubsection{\textsc{Mnist}}\label{subsubsec:mnist}

\begin{figure}
    \begin{center}
        \includegraphics[width=\textwidth]{images/mnist_sample.png}
    \end{center}
    \caption[\textsc{Mnist} dataset example images]{Examples from the \textsc{Mnist} dataset.}
    \label{fig:mnist_sample}
\end{figure}

\textsc{Mnist}\footnote{\href{http://yann.lecun.com/exdb/mnist/}{http://yann.lecun.com/exdb/mnist/}, last access: 23/04/2020}~\citep{lecun1998gradient} is a widely-used dataset of hand-written digits.
Figure~\ref{fig:mnist_sample} shows ten examples from this dataset.
The data is subdivided into a training set of 60.000 images and a test set containing 10.000 images.
The digits are all of the same size and centered.
The samples are grayscale images of size $28\times 28$pixels.

\subsubsection{Morpho-\textsc{Mnist}}\label{subsubsec:morphomnist}

\begin{figure}[H]
    \centering
    \includegraphics[width=\textwidth]{images/morpho_mnist_distribution.png}
    \caption[Morpho-\textsc{Mnist} distribution]{Distribution of the Morpho-\textsc{Mnist} attributes for the different digits. Taken from~\citep{castro2019morpho}.}
    \label{fig:morpho_mnist_distribution}
\end{figure}

Morpho-\textsc{Mnist}~\citep{castro2019morpho} is an extension of the \textsc{Mnist} dataset that addresses the question: \say{(T)o what extent has my model learned to represent specific factors of
variation in the data?} ~\citep{castro2019morpho}.
To address this questions, Morpho-\textsc{Mnist} provides the following (continuous) labels of morphological attributes of the \textsc{Mnist} samples: stroke length, stroke thickness, slant, width, and height.

Besides providing additional labels of low-level \textsc{Mnist} attributes, Morpho-\textsc{Mnist} provides a toolbox to measure (i.e~ calculate the morphological labels) and perturb MNIST images.
The perturbation toolbox allows it to thin, thicken, swell, and to add fractures to an image.
Morpho-\textsc{Mnist} also provides pre-computed datasets that were built using the perturbation toolbox.

Importantly, the distribution of the morphological attributes partly is highly skewed (for example Thickness and Height, see Figure~\ref{fig:morpho_mnist_distribution}).

\subsubsection{dSprites}
dSprites\footnote{\href{https://github.com/deepmind/dsprites-dataset/}{https://github.com/deepmind/dsprites-dataset/}, last access: 5/28/2020}~\citep{dsprites17} is a dataset designed \say{to assess the disentanglement properties of unsupervised learning methods.}.
It contains 737,280 grayscale images of size $64\times 64$ pixels.
The images were generated from \say{6 ground truth independent latent factors}: color, shape, scale, orientation, $x$-position, and $y$-position.
The color is white in all images.
The shapes are: square, ellipse, and heart.
For the other factors, points are chosen evenly along their support: six values in $[0.5, 1]$ (scale), 40 values in $[0, 2\pi]$ (orientation), 32 values in $[0, 1]$ ($x$-position and $y$-position).
Each factor combination only occurs once in the data set.
The dataset also contains the factor labels for each image.

\subsection{Models}\label{subsec:models}

Eight different \ac{VAE}, six different \ac{VLAE}, four different \ac{VAE}-\ac{GAN}, and four different \ac{VLAE}-\ac{GAN} were evaluated in the course of this thesis.
Furthermore, two \say{AlexNet} models were used for some additional experiments.

The models vary depending on the dataset and are described in the following.
An overview is given in Table~\ref{tbl:model_overview}.
A more detailed description can be found in Appendix~\ref{sec:appendix_network_architectures}.
\begin{table}
    \centering
    \begin{tabular}{lrrrrr}
        \toprule
        model name              & dataset        & \parbox[t]{2cm}{\normalsize\raggedleft input/output\\size}       & \parbox[t]{2cm}{\normalsize\raggedleft latent\\space\\size} & \parbox[t]{2cm}{\normalsize\raggedleft reconstruction\\term weight} & \parbox[t]{2cm}{\normalsize\raggedleft feature\\map\\reduction factor} \\
        \midrule
        \textsc{Mnist}-\ac{VAE} & \textsc{Mnist} & $28\times 28\times 1$   & 2                 & 10,000                     & 1                            \\
        (dSprites/10,000)-\ac{VAE}       & dSprites       & $64\times 64\times 1$   & 10                & 10,000                     & 1                            \\
        7,500-\ac{VAE}          & dSprites       & $64\times 64\times 1$   & 10                & 7,500                      & 1                            \\
        6,250-\ac{VAE}          & dSprites       & $64\times 64\times 1$   & 10                & 6,250                      & 1                            \\
        5,000-\ac{VAE}          & dSprites       & $64\times 64\times 1$   & 10                & 5,000                      & 1                            \\
        3,750-\ac{VAE}          & dSprites       & $64\times 64\times 1$   & 10                & 3,750                      & 1                            \\
        dSprites-\ac{VAE}-dim6  & dSprites       & $64\times 64\times 1$   & 6                 & 10,000                     & 1                            \\
        CelebA-\ac{VAE}         & CelebA         & $128\times 128\times 3$ & 8                 & 3,750                      & 1                            \\
        \midrule
        \textsc{Mnist}-\ac{VLAE}(-factor-1) & \textsc{Mnist} & $28\times 28\times 1$ & 2,2,2 & 10,000 & 1 \\
        \textsc{Mnist}-\ac{VLAE}-factor-2 & \textsc{Mnist} & $28\times 28\times 1$ & 2,2,2 & 10,000 & 2 \\
        \textsc{Mnist}-\ac{VLAE}-factor-3 & \textsc{Mnist} & $28\times 28\times 1$ & 2,2,2 & 10,000 & 3 \\
        dSprites-\ac{VLAE} & dSprites & $64\times 64\times 1$ & 4,4,4 & 10,000 & 1 \\
        dSprites-\ac{VLAE}-dim2 & dSprites & $64\times 64\times 1$ & 2,2,2 & 10,000 & 1 \\
        CelebA-\ac{VLAE} & CelebA & $128\times 128\times 3$ & 2,2,2 & 10,000 & 1 \\
        \midrule
        \textsc{Mnist}-\ac{VAE}-\ac{GAN} & \textsc{Mnist} & $28\times 28\times 1$ & 2 & 10,000 & 1\\
        dSprites-\ac{VAE}-\ac{GAN} & dSprites & $64\times 64\times 1$ & 10 & 10,000 & 1\\
        CelebA-\ac{VAE}-\ac{GAN} & CelebA & $128\times 128\times 3$ & 8 & 10,000 & 1\\
        \midrule
        \textsc{Mnist}-\ac{VLAE}-\ac{GAN} & \textsc{Mnist} & $28\times 28\times 1$ & 2,2,2 & 10,000 & 1\\
        dSprites-\ac{VLAE}-\ac{GAN} & dSprites & $64\times 64\times 1$ & 4,4,4 & 10,000 & 1\\
        CelebA-\ac{VLAE}-\ac{GAN} & CelebA & $128\times 128\times 3$ & 2,2,2 & 10,000 & 1\\
        \midrule
        AlexNet Classifier & ImageNet & $224\times 224\times 3$ & -- & -- & 1\\
        \midrule
        AlexNet \ac{VAE} & ImageNet & $224\times 224\times 3$ & 2000 & 10,000 & 1\\
        \bottomrule
    \end{tabular}
    \caption[Models Overview]{Overview of all models with important parameters.}
    \label{tbl:model_overview}
\end{table}

\subsubsection{VAE Models}\label{subsubsec:vae_models}

\begin{figure}
    \centering
    \begin{subfigure}{.5\textwidth}
        \centering
        \includegraphics[width=\textwidth,height=.85\textheight,keepaspectratio]{images/vae/encoder.png}
        \caption{Encoder}
    \end{subfigure}%
    \begin{subfigure}{.5\textwidth}
        \centering
        \includegraphics[width=\textwidth,height=.85\textheight,keepaspectratio]{images/vae/decoder.png}
        \caption{Decoder}
    \end{subfigure}
    \caption{VAE model structure}
    \label{fig:vae_model_structure}
\end{figure}

The \ac{VAE} model (see Section~\ref{fig:vae_model_structure}) consists of an encoder and a decoder.
The encoder is made up of multiple \say{Convolution, Activation, Batch-Normalization}-blocks, followed by the embedding layer.
The embedding layer predicts $\mu$ and $\log \sigma^2$ and performs the resampling by:
\begin{align}
    z &= \mu + \epsilon\sigma \\
    \epsilon &\sim \mathcal{N}(0, \bm{I}). \label{eq:resampling_vae}
\end{align}
The encoder input size is equal to the decoder output size and depends on the dataset.
The number of \say{Convolution, Activation, Batch-Normalization}-blocks is chosen depending on the input size, as smaller input sizes require fewer layers to achieve a receptive field of the input size.
The batch-normalization~~\citep[pp. 317, ff.]{Goodfellow-et-al-2016} can be omitted\footnote{It is stated in the experiments if batch-normalization is omitted.}.
The activation can be either ReLU~\citep[p. 173]{Goodfellow-et-al-2016} or LeakyReLU~\citep[p. 192]{Goodfellow-et-al-2016} and is ReLU unless stated otherwise.
The convolutions use zero-padding unless stated otherwise.
Encoder and decoder use stridden convolutions for downsampling, unless stated otherwise.

The \ac{VAE} model implements the loss function from Equation~\ref{eq:elbo_error_term} but with a pre-factor for the reconstruction term.
The reconstruction term pre-factor was determined empirically, observing reconstruction and generation quality.

The decoder uses similar blocks as the encoder but employs transposed convolutions~\citep[pp. 356, ff.]{Goodfellow-et-al-2016} instead of convolutions to upsample feature maps.
The output layer of the decoder uses a sigmoid activation instead of ReLU.

In total, eight \ac{VAE}-models are used: \say{\textsc{Mnist}-\ac{VAE}}, \say{dsprites-\ac{VAE}}, \say{7,500-\ac{VAE}}, \say{6,250-\ac{VAE}}, \say{5,000-\ac{VAE}},  \say{3,750-\ac{VAE}}, \say{dsprites-\ac{VAE}-dim6}, and \say{CelebA-\ac{VAE}}.
The model structures can be found in Appendix~\ref{subseq:appendix_vae_models}.

\paragraph{\textsc{Mnist}-\ac{VAE}} \textsc{Mnist}-\ac{VAE} uses an input- and output-size of $28\times 28\times 1$ (\textsc{Mnist} images are grayscale images).
The model is trained with the Adam optimizer on the \textsc{Mnist} training set with a batch size of 128 and a learning rate of 0.001 for 200 epochs.
The reconstruction loss factor is 10,000.
The latent space is two-dimensional.
The inner activation function is ReLU.

\paragraph{dSprites-\ac{VAE}} dSprites-\ac{VAE} uses an input- and output-size of $64\times 64\times 1$.
The model is trained with the Adam optimizer on a training set consisting of 90\% of the dSprites dataset with a batch size of 128 and a learning rate of 0.001 for 200 epochs.
The reconstruction loss factor is 10,000.
The latent space is ten-dimensional.
The inner activation function is ReLU.
For dSprites, four additional models have been trained with a different reconstruction term weight: 7,500-\ac{VAE}, 6,250-\ac{VAE}, 5,000-\ac{VAE}, and 3,750-\ac{VAE}.
These models differ from dSprites-\ac{VAE} only in the reconstruction term weight.

\paragraph{dsprites-VAE-dim6}
The dsprites-\ac{VAE}-dim6 is equivalent to the dSprites-\ac{VAE} model but uses a six-dimensional latent space.

\paragraph{CelebA-VAE} CelebA-\ac{VAE} uses an input- and output-size of $128\times 128\times 3$.
The model is trained with the Adam optimizer on a training set consisting of 90\% of the CelebA dataset with a batch size of 128 and a learning rate of 0.001 for 200 epochs.
The reconstruction loss factor is 10,000.
The latent space is eight-dimensional.
The inner activation function is ReLU.

\subsubsection{VLAE Models}\label{subsubsec:vlae_models}
\begin{figure}
    \centering
    \begin{subfigure}{.5\textwidth}
        \centering
        \includegraphics[width=\textwidth,height=.85\textheight,keepaspectratio]{images/vlae/encoder.png}
        \caption{Encoder}
        \label{subfig:vlae_encoder}
    \end{subfigure}%
    \begin{subfigure}{.5\textwidth}
        \centering
        \includegraphics[width=\textwidth,height=.85\textheight,keepaspectratio]{images/vlae/decoder.png}
        \caption{Decoder}
        \label{subfig:vlae_decoder}
    \end{subfigure}
    \caption{VLAE model structure}
    \label{fig:vlae_model_structure}
\end{figure}

Figure~\ref{fig:vlae_model_structure} shows the \ac{VLAE} model structure.
Like the \ac{VAE}, it consists of an encoder and a decoder.
The encoder has three latent spaces\footnote{\textit{z\_1\_latent}, \textit{z\_2\_latent}, and \textit{z\_3\_latent} in Figure~\ref{subfig:vlae_encoder}.}
A re-sampling according to Equation~\ref{eq:resampling_vae} is performed for each of the latent spaces.
Lower latent spaces are equipped with a less powerful encoder (e.g., \textit{z\_1\_latent} in Figure~\ref{subfig:vlae_encoder}), higher latent spaces with a more powerful encoder.
Again, the network is composed of multiple \say{Convolution, Activation, Batch-Normalization}-blocks.
The number of these blocks is variable and chosen depending on the dataset.
Batch-normalization can be omitted (default), the inner activation can be either ReLU (default) or LeakyReLU.
The convolutions use-zero padding, and encoder and decoder use stridden convolutions for downsampling.

The \ac{VAE} model implements the loss function from Equation~\ref{eq:elbo_error_term} but with a pre-factor for the reconstruction term.
The \ac{KL}-terms of different layers are totalized.

The decoder has three inputs, where the first input of the decoder\footnote{\textit{z\_3} in Figure~\ref{subfig:vlae_decoder}} receives input from the last output of the encoder\footnote{\textit{z\_3\_latent} in Figure~\ref{subfig:vlae_encoder}}.
The decoder uses blocks similar to the encoder but with transposed convolutions instead of regular convolutions.

In total, six \ac{VLAE}-models are used: \say{\textsc{Mnist}-\ac{VLAE}-factor-1}, \say{\textsc{Mnist}-\ac{VLAE}-factor-2}, \say{\textsc{Mnist}-\ac{VLAE}-factor-3} \say{dSprites-\ac{VLAE}}, \say{dSprites-\ac{VLAE}-dim2}, and \say{CelebA-\ac{VLAE}}.
The model structures can be found in Appendix~\ref{subseq:appendix_vlae_models}.

\paragraph{\textsc{Mnist}-VLAE} The three \textsc{Mnist}-\ac{VLAE}s\footnote{\say{\textsc{Mnist}-\ac{VLAE}-factor-1}, \say{\textsc{Mnist}-\ac{VLAE}-factor-2}, \say{\textsc{Mnist}-\ac{VLAE}-factor-3}} use an input- and output-size of $28\times 28\times 1$.
The models are trained with the Adam optimizer on the \textsc{Mnist} training set with a batch size of 128 for 200 epochs.
The reconstruction loss factor is 10,000.
The latent spaces are two-dimensional.
The inner activation function is ReLU.
The models use no batch-normalization.
\textsc{Mnist}-\ac{VLAE}-factor-1 is the model with the original number of feature maps, for \textsc{Mnist}-\ac{VLAE}-factor-2 and \textsc{Mnist}-\ac{VLAE}-factor-3, the number of feature maps is reduced according to the factor.
\textsc{Mnist}-\ac{VLAE}-factor-1 is trained with a learning rate of 0.005.
\textsc{Mnist}-\ac{VLAE}-factor-2 and \textsc{Mnist}-\ac{VLAE}-factor-3 are trained with a learning rate of 0.001.

\paragraph{dSprites-\ac{VLAE}} dSprites-\ac{VLAE} uses an input- and output-size of $64\times 64\times 1$.
The model is trained with the Adam optimizer on a training set consisting of 90\% of the dSprites dataset with a batch size of 128 and a learning rate of 0.001 for 200 epochs.
The reconstruction loss factor is 10,000.
The latent spaces are four-dimensional.
The inner activation function is ReLU.

\paragraph{dSprites-\ac{VLAE}-dim2}
The dSprites-\ac{VLAE}-dim2 model is equivalent to dSprites-\ac{VLAE} but uses a two-dimensional latent space.

\paragraph{CelebA-\ac{VLAE}} CelebA-\ac{VLAE} uses an input- and output-size of $128\times 128\times 3$.
The model is trained with the Adam optimizer on a training set consisting of 90\% of the CelebA dataset with a batch size of 128, a learning rate of 0.001 with an additional learning rate decay of 0.01 for 200 epochs.
The reconstruction loss factor is 10,000.
The latent spaces are two-dimensional.
The inner activation function is ReLU.

\subsubsection{VAE-GAN Models}\label{subsubsec:vae_gan_models}

The \ac{VAE}-\ac{GAN}-model is similar to the \ac{VAE}-model.
However, it implements the \ac{VAE}-\ac{GAN} loss function (see Section~\ref{subsubsec:representation_learning}) instead of Equation~\ref{eq:elbo_error_term}.
Therefore, the \ac{VAE}-\ac{GAN} has an additional \textit{discriminator} network.
The feature loss compares inner activations in the discriminator of true and generated samples.
The discriminator loss signifies by how much the discriminator violates the \ac{GAN} training objective.
See Section~\ref{subsubsec:representation_learning} for more details.

For the encoder, the \ac{KL}-term is weighted ten-times more strongly than the feature loss.
The decoder weights the discriminator loss with factor 1 and the feature loss with factor 0.75.

In total, seven \ac{VAE}-\ac{GAN}-models are used: \say{\textsc{Mnist}-\ac{VAE}-\ac{GAN}}, \say{dSprites-\ac{VAE}-\ac{GAN}}, and \say{CelebA-\ac{VAE}-\ac{GAN}}.
The model structures can be found in Appendix~\ref{subseq:appendix_vae_gan_models}.

\paragraph{\textsc{Mnist}-\ac{VAE}-\ac{GAN}} \textsc{Mnist}-\ac{VAE}-\ac{GAN} uses an input- and output-size of $28\times 28\times 1$.
The model is trained with the Adam optimizer on the \textsc{Mnist} training set with a batch size of 128 and a learning rate of 0.0001 for 200 epochs.
The reconstruction loss factor is 10,000.
The latent space is two-dimensional.
The inner activation function is ReLU.

\paragraph{dSprites-\ac{VAE}-\ac{GAN}} dSprites-\ac{GAN} uses an input- and output-size of $64\times 64\times 1$.
The model is trained with the Adam optimizer on a training set consisting of 90\% of the dSprites dataset with a batch size of 128 and a learning rate of 0.0001 for 200 epochs.
The reconstruction loss factor is 10,000.
The latent space is ten-dimensional.
The inner activation function is ReLU.

\paragraph{CelebA-\ac{VAE}-\ac{GAN}} CelebA-\ac{VAE}-\ac{GAN} uses an input- and output-size of $128\times 128\times 3$.
The model is trained with the Adam optimizer on a training set consisting of 90\% of the CelebA dataset with a batch size of 128 and a learning rate of 0.001 with an additional learning rate decay of 0.02 for 200 epochs.
The reconstruction loss factor is 10,000.
The latent space is eight-dimensional.
The inner activation function is ReLU.

\subsubsection{VLAE-GAN Models}\label{subsubsec:vlae_gan_models}

The \ac{VLAE}-\ac{GAN}-model is similar to the \ac{VAE}-\ac{GAN}-model in terms of the loss functions.
However, it uses the structure of the \ac{VLAE}-model and totalizes the three $KL$-losses from the different layers.

The model structures can be found in Appendix~\ref{subseq:appendix_vlae_gan_models}.

\paragraph{\textsc{Mnist}-\ac{VLAE}-\ac{GAN}} The three \textsc{Mnist}-\ac{VLAE}-\ac{GAN} uses an input- and output-size of $28\times 28\times 1$.
The model is trained with the Adam optimizer on the \textsc{Mnist} training set with a batch size of 128 and a learning rate of 0.0001 for 200 epochs.
The reconstruction loss factor is 10,000.
The latent spaces are two-dimensional.
The inner activation function is ReLU.
The model uses no batch-normalization.

\paragraph{dSprites-\ac{VLAE}-\ac{GAN}} dSprites-\ac{VLAE}-\ac{GAN} uses an input- and output-size of $64\times 64\times 1$.
The model is trained with the Adam optimizer on a training set consisting of 90\% of the dSprites dataset with a batch size of 128, a learning rate of 0.0001, and an additional learning rate decay of 0.01 for 200 epochs.
The reconstruction loss factor is 10,000.
The latent spaces are four-dimensional.
The inner activation function is ReLU.

\paragraph{CelebA-\ac{VLAE}-\ac{GAN}} CelebA-\ac{VLAE}-\ac{GAN} uses an input- and output-size of $128\times 128\times 3$.
The model is trained with the Adam optimizer on a training set consisting of 90\% of the CelebA dataset with a batch size of 128, a learning rate of 0.0001 with an additional learning rate decay of 0.01 for 200 epochs.
The reconstruction loss factor is 10,000.
The latent spaces are two-dimensional.
The inner activation function is ReLU.

\subsubsection{AlexNet Classifier}\label{subsubsec:alexnet_classifier}
The AlexNet Classifier resembles the architecture from \citet{krizhevsky2012imagenet}.
It uses dropout and a dropout rate of 0.3.
The model is trained with the Adam optimizer and a learning rate of 0.0001 using batch normalization and a batch size of 32 for ten epochs.
The model structure can be found in Appendix~\ref{subseq:appendix_alexnet_classifiers}.

\subsubsection{AlexNet-VAE}\label{subsubsec:alexnet_vae}
The AlexNet-\ac{VAE} resembles the AlexNet classifier but uses a 2000-dimensional latent space with re-sampling (see Equation~\ref{eq:resampling_vae}).
For AlexNet-\ac{VAE}, no dropout is used.
The model is trained with the Adam optimizer and a learning rate of 0.0001 using batch normalization and a batch size of 32 for 100 epochs..
The model structure can be found in Appendix~\ref{subseq:appendix_alexnet_vae}.


    \section{Results}\label{sec:results}

    \subsection{Visual Features in Variational Autoencoders}\label{subsec:results_visual_features_in_variational_autoencoders}
    The following sections describe the results of the experiments on visual features in variational autoencoders.

    \begin{itemize}
        \item Results non-AlexNet-like-VAE on CelebA: especially layer 1 kernels
        \item Results AlexNet-like-VAE on ImageNet: especially layer 1 kernels
        \item Results AlexNet-like image classification CNN on ImageNet: especially layer 1 kernels
        \item Results AlexNet-like-VAE with frozen decoder from classification network
    \end{itemize}

    \subsubsection{AlexNet Image Classification}
    To make sure that the network structure of the \ac{VAE} is apt to generate Gabor wavelets, image classification was performed on ImageNet (see Figure~\ref{fig:alexnet} and Section~\ref{subsec:visual-features-variational-autoencoders} for the used network, see Section~\ref{ssec:imagenet} for ImageNet).
    As the accuracy was not of primary concern, the training was stopped during the tenth epoch (\textbf{If time: train all 100 epochs}), the top-1 training accuracy at this time was at 0.85.

    \begin{figure}
        \centering
        \includegraphics[width=0.9\textwidth]{images/alexnet_classification_l1_kernels.png}
        \caption[Image classification - Layer 1 Kernels]{Convolutional Kernels in the first layer of the image classification network. The filters are shown in their original size (11x11).}
        \label{fig:classification_layer1_kernels}
    \end{figure}

    Figure~\ref{fig:classification_layer1_kernels} shows all 256 convolutional kernels of the image classification network.
    It is easy to see that in many kernels, Gabor wavelet-like filters emerge.

    \subsection{Activity Correlation in \acp{VAE}}\label{subsec:results_activity-correlation-in-vaes}

    \subsection{Feature Map Stripes}\label{subsec:feature-map-stripes}

    \begin{figure}
        \centering
        \begin{subfigure}{0.3\textwidth}
            \centering
            \includegraphics[width=\textwidth]{images/stripes/original.jpg}
            \caption{The original image.}
            \label{subfig:stripes_original}
        \end{subfigure}
        \hfill
        \begin{subfigure}{0.3\textwidth}
            \centering
            \includegraphics[width=\textwidth]{images/stripes/leaky_re_lu_5.png}
            \caption{The feature maps after LeakyReLU 5}
            \label{subfig:lakyrelu5}
        \end{subfigure}
        \hfill
        \begin{subfigure}{0.3\textwidth}
            \centering
            \includegraphics[width=\textwidth]{images/stripes/max_pooling2d_3.png}
            \caption{The feature maps after max pooling of LeakyReLU 5}
            \label{subfig:maxpool}
        \end{subfigure}
        \caption{The original image and the feature maps after passing the image through the network until after the specified layer. The stripe artifacts can be observed in many feature maps in Subfigure~\ref{subfig:lakyrelu5}. They vanish after max-pooling (Subfigure~\ref{subfig:maxpool}).}
        \label{fig:stripes}
    \end{figure}


    One observation being made during the analysis of the networks was the emergence of striped artifacts in the networks feature maps (Figure~\ref{fig:stripes}).
    These stripes were observed in the Vanilla VAE, AlexNetVAE, as well as the AlexNet Classifier and might be also present in other networks (\textbf{CHECK THIS}).
    Apparently, the stripes are either always horizontal or vertical for one network type.
    If they are vertical, they can appear on the left or the right side of the same network.
    If they are horizontal, they appear either on the top or on the bottom of the same network.
    The exact reason why the networks show this behavior was not found, however some insights were won.

    \paragraph{Padding}
    The stripes occur due to the zero-padding in the network and the resulting contrast observed by the convolutional filters.
    Take Figure~\ref{fig:stripes}.
    Here, the stripes appear strongest on the bottom left of the image.
    Noteworthy, the stripes indicate less active regions in the feature map: The feature map has an activity of around zero everywhere except for the location of the stripes.
    Here, the activity is strongly negative.

    A comparison with the original image (Figure~\ref{subfig:stripes_original}) shows that for the left side of the image, the contrast is highest on the bottom if the image is zero-padded\footnote{Zero-padding can be understood as adding black pixels around the image.}.
    Importantly, the contrast is as high on the bottom of the image, the right side, and the right side of the top of the image.
    However, for this network, the stripes seem to occur for a sharp shift of black on the left to white on the right.

    \begin{figure}
        \centering
        \foreach \n in {0,...,11}{
            \begin{subfigure}{0.05\textwidth}
                \frame{\includegraphics[width=\textwidth]{images/stripes/test_images/original\n.jpg}}
                \caption{}
                \label{subfig:test_images_stripes\n}
            \end{subfigure}
            \hfill
        }
        \caption{The test images used to analyze the networks behavior.}
        \label{fig:test_images_stripes}
    \end{figure}

    To better understand the behavior, the network was applied to a set of artifical test images (Figure~\ref{fig:test_images_stripes}).
    All different feature maps with respect to each image can be found in the appendix.

    \begin{figure}
        \centering
        \begin{subfigure}{0.45\textwidth}
            \centering
            \includegraphics[width=\textwidth]{images/stripes/test_img_9/leaky_re_lu_5.png}
            \caption{The original image.}
            \label{subfig:stipes_test_img_leakyrelu5}
        \end{subfigure}
        \hfill
        \begin{subfigure}{0.45\textwidth}
            \centering
            \includegraphics[width=\textwidth]{images/stripes/test_img_9/max_pooling2d_3.png}
            \caption{The feature maps after LeakyReLU 5}
            \label{subfig:stripes_test_img_maxpool3}
        \end{subfigure}
        \caption{The feature maps with respect to test image~\ref{subfig:test_images_stripes8}.}
        \label{fig:stripes_test_img}
    \end{figure}

    Subfigure~\ref{subfig:test_images_stripes8} and~\ref{subfig:test_images_stripes11} turned to be most equally high insightful.
    Figure~\ref{fig:stripes_test_img} shows the feature map after the last LeakyReLU (LeakyReLU 5) activation of the network given this input image (Subfigure~\ref{subfig:stipes_test_img_leakyrelu5}) as well as the feature map after max-pooling of this feature map (Subfigure~\ref{subfig:stripes_test_img_maxpool3}).
    Multiple things can be observed in Subfigure~\ref{subfig:stipes_test_img_leakyrelu5}.
    Firstly, the feature map in the first column of the fifth row resembles the input stimulus itself.
    The fact that this feature map shows most of the activity (especially after max-pooling, see Subfigure~\ref{subfig:stripes_test_img_maxpool3}) has been observed for natural images as well, however not the resemblance of the input stimulus.
    Except for this feature map, stripes emerge either on the bottom of the left side or the bottom of the right side of the feature map.
    For the bottom of the left side, this is where the sharp black-white contrast is.
    The bottom of the right side is more complicated.
    Here the test image was black and the \say{contrast} is a black-black contrast - or no contrast at all.
    However, this is only true for the first feature map\footnote{The first feature maps are not shown here.}.
    The following feature maps, again, are zero-padded.
    However, due to the bias term in the convolutions, these might be non-zero in the bottom-right and top-left square of the image, thus leading to a contrast.
    This explains why the network can be sensitive towards these black-black contrasts in the input image.


    \section{Discussion}\label{sec:discussion}

    \subsection{Visual Features in Variational Autoencoders}\label{subsec:discussion_visual_features_in_variational_autoencoders}
    \begin{itemize}
        \item Reconstructions of AlexNet-VAE don't look natural at all, this might be because ImageNet, unlike CelebA, is not properly aligned and contains images from a large variety of classes.
        This might be another hint towards the assumption that \acp{VAE} are no suitable model for semantic representations
        \item How does it come that supervised models seem to be a good representation of \ac{IT} cortical representation~\citep{khaligh2014deep}? Maybe the brain learns in a \say{supervised} manner in the sense that if you have never seen a horse, you need someone to tell you that that is a horse in order to make sense out of it.
        However, you might, at least, be able to recognize that it is an animal.
        Also, if someone tells you \say{This is a horse.}, you'll be able to recognize horses from different orientations and in different body positions without further training.
        This is quite different from how \acp{CNN} capture animals and is a strong hint to that supervised \acp{CNN} alone a no sufficient model for cortical IT representations.
    \end{itemize}


    \section{Conclusion}\label{sec:conclusion}

    \newpage
    \printbibliography

    \newpage
    \pagenumbering{Roman}
    \setcounter{page}{\thesavepage}
    \section*{Acronyms}
    \begin{acronym}[TDMA]
        \acro{VAE}{Variational Autoencoder}
        \acrodefplural{VAE}{Variational Autoencoders}
        \acro{CNN}{Convolutional Neural Network}
        \acrodefplural{CNN}{Convolutional Neural Networks}
        \acro{IT}{Inferior Temporal Cortex}
        \acro{ReLU}{Rectified Linear Unit}
        \acro{LeakyReLU}{Leaky Rectified Linear Unit}
        \acro{ILSVRC2017}{Large Scale Visual Recognition Challenge 2017}
        \acro{MSE}{Mean Squared Error}
        \acro{KL-divergence}{Kullback-Leibler divergence}
        \acro{ELBO}{Evidence Lower Bound}
        \acro{PDF}{Probability Density Function}
        \acro{MSE}{Mean Squared Error}
        \acro{GAN}{Generative Adversarial Network}
    \end{acronym}
    \newpage
    \listoffigures
    \newpage
    \listoftables
    \newpage
    \section*{Erklärung}

Ich erkläre, dass das Thema dieser Arbeit nicht identisch ist mit dem Thema einer von mir bereits für eine andere Prüfung eingereichte Arbeit.\par
Ich erkläre weiterhin, dass ich die Arbeit nicht bereits an einer anderen Hochschule zur Erlangung eines akademischen Grades eingereicht habe.\par
\vspace{2cm}
Ich versichere, dass ich die Arbeit selbstständig verfasst und keine anderen als die angegebenen Quellen benutzt habe. Die Stellen der Arbeit, die anderen Werken dem Wortlaut oder dem Sinn nach entnommen sind, habe ich unter Angabe der Quellen der Entlehnung kenntlich gemacht, Dies gilt sinngemäß auch für gelieferte Zeichnungen, Zkizzen, bildliche Darstellungen und dergleichen.

\vfill

\hspace{2cm} Ort, Datum \hfill Unterschrift \hspace{2cm}


\end{document}
