\documentclass[11pt]{article}
\usepackage[utf8]{inputenc}
\usepackage{graphicx}
\usepackage{geometry}
\usepackage{parskip}
\usepackage{subcaption}
\usepackage{wrapfig}
\usepackage{acronym}
\usepackage[natbib=true]{biblatex}
\usepackage{amsmath}
\usepackage{dirtytalk}
\usepackage{hyperref}
\usepackage{mathtools}
\usepackage{amsfonts}
\usepackage{csquotes}
\usepackage{pgffor}
\usepackage{algorithm}
\usepackage{algpseudocode}
\usepackage{booktabs}
\usepackage{listings}
\usepackage{bm}
\usepackage{amssymb}
\usepackage{tikz}
\usepackage{lscape}
\usetikzlibrary{shapes,positioning,decorations.pathreplacing}
\addbibresource{literature.bib}
\renewcommand{\baselinestretch}{1.5}
\let\oldsection\section
\renewcommand


\section{\clearpage\oldsection}
\newcounter{savepage}

\lstset{lineskip=-0.7ex,frame=single,breaklines=true}

\DeclarePairedDelimiterX{\infdivx}[2]{(}{)}{%
#1\;\delimsize\|\;#2%
}
\newcommand{\kldiv}{D_{KL}\infdivx}
\DeclarePairedDelimiter{\norm}{\lVert}{\rVert}

\DeclareMathOperator*{\argmin}{arg\,min}
\DeclareMathOperator*{\argmax}{arg\,max}

\makeatletter
\newenvironment{breakablealgorithm}
{% \begin{breakablealgorithm}
\begin{center}
\refstepcounter{algorithm}% New algorithm
    \vspace{1em}
    \hrule height.8pt depth0pt \kern2pt% \@fs@pre for \@fs@ruled
    \renewcommand{\caption}[2][\relax]{% Make a new \caption
    {\raggedright\textbf{\ALG@name~\thealgorithm} ##2\par}%
    \ifx\relax##1\relax % #1 is \relax
    \addcontentsline{loa}{algorithm}{\protect\numberline{\thealgorithm}##2}%
    \else % #1 is not \relax
    \addcontentsline{loa}{algorithm}{\protect\numberline{\thealgorithm}##1}%
    \fi
    \kern2pt\hrule\kern4pt
    }
}{% \end{breakablealgorithm}
\kern2pt\hrule\relax% \@fs@post for \@fs@ruled
\vspace{1em}
\end{center}
}
\makeatother

\begin{document}

\begin{titlepage}
\centering
\includegraphics[width=0.25\textwidth]{rublogo.png}\par
{\scshape\huge\bfseries Semantic Representations in Variational Autoencoders as a Model of the Visual System \par}
{\scshape\large Schriftliche Prüfungsarbeit für die Master-Prüfung des Studiengangs Angewandte Informatik an der Ruhr-Universität Bochum\par}
\vspace{1em}
vorgelegt von\par
\vspace{2em}
Leonard Papenmeier\par 108017257755\par
\vspace{2em}
01.01.1980\par

\vfill
Prof. Dr. Laurenz Wiskott\par
M.Sc. Zahra Fayyaz


\end{titlepage}
\pagenumbering{Roman}
\tableofcontents
\newpage
\setcounter{savepage}{\arabic{page}}
\pagenumbering{arabic}


\section{Introduction}\label{sec:introduction}

Research in artificial neural networks has risen in recent years due to their success in a large variety of different tasks.
This increase in research has led to progressively better network architectures, achieving better and better results.
The improvement of network architectures, however, is mainly driven by the question whether a new improvement will lead to better results \citep{lindsay2020convolutional}.

Biological plausibility, an important consideration in the field of computational neuroscience, often is of no interest to researchers aiming at solving a complex problem.
Even though progress in neural network research has become more and more disconnected from the biological example, they are inspired by biology.
Modern neural networks operating on images (mainly \acp{CNN}) indeed share many features with the visual system.

Even though neural networks are trained in a manner largely detached from the way the brain learns and builds memories, it has been shown that image classification networks are related to the visual system, not only areas where this relatedness has been built into the model.

Both, the biological foundation of \acp{CNN} and the more recent insights on their more indirect relatedness to the visual system qualify them as a potentially good model of the visual system.

Unfortunately, the neural networks where this kind of relatedness has been discovered so far are trained in a supervised manner, requiring lots of labeled data.
This kind of learning is disconnected from human perception where often one single example is sufficient when it comes to grouping objects into classes.

The \ac{VAE} is one important representative of a class of models \textit{generative models}.
This class of models allows a different training procedure, still requiring many samples but no labels.
Furthermore, \acp{VAE} build generalizable latent representations of inputs which might be similar to the abstract representation of the brain when perceiving the world.

\acp{VAE}, therefore, could be an important step towards a more realistic model of the visual system.

This thesis aims at answering the question whether \acp{VAE} prove to be a successful model of the visual system.

\section{Theoretical Background}\label{sec:theoretical-background}
\subsection{Primary Visual Cortex}\label{subsec:primary-visual-cortex}

\subsection{Visual Object Perception}\label{subsec:visual-object-perception}

\begin{wrapfigure}[12]{r}{0.3\textwidth}
    \begin{center}
        \includegraphics[width=0.28\textwidth]{images/rubens_sketches.jpg}
    \end{center}
    \caption[Copies of line drawings]{\say{Copies of line drawings.} taken from \citet{rubens1971associative}}
    \label{fig:copies_line_drawings}
\end{wrapfigure}
Recognizing an object as what it is is different from the ability of seeing an object or making a copy of it.
\citet{rubens1971associative} report the case of a 47-year old man who, on March 5--1969, \say{was found unconscious with vomitus on his face and bathrobe}.
Only after \say{his breathing became irregularly}, he was taken to a hospital where a low blood pressure was diagnosed.
The man showed an inability to recognize objects and in cases where he was unable to recognize an object, he also could not describe its use.
When given the category of an object, \say{identification improved very slightly}.
He claimed to recognize the item after being told the name.
In such cases, he was able to \say{point out various parts of the previously unrecognized item}.
When shown sketches of items, he was generally unable to recognize the items.
However, he was able to name geometric shapes such as circles or squares present in the sketch.
Even though the man did not recognize the objects, he was able to make copies of them (see Figure~\ref{fig:copies_line_drawings}).
\citet{rubens1971associative} report, that the Patient \say{was unable to identify any [items] before copying}.
However, he was able to contain some of the objects categories after copying them.

The example presented above shows that the ability to reproduce an object is different from the ability to \textit{perceive} it.

For monkeys, the \ac{IT} is assumed to be the brain region being crucial for object perception~\citep[pp. 1070, 1071]{squire2012fundamental}.
Bilateral lesions of the \ac{IT} in monkeys affect their ability to \say{distinguish between different visual patterns or objects, and in retaining previously acquired visual discriminations}~\citep[p. 1070]{squire2012fundamental}.
They are no longer able to generalize from tasks learned in one half of the visual space to the other half, presumably because the invariance of representations is lost~\citep[p. 1070]{squire2012fundamental}.
\citet[p. 1071]{squire2012fundamental} explicitly point out \say{the crucial role of the inferior temporal cortex during object perception and recognition}.

\subsection{Variational Autoencoders}\label{subsec:variational-autoencoders}

Since \acfp{VAE} are a specialization of the autoencoder, the traditional autoencoder is introduced first.

\subsubsection{Autoencoders}

Autoencoders are neural networks trained to reconstruct their input~\citep[p. 499]{Goodfellow-et-al-2016}.
For autoencoders, it is common to speak of an \textit{encoder}-part and a \textit{decoder}-part.
The encoder $f: \mathbb{R}^n \mapsto \mathbb{R}^m$ transforms an input $\mathbf{x}$ to a hidden representation $\mathbf{r} = f(\mathbf{x})$.
Usually $m < n$, i.e.\ the encoder transforms the input to a lower-dimensional representation.
This can be beneficial for dimensionality reduction or feature learning~\citep[p. 499]{Goodfellow-et-al-2016}.
The decoder $g: \mathbb{R}^m \mapsto \mathbb{R}^n$ transforms the hidden representation back into the original feature space.
Usually, one wants the reconstruction $\tilde{x}$ to be close to the original feature $x$ ($\tilde{x} \approx x$).
In order to achieve this, the autoencoder is usually trained by minimizing $\mathcal{L}(\mathbf{x}, g(f(\mathbf{x})))$ with
\begin{align}
    \mathcal{L}: \mathbb{R}^n \times \mathbb{R}^n \mapsto \mathbb{R}
\end{align}
One common choice for $\mathcal{L}$ is the \ac{MSE} which is defined as
\begin{align}
    \mathcal{L}(\mathbf{x}, \mathbf{y}) = \frac{1}{n}\sum (\mathbf{x}_i - \mathbf{y}_i)^2
\end{align}~\citep[p. 106]{Goodfellow-et-al-2016}.

\subsubsection{Variational Autoencoders}

\subsection{Visual Features in Neural Networks}\label{subsec:visual_features_in_neural_networks}
\begin{itemize}
    \item~\cite{krizhevsky2012imagenet} report Gabor wavelets in \acp{CNN} trained on image classification
\end{itemize}

\subsection{Semantic Representations}\label{subsec:semantic-representations}

\subsubsection{Supervised Models}
\begin{itemize}
    \item \citet{khaligh2014deep} found evidence for that supervised models may explain \ac{IT} cortical representation.
\end{itemize}

\subsubsection{Unsupervised Models}
\begin{itemize}
    \item \citet{han2019variational} found no evidence for or against Gabor wavelets in \acp{VAE} due to too small kernel size
    \item \citet{khaligh2014deep} found evidence against the assumption that unsupervised models might explain \ac{IT} cortical representation, however not explicitly for \acp{VAE}
\end{itemize}


\section{Methods}\label{sec:methods}
The following Sections describe the methods used in the course of this thesis.

\subsection{Research Questions}\label{subsec:research-questions}

The studies in Section~\ref{sec:results} are guided by the following research questions:

\begin{table}[H]
    \begin{tabularx}{\textwidth}{llX}
        \toprule \\
        \multicolumn{2}{l}{Number} & Question \\
        \midrule \\
        RQ1 &    & Are \acp{VAE} or \acp{VLAE} related to the visual cortex in terms of \ldots                                      \\
        & a) & \hspace{1cm} \ldots the emergence of Gabor wavelets?                                                             \\
        & b) & \hspace{1cm} \ldots sparse coding?                                                                               \\
        RQ2 &    & Do \acp{VAE} or \acp{VLAE} fulfil the requirements of latent space disentanglement or latent space separability? \\
        RQ3 &    & Can \acp{VAE} or \acp{VLAE} represent both continuous and categorical factors of variation in the latent space?  \\
        RQ4 &    & How do \acp{VAE} and \acp{VLAE} represent lower factors of variation in the latent space?                        \\
        RQ5 &    & Do \acp{VAE} or \acp{VLAE} learn independent factors of variation independently in the latent space?             \\
        RQ6 &    & Are the latent spaces of \acp{VLAE} independent in terms of generated images?                                    \\
        RQ7 &    & Do \acp{VAE}/\acp{VLAE}-generated images resemble the data distribution?                                         \\
        RQ8 &    & Is the discriminative loss superior to the pixel-wise loss in terms of the previous research questions?          \\
        \bottomrule
    \end{tabularx}
    \caption{Research Questions}
    \label{tbl:research_questions}
\end{table}

\subsection{Implementation Details}\label{subsec:implementation-details}

All models are implemented with Keras\footnote{\href{https://keras.io/}{https://keras.io/}, last access: 07/01/2020} in Version 2.2.4 using the TensorFlow\footnote{\href{https://www.tensorflow.org/}{https://www.tensorflow.org/}, last access: 07/01/2020} backend in Version 1.15..
The models are trained on Tesla V100-DGXS GPUs with 16GB of RAM.
The model code can be found under \href{https://github.com/LeoIV/master-thesis-leonard}{https://github.com/LeoIV/master-thesis-leonard}.

\subsection{Datasets}\label{subsec:datasets}

Five different datasets were used to train the models.
Four of the datasets contain images of different sizes, the fifth dataset provides additional labels for one of the datasets.
The images were resized to match the models' expected input sizes using Lanczos interpolation~\citep[pp. 223, ff]{burger2009principles}.

\subsubsection{CelebA}\label{subsubsec:celeba_dataset}

\begin{wrapfigure}[14]{R}{0.3\textwidth}
    \begin{center}
        \includegraphics[width=0.28\textwidth]{images/celeba_sample_63.jpg}
    \end{center}
    \caption[CelebA dataset sample image]{A sample image from the CelebA dataset.}
    \label{fig:celeba_sample}
\end{wrapfigure}

The \textit{CelebA} dataset~\citep{liu2015faceattributes} consists of 202,599 RGB images of size 178 x 218 pixels representing celebrities, as well as 40 binary attributes.
The images belong to 10.177 unique identities\footnote{The identities are not revealed.} as well as five \say{landmark annotations}.
They are aligned and cropped resulting in images of same size always showing only one face (see Figure~\ref{fig:celeba_sample} for an example).
The landmark annotations give the positions of facial attributes in the image; the left and right eye, the nose, and the left and right corner of the mouth.
The binary attributes indicate if the image has certain characteristics, for example if the person wears eyeglasses, has black hair, is smiling\footnote{See \href{https://www.kaggle.com/jessicali9530/celeba-dataset\#list\_attr\_celeba.csv}{https://www.kaggle.com/jessicali9530/celeba-dataset\#list\_attr\_celeba.csv} for a complete list of the attributes, login required. Last access: 12/02/2020.}.

\subsubsection{ImageNet}\label{ssec:imagenet}

ImageNet\footnote{\href{http://image-net.org/}{http://image-net.org/}, last access: 12/02/2020.} is a large-scale \say{image database organized according to the WordNet hierarchy}~\citep{imagenet_cvpr09}.
It contains of over 14 million images as of February 2020.
According to WordNet\footnote{See \href{https://wordnet.princeton.edu/}{https://wordnet.princeton.edu/}, last access: 12/02/2020.}, the images are subdivided into groups called \say{synsets}~\citep{imagenet_cvpr09} on different levels of granularity.
For example, the group \textit{woman, adult female} is subordinated to \textit{person, individual, someone, somebody, mortal, soul} and is further subdivided into groups like \textit{old woman} or \textit{lady} \footnote{ImageNet 2011 Fall Release, \href{http://image-net.org/explore}{http://image-net.org/explore}, last access: 12/02/2020.}

A smaller version of ImageNet, commonly called \textit{ILSVRC2012} has been used for \ac{ILSVRC2017}~\citep{ILSVRC15}, consisting of approximately 1,3 million images from 1000 different classes, that were selected, such that \say{there is no overlap between synsets: for any synsets $i$ and $j$, $i$ is not an ancestor of $j$ in the ImageNet hierarchy}~\citep{imagenet_cvpr09}.

This curated version is commonly used as a baseline~\citep{krizhevsky2012imagenet,Szegedy_2015_CVPR}\footnote{See \href{https://paperswithcode.com/sota/image-classification-on-imagenet}{https://paperswithcode.com/sota/image-classification-on-imagenet} for an overview of models on ImageNet. Last access: 07/16/2020}.

\subsubsection{\textsc{Mnist}}\label{subsubsec:mnist}

\begin{figure}
    \begin{center}
        \includegraphics[width=\textwidth]{images/mnist_sample.png}
    \end{center}
    \caption[\textsc{Mnist} dataset example images]{Examples from the \textsc{Mnist} dataset.}
    \label{fig:mnist_sample}
\end{figure}

\textsc{Mnist}\footnote{\href{http://yann.lecun.com/exdb/mnist/}{http://yann.lecun.com/exdb/mnist/}, last access: 23/04/2020}~\citep{lecun1998gradient} is a widely-used dataset of hand-written digits.
Figure~\ref{fig:mnist_sample} shows ten examples from this dataset.
The data is subdivided into a training set of 60.000 images and a test set containing 10.000 images.
The digits are all of the same size and centered.
The samples are grayscale images of size $28\times 28$pixels.

\subsubsection{Morpho-\textsc{Mnist}}\label{subsubsec:morphomnist}

\begin{figure}[H]
    \centering
    \includegraphics[width=\textwidth]{images/morpho_mnist_distribution.png}
    \caption[Morpho-\textsc{Mnist} distribution]{Distribution of the Morpho-\textsc{Mnist} attributes for the different digits. Taken from~\citep{castro2019morpho}.}
    \label{fig:morpho_mnist_distribution}
\end{figure}

Morpho-\textsc{Mnist}~\citep{castro2019morpho} is an extension of the \textsc{Mnist} dataset that addresses the question: \say{(T)o what extent has my model learned to represent specific factors of
variation in the data?} ~\citep{castro2019morpho}.
To address this questions, Morpho-\textsc{Mnist} provides the following (continuous) labels of morphological attributes of the \textsc{Mnist} samples: stroke length, stroke thickness, slant, width, and height.

Besides providing additional labels of low-level \textsc{Mnist} attributes, Morpho-\textsc{Mnist} provides a toolbox to measure (i.e~ calculate the morphological labels) and perturb MNIST images.
The perturbation toolbox allows it to thin, thicken, swell, and to add fractures to an image.
Morpho-\textsc{Mnist} also provides pre-computed datasets that were built using the perturbation toolbox.

Importantly, the distribution of the morphological attributes partly is highly skewed (for example Thickness and Height, see Figure~\ref{fig:morpho_mnist_distribution}).

\subsubsection{dSprites}
dSprites\footnote{\href{https://github.com/deepmind/dsprites-dataset/}{https://github.com/deepmind/dsprites-dataset/}, last access: 5/28/2020}~\citep{dsprites17} is a dataset designed \say{to assess the disentanglement properties of unsupervised learning methods.}.
It contains 737,280 grayscale images of size $64\times 64$ pixels.
The images were generated from \say{6 ground truth independent latent factors}: color, shape, scale, orientation, $x$-position, and $y$-position.
The color is white in all images.
The shapes are: square, ellipse, and heart.
For the other factors, points are chosen evenly along their support: six values in $[0.5, 1]$ (scale), 40 values in $[0, 2\pi]$ (orientation), 32 values in $[0, 1]$ ($x$-position and $y$-position).
Each factor combination only occurs once in the data set.
The dataset also contains the factor labels for each image.

\subsection{Models}\label{subsec:models}

Eight different \ac{VAE}, six different \ac{VLAE}, four different \ac{VAE}-\ac{GAN}, and four different \ac{VLAE}-\ac{GAN} were evaluated in the course of this thesis.
Furthermore, two \say{AlexNet} models were used for some additional experiments.

The models vary depending on the dataset and are described in the following.
An overview is given in Table~\ref{tbl:model_overview}.
A more detailed description can be found in Appendix~\ref{sec:appendix_network_architectures}.
\begin{table}
    \centering
    \begin{tabular}{lrrrrr}
        \toprule
        model name              & dataset        & \parbox[t]{2cm}{\normalsize\raggedleft input/output\\size}       & \parbox[t]{2cm}{\normalsize\raggedleft latent\\space\\size} & \parbox[t]{2cm}{\normalsize\raggedleft reconstruction\\term weight} & \parbox[t]{2cm}{\normalsize\raggedleft feature\\map\\reduction factor} \\
        \midrule
        \textsc{Mnist}-\ac{VAE} & \textsc{Mnist} & $28\times 28\times 1$   & 2                 & 10,000                     & 1                            \\
        (dSprites/10,000)-\ac{VAE}       & dSprites       & $64\times 64\times 1$   & 10                & 10,000                     & 1                            \\
        7,500-\ac{VAE}          & dSprites       & $64\times 64\times 1$   & 10                & 7,500                      & 1                            \\
        6,250-\ac{VAE}          & dSprites       & $64\times 64\times 1$   & 10                & 6,250                      & 1                            \\
        5,000-\ac{VAE}          & dSprites       & $64\times 64\times 1$   & 10                & 5,000                      & 1                            \\
        3,750-\ac{VAE}          & dSprites       & $64\times 64\times 1$   & 10                & 3,750                      & 1                            \\
        dSprites-\ac{VAE}-dim6  & dSprites       & $64\times 64\times 1$   & 6                 & 10,000                     & 1                            \\
        CelebA-\ac{VAE}         & CelebA         & $128\times 128\times 3$ & 8                 & 3,750                      & 1                            \\
        \midrule
        \textsc{Mnist}-\ac{VLAE}(-factor-1) & \textsc{Mnist} & $28\times 28\times 1$ & 2,2,2 & 10,000 & 1 \\
        \textsc{Mnist}-\ac{VLAE}-factor-2 & \textsc{Mnist} & $28\times 28\times 1$ & 2,2,2 & 10,000 & 2 \\
        \textsc{Mnist}-\ac{VLAE}-factor-3 & \textsc{Mnist} & $28\times 28\times 1$ & 2,2,2 & 10,000 & 3 \\
        dSprites-\ac{VLAE} & dSprites & $64\times 64\times 1$ & 4,4,4 & 10,000 & 1 \\
        dSprites-\ac{VLAE}-dim2 & dSprites & $64\times 64\times 1$ & 2,2,2 & 10,000 & 1 \\
        CelebA-\ac{VLAE} & CelebA & $128\times 128\times 3$ & 2,2,2 & 10,000 & 1 \\
        \midrule
        \textsc{Mnist}-\ac{VAE}-\ac{GAN} & \textsc{Mnist} & $28\times 28\times 1$ & 2 & 10,000 & 1\\
        dSprites-\ac{VAE}-\ac{GAN} & dSprites & $64\times 64\times 1$ & 10 & 10,000 & 1\\
        CelebA-\ac{VAE}-\ac{GAN} & CelebA & $128\times 128\times 3$ & 8 & 10,000 & 1\\
        \midrule
        \textsc{Mnist}-\ac{VLAE}-\ac{GAN} & \textsc{Mnist} & $28\times 28\times 1$ & 2,2,2 & 10,000 & 1\\
        dSprites-\ac{VLAE}-\ac{GAN} & dSprites & $64\times 64\times 1$ & 4,4,4 & 10,000 & 1\\
        CelebA-\ac{VLAE}-\ac{GAN} & CelebA & $128\times 128\times 3$ & 2,2,2 & 10,000 & 1\\
        \midrule
        AlexNet Classifier & ImageNet & $224\times 224\times 3$ & -- & -- & 1\\
        \midrule
        AlexNet \ac{VAE} & ImageNet & $224\times 224\times 3$ & 2000 & 10,000 & 1\\
        \bottomrule
    \end{tabular}
    \caption[Models Overview]{Overview of all models with important parameters.}
    \label{tbl:model_overview}
\end{table}

\subsubsection{VAE Models}\label{subsubsec:vae_models}

\begin{figure}
    \centering
    \begin{subfigure}{.5\textwidth}
        \centering
        \includegraphics[width=\textwidth,height=.85\textheight,keepaspectratio]{images/vae/encoder.png}
        \caption{Encoder}
    \end{subfigure}%
    \begin{subfigure}{.5\textwidth}
        \centering
        \includegraphics[width=\textwidth,height=.85\textheight,keepaspectratio]{images/vae/decoder.png}
        \caption{Decoder}
    \end{subfigure}
    \caption{VAE model structure}
    \label{fig:vae_model_structure}
\end{figure}

The \ac{VAE} model (see Section~\ref{fig:vae_model_structure}) consists of an encoder and a decoder.
The encoder is made up of multiple \say{Convolution, Activation, Batch-Normalization}-blocks, followed by the embedding layer.
The embedding layer predicts $\mu$ and $\log \sigma^2$ and performs the resampling by:
\begin{align}
    z &= \mu + \epsilon\sigma \\
    \epsilon &\sim \mathcal{N}(0, \bm{I}). \label{eq:resampling_vae}
\end{align}
The encoder input size is equal to the decoder output size and depends on the dataset.
The number of \say{Convolution, Activation, Batch-Normalization}-blocks is chosen depending on the input size, as smaller input sizes require fewer layers to achieve a receptive field of the input size.
The batch-normalization~~\citep[pp. 317, ff.]{Goodfellow-et-al-2016} can be omitted\footnote{It is stated in the experiments if batch-normalization is omitted.}.
The activation can be either ReLU~\citep[p. 173]{Goodfellow-et-al-2016} or LeakyReLU~\citep[p. 192]{Goodfellow-et-al-2016} and is ReLU unless stated otherwise.
The convolutions use zero-padding unless stated otherwise.
Encoder and decoder use stridden convolutions for downsampling, unless stated otherwise.

The \ac{VAE} model implements the loss function from Equation~\ref{eq:elbo_error_term} but with a pre-factor for the reconstruction term.
The reconstruction term pre-factor was determined empirically, observing reconstruction and generation quality.

The decoder uses similar blocks as the encoder but employs transposed convolutions~\citep[pp. 356, ff.]{Goodfellow-et-al-2016} instead of convolutions to upsample feature maps.
The output layer of the decoder uses a sigmoid activation instead of ReLU.

In total, eight \ac{VAE}-models are used: \say{\textsc{Mnist}-\ac{VAE}}, \say{dsprites-\ac{VAE}}, \say{7,500-\ac{VAE}}, \say{6,250-\ac{VAE}}, \say{5,000-\ac{VAE}},  \say{3,750-\ac{VAE}}, \say{dsprites-\ac{VAE}-dim6}, and \say{CelebA-\ac{VAE}}.
The model structures can be found in Appendix~\ref{subseq:appendix_vae_models}.

\paragraph{\textsc{Mnist}-\ac{VAE}} \textsc{Mnist}-\ac{VAE} uses an input- and output-size of $28\times 28\times 1$ (\textsc{Mnist} images are grayscale images).
The model is trained with the Adam optimizer on the \textsc{Mnist} training set with a batch size of 128 and a learning rate of 0.001 for 200 epochs.
The reconstruction loss factor is 10,000.
The latent space is two-dimensional.
The inner activation function is ReLU.

\paragraph{dSprites-\ac{VAE}} dSprites-\ac{VAE} uses an input- and output-size of $64\times 64\times 1$.
The model is trained with the Adam optimizer on a training set consisting of 90\% of the dSprites dataset with a batch size of 128 and a learning rate of 0.001 for 200 epochs.
The reconstruction loss factor is 10,000.
The latent space is ten-dimensional.
The inner activation function is ReLU.
For dSprites, four additional models have been trained with a different reconstruction term weight: 7,500-\ac{VAE}, 6,250-\ac{VAE}, 5,000-\ac{VAE}, and 3,750-\ac{VAE}.
These models differ from dSprites-\ac{VAE} only in the reconstruction term weight.

\paragraph{dsprites-VAE-dim6}
The dsprites-\ac{VAE}-dim6 is equivalent to the dSprites-\ac{VAE} model but uses a six-dimensional latent space.

\paragraph{CelebA-VAE} CelebA-\ac{VAE} uses an input- and output-size of $128\times 128\times 3$.
The model is trained with the Adam optimizer on a training set consisting of 90\% of the CelebA dataset with a batch size of 128 and a learning rate of 0.001 for 200 epochs.
The reconstruction loss factor is 10,000.
The latent space is eight-dimensional.
The inner activation function is ReLU.

\subsubsection{VLAE Models}\label{subsubsec:vlae_models}
\begin{figure}
    \centering
    \begin{subfigure}{.5\textwidth}
        \centering
        \includegraphics[width=\textwidth,height=.85\textheight,keepaspectratio]{images/vlae/encoder.png}
        \caption{Encoder}
        \label{subfig:vlae_encoder}
    \end{subfigure}%
    \begin{subfigure}{.5\textwidth}
        \centering
        \includegraphics[width=\textwidth,height=.85\textheight,keepaspectratio]{images/vlae/decoder.png}
        \caption{Decoder}
        \label{subfig:vlae_decoder}
    \end{subfigure}
    \caption{VLAE model structure}
    \label{fig:vlae_model_structure}
\end{figure}

Figure~\ref{fig:vlae_model_structure} shows the \ac{VLAE} model structure.
Like the \ac{VAE}, it consists of an encoder and a decoder.
The encoder has three latent spaces\footnote{\textit{z\_1\_latent}, \textit{z\_2\_latent}, and \textit{z\_3\_latent} in Figure~\ref{subfig:vlae_encoder}.}
A re-sampling according to Equation~\ref{eq:resampling_vae} is performed for each of the latent spaces.
Lower latent spaces are equipped with a less powerful encoder (e.g., \textit{z\_1\_latent} in Figure~\ref{subfig:vlae_encoder}), higher latent spaces with a more powerful encoder.
Again, the network is composed of multiple \say{Convolution, Activation, Batch-Normalization}-blocks.
The number of these blocks is variable and chosen depending on the dataset.
Batch-normalization can be omitted (default), the inner activation can be either ReLU (default) or LeakyReLU.
The convolutions use-zero padding, and encoder and decoder use stridden convolutions for downsampling.

The \ac{VAE} model implements the loss function from Equation~\ref{eq:elbo_error_term} but with a pre-factor for the reconstruction term.
The \ac{KL}-terms of different layers are totalized.

The decoder has three inputs, where the first input of the decoder\footnote{\textit{z\_3} in Figure~\ref{subfig:vlae_decoder}} receives input from the last output of the encoder\footnote{\textit{z\_3\_latent} in Figure~\ref{subfig:vlae_encoder}}.
The decoder uses blocks similar to the encoder but with transposed convolutions instead of regular convolutions.

In total, six \ac{VLAE}-models are used: \say{\textsc{Mnist}-\ac{VLAE}-factor-1}, \say{\textsc{Mnist}-\ac{VLAE}-factor-2}, \say{\textsc{Mnist}-\ac{VLAE}-factor-3} \say{dSprites-\ac{VLAE}}, \say{dSprites-\ac{VLAE}-dim2}, and \say{CelebA-\ac{VLAE}}.
The model structures can be found in Appendix~\ref{subseq:appendix_vlae_models}.

\paragraph{\textsc{Mnist}-VLAE} The three \textsc{Mnist}-\ac{VLAE}s\footnote{\say{\textsc{Mnist}-\ac{VLAE}-factor-1}, \say{\textsc{Mnist}-\ac{VLAE}-factor-2}, \say{\textsc{Mnist}-\ac{VLAE}-factor-3}} use an input- and output-size of $28\times 28\times 1$.
The models are trained with the Adam optimizer on the \textsc{Mnist} training set with a batch size of 128 for 200 epochs.
The reconstruction loss factor is 10,000.
The latent spaces are two-dimensional.
The inner activation function is ReLU.
The models use no batch-normalization.
\textsc{Mnist}-\ac{VLAE}-factor-1 is the model with the original number of feature maps, for \textsc{Mnist}-\ac{VLAE}-factor-2 and \textsc{Mnist}-\ac{VLAE}-factor-3, the number of feature maps is reduced according to the factor.
\textsc{Mnist}-\ac{VLAE}-factor-1 is trained with a learning rate of 0.005.
\textsc{Mnist}-\ac{VLAE}-factor-2 and \textsc{Mnist}-\ac{VLAE}-factor-3 are trained with a learning rate of 0.001.

\paragraph{dSprites-\ac{VLAE}} dSprites-\ac{VLAE} uses an input- and output-size of $64\times 64\times 1$.
The model is trained with the Adam optimizer on a training set consisting of 90\% of the dSprites dataset with a batch size of 128 and a learning rate of 0.001 for 200 epochs.
The reconstruction loss factor is 10,000.
The latent spaces are four-dimensional.
The inner activation function is ReLU.

\paragraph{dSprites-\ac{VLAE}-dim2}
The dSprites-\ac{VLAE}-dim2 model is equivalent to dSprites-\ac{VLAE} but uses a two-dimensional latent space.

\paragraph{CelebA-\ac{VLAE}} CelebA-\ac{VLAE} uses an input- and output-size of $128\times 128\times 3$.
The model is trained with the Adam optimizer on a training set consisting of 90\% of the CelebA dataset with a batch size of 128, a learning rate of 0.001 with an additional learning rate decay of 0.01 for 200 epochs.
The reconstruction loss factor is 10,000.
The latent spaces are two-dimensional.
The inner activation function is ReLU.

\subsubsection{VAE-GAN Models}\label{subsubsec:vae_gan_models}

The \ac{VAE}-\ac{GAN}-model is similar to the \ac{VAE}-model.
However, it implements the \ac{VAE}-\ac{GAN} loss function (see Section~\ref{subsubsec:representation_learning}) instead of Equation~\ref{eq:elbo_error_term}.
Therefore, the \ac{VAE}-\ac{GAN} has an additional \textit{discriminator} network.
The feature loss compares inner activations in the discriminator of true and generated samples.
The discriminator loss signifies by how much the discriminator violates the \ac{GAN} training objective.
See Section~\ref{subsubsec:representation_learning} for more details.

For the encoder, the \ac{KL}-term is weighted ten-times more strongly than the feature loss.
The decoder weights the discriminator loss with factor 1 and the feature loss with factor 0.75.

In total, seven \ac{VAE}-\ac{GAN}-models are used: \say{\textsc{Mnist}-\ac{VAE}-\ac{GAN}}, \say{dSprites-\ac{VAE}-\ac{GAN}}, and \say{CelebA-\ac{VAE}-\ac{GAN}}.
The model structures can be found in Appendix~\ref{subseq:appendix_vae_gan_models}.

\paragraph{\textsc{Mnist}-\ac{VAE}-\ac{GAN}} \textsc{Mnist}-\ac{VAE}-\ac{GAN} uses an input- and output-size of $28\times 28\times 1$.
The model is trained with the Adam optimizer on the \textsc{Mnist} training set with a batch size of 128 and a learning rate of 0.0001 for 200 epochs.
The reconstruction loss factor is 10,000.
The latent space is two-dimensional.
The inner activation function is ReLU.

\paragraph{dSprites-\ac{VAE}-\ac{GAN}} dSprites-\ac{GAN} uses an input- and output-size of $64\times 64\times 1$.
The model is trained with the Adam optimizer on a training set consisting of 90\% of the dSprites dataset with a batch size of 128 and a learning rate of 0.0001 for 200 epochs.
The reconstruction loss factor is 10,000.
The latent space is ten-dimensional.
The inner activation function is ReLU.

\paragraph{CelebA-\ac{VAE}-\ac{GAN}} CelebA-\ac{VAE}-\ac{GAN} uses an input- and output-size of $128\times 128\times 3$.
The model is trained with the Adam optimizer on a training set consisting of 90\% of the CelebA dataset with a batch size of 128 and a learning rate of 0.001 with an additional learning rate decay of 0.02 for 200 epochs.
The reconstruction loss factor is 10,000.
The latent space is eight-dimensional.
The inner activation function is ReLU.

\subsubsection{VLAE-GAN Models}\label{subsubsec:vlae_gan_models}

The \ac{VLAE}-\ac{GAN}-model is similar to the \ac{VAE}-\ac{GAN}-model in terms of the loss functions.
However, it uses the structure of the \ac{VLAE}-model and totalizes the three $KL$-losses from the different layers.

The model structures can be found in Appendix~\ref{subseq:appendix_vlae_gan_models}.

\paragraph{\textsc{Mnist}-\ac{VLAE}-\ac{GAN}} The three \textsc{Mnist}-\ac{VLAE}-\ac{GAN} uses an input- and output-size of $28\times 28\times 1$.
The model is trained with the Adam optimizer on the \textsc{Mnist} training set with a batch size of 128 and a learning rate of 0.0001 for 200 epochs.
The reconstruction loss factor is 10,000.
The latent spaces are two-dimensional.
The inner activation function is ReLU.
The model uses no batch-normalization.

\paragraph{dSprites-\ac{VLAE}-\ac{GAN}} dSprites-\ac{VLAE}-\ac{GAN} uses an input- and output-size of $64\times 64\times 1$.
The model is trained with the Adam optimizer on a training set consisting of 90\% of the dSprites dataset with a batch size of 128, a learning rate of 0.0001, and an additional learning rate decay of 0.01 for 200 epochs.
The reconstruction loss factor is 10,000.
The latent spaces are four-dimensional.
The inner activation function is ReLU.

\paragraph{CelebA-\ac{VLAE}-\ac{GAN}} CelebA-\ac{VLAE}-\ac{GAN} uses an input- and output-size of $128\times 128\times 3$.
The model is trained with the Adam optimizer on a training set consisting of 90\% of the CelebA dataset with a batch size of 128, a learning rate of 0.0001 with an additional learning rate decay of 0.01 for 200 epochs.
The reconstruction loss factor is 10,000.
The latent spaces are two-dimensional.
The inner activation function is ReLU.

\subsubsection{AlexNet Classifier}\label{subsubsec:alexnet_classifier}
The AlexNet Classifier resembles the architecture from \citet{krizhevsky2012imagenet}.
It uses dropout and a dropout rate of 0.3.
The model is trained with the Adam optimizer and a learning rate of 0.0001 using batch normalization and a batch size of 32 for ten epochs.
The model structure can be found in Appendix~\ref{subseq:appendix_alexnet_classifiers}.

\subsubsection{AlexNet-VAE}\label{subsubsec:alexnet_vae}
The AlexNet-\ac{VAE} resembles the AlexNet classifier but uses a 2000-dimensional latent space with re-sampling (see Equation~\ref{eq:resampling_vae}).
For AlexNet-\ac{VAE}, no dropout is used.
The model is trained with the Adam optimizer and a learning rate of 0.0001 using batch normalization and a batch size of 32 for 100 epochs..
The model structure can be found in Appendix~\ref{subseq:appendix_alexnet_vae}.


\section{Results and Discussion}\label{sec:results}
\subsection{Gabor Wavelets in Variational Autoencoders}\label{subsec:results_visual_features_in_variational_autoencoders}
Simple and cells in \ac{V1} show the strongest excitation for Gabor wavelets (see Section~\ref{subsubseq:simple_complex_cells}).
As discussed earlier, the optimal stimulus for a convolutional kernel resembling a Gabor wavelet, is a Gabor wavelet itself (see Section~\ref{subsubsec:cnn_model_visual_system}).
Therefore, the emergence of Gabor-like filters is considered evidence for a model's biological plausibility~\citep{berkes2005slow}.
\citet{krizhevsky2012imagenet} found that AlexNet learns Gabor-like features in the first layer.
It was later found that supervised \acp{CNN} like AlexNet explain \ac{IT} activity~\citep{khaligh2014deep}.

Therefore, emergence of Gabor wavelets in \acp{VAE} would be evidence for their suitability as a model of the visual system.
Potentially, they would even indicate that higher layers explain activity of higher regions in the visual cortex.
Therefore, the following experiment was conducted.

Since it is known that AlexNet learns Gabor wavelets in lower-level filters, a \ac{VAE} network with a similar architecture (AlexNet-\ac{VAE}, see Section~\ref{subsubsec:alexnet_vae}) was designed.
The hypothesis is that if \acp{VAE} can learn Gabor wavelets, this model should show them because it is very similar to the supervised model for which they are learned.
First, it was verified that the supervised network (AlexNet Classifier, see Section~\ref{subsubsec:alexnet_classifier}) actually learns Gabor wavelets.
The network was trained for ten epochs, the top-1 training accuracy at this time was at 0.85.

\begin{figure}
    \centering
    \includegraphics[width=0.9\textwidth]{images/alexnet_classification_l1_kernels.png}
    \caption[Image classification - Layer 1 Kernels]{Convolutional Kernels in the first layer of the image classification network. The filters are shown in their original size (11x11).}
    \label{fig:classification_layer1_kernels}
\end{figure}

\begin{figure}
    \centering
    \includegraphics[width=0.9\textwidth]{images/alexnet_vae_l1_kernels.png}
    \caption[\ac{VAE} - Layer 1 Kernels]{Convolutional Kernels in the first layer of the AlexNet \ac{VAE} network. The filters are shown in their original size (11x11).}
    \label{fig:vae_layer1_kernels}
\end{figure}

Figure~\ref{fig:classification_layer1_kernels} shows all 288 ($96 \cdot 3$) convolutional kernels of the image classification network.
In many kernels, Gabor wavelet-like filters emerge.

However, the same effect cannot be observed in the kernels of the AlexNet-\ac{VAE} model.
Figure~\ref{fig:vae_layer1_kernels} shows the kernels of the first layer in the AlexNet \ac{VAE} network.
Moreover, no Gabor wavelets emerged in any of the \ac{VAE}-models or for any of the datasets.

It is therefore concluded that a simple feed-forward \ac{VAE} trained on static images probably cannot learn Gabor-wavelets.
This does not imply that \acp{VAE} do not explain cortical activity (this is discussed in Section~\ref{subsec:representational-dissimilarity-matrices}).
Furthermore, variations of the \ac{VAE}-design are proposed that could lead to a more biologically plausible model (see Sections~\ref{subsec:feedback-connections-of-the-lateral-geniculate-nucleus} and \ref{subsec:sequential-data}).

% TODO: show generations and reconstructions of AlexNet VAE

\subsection{Sparseness in Generative Models}\label{subsec:effective-network-capacity}
One hyperparameter to choose in the different models is the number of feature maps of one layer.
During the implementation, it could be observed that some feature maps show little activity compared to others.
As the number of features maps is a model hyperparameter, it was investigated by how much the number of feature maps can be reduced without increasing the network loss.

Furthermore, the question arose whether sparseness in \acp{VLAE} is related to sparseness in the primary visual cortex (see Section~\ref{subsubsec:sparse_representations}).
If the brain uses sparseness to represent information, do \acp{VLAE} use a similar mechanism?
If they do, this would be evidence for the relatedness of \acp{VLAE} to the biological example.

The following experiment was conducted to examine if \acp{VLAE} employ sparseness or if the model has only too much capacity.
The feature map activity of a \ac{VLAE} on \textsc{Mnist} with input size $28\times 28$ and different numbers of feature maps and nodes in fully connected layers was measured.
By gradually decreasing the network capacity, the evolvement of sparseness was analyzed.
In total, three models (\textsc{Mnist}-\ac{VLAE}-factor-1, \textsc{Mnist}-\ac{VLAE}-factor-2, and \textsc{Mnist}-\ac{VLAE}-factor-3, see Section~\ref{subsubsec:vlae_models}) were trained.
Batch normalization was disabled for this experiment, because it re-scales the feature map activity.

\begin{figure}
    \centering
    \begin{subfigure}{.95\textwidth}
        \centering
        % include first image
        \includegraphics[width=\textwidth]{images/sparseness/encoder_fm1_fms.png}
        \caption{Feature map activities - large model}
    \end{subfigure}
\end{figure}
\begin{figure}
    \ContinuedFloat
    \centering
    \begin{subfigure}{.95\textwidth}
        \centering
        % include second image
        \includegraphics[width=\textwidth]{images/sparseness/encoder_fm2_fms.png}
        \caption{Feature map activities - medium model}
    \end{subfigure}
\end{figure}
\begin{figure}
    \ContinuedFloat
    \centering
    \begin{subfigure}{.95\textwidth}
        \centering
        % include second image
        \includegraphics[width=\textwidth]{images/sparseness/encoder_fm3_fms.png}
        \caption{Feature map activities - small model}
    \end{subfigure}
    \caption{Feature map activities for the different models}
    \label{fig:fm_activities_sparseness}
\end{figure}

Listing~\ref{lst:sparsity-vlae-encoder-28-fm1} shows the initial \ac{VLAE} encoder used to analyze sparseness.

\begin{figure}
    \centering
    \begin{subfigure}{.45\textwidth}
        \centering
        % include first image
        \includegraphics[width=\textwidth]{images/sparseness/sparseness_train_loss.png}
        \caption{Training loss}
    \end{subfigure}
    \hfill
    \begin{subfigure}{.45\textwidth}
        \centering
        % include second image
        \includegraphics[width=\textwidth]{images/sparseness/sparseness_validation_loss.png}
        \caption{Validation loss}
    \end{subfigure}
    \caption[Sparse Models - Loss Curves]{Loss curves of models with different number of feature maps (see Appendix~\ref{sec:listings_sparsity_networks}). $fm$ is the reduction factor of the feature maps; $fm=1$, therefore, is the original model, $fm=2$ the model with half the number of feature maps, and $fm=3$ the model with one third of the feature maps.}
    \label{fig:learning_curves_sparseness}
\end{figure}

Figure~\ref{fig:learning_curves_sparseness} shows the training and validation losses for the models.
Firstly, increasing the number of feature maps increases convergence speed.
Even though the large model (the model with the original number of feature maps) is trained with a slightly lower learning rate, it converges faster than the smaller models.
It also achieves the lowest validation loss among the three models and achieves the minimum validation loss faster than the other models; it is minimal after 26 epochs for the large, 39 epochs for the medium, and 52 epochs for the smallest model.
After achieving the minimum validation loss, however, all models overfit.
The large model overfits stronger than the smaller models, leading to the highest validation loss after 200 epochs.

Figure~\ref{fig:fm_activities_sparseness} shows the activities of the feature maps in the different models.
Each line in a subplot is a boxplot, each subfigure the output of either a convolutional, activation, or fully connected layer.
Most prominently, the inner activations are sparse after convolutions.
The activations often increase the level of sparsity as the ReLU activation functions map negative values to zero.
Sparsity is apparent in almost all feature maps of all models.

\paragraph{Sparseness in \acp{VLAE} vs. Sparse Representations}
In the brain, the overlap of active neurons is small, i.e., different neurons are active for different stimuli (see Section~\ref{subsubsec:sparse_representations} and \citet{yoshida2020natural}).
Contrarily, the \acp{VLAE} show high activity mostly in the same feature maps regardless of the input.

The \ac{VLAE} uses sparseness, mostly regardless of the model capacity, which can be explained by the Lottery Ticket Hypothesis~\citep{frankle2018lottery}.
The Lottery Ticket Hypothesis states that, in general, it is highly unlikely that re-training a pruned network from scratch will result in the same weight configuration of the subnetwork in the initial model.
In the three models (small, medium, large), there seems to be a sub-network that accounts for the activity in the model.
Regardless of the reduction factor, there always remains an active sub-network for all three models.

Therefore, the sparseness in \acp{VLAE} does not seem to be related to sparse representations in the brain.

\subsection{Latent Space Entanglement and Categorical Factors of Variation}\label{subsec:latent-space-entanglement-and-categorical-factors-of-variation}

As discussed in Section~\ref{subsec:feature-disentanglement} and Section~\ref{subsubsec:representation_learning} ($\beta$-\ac{VAE}), in practice, there is a trade-off between reconstruction quality and prior-posterior matching.
Increasing the weight of the reconstruction term in the \ac{VAE} training objective (smaller $\beta$ in Equation~\ref{eq:beta-vae}\footnote{The $\beta$ in $\beta$-\ac{VAE} is the \ac{KL}-term weight. Therefore, decreasing $\beta$ analog to increasing the reconstruction term weight.}) leads to reconstructions more similar to the training samples.
The posterior distribution, however, will match the prior distribution less precisely.
Contrarily, increasing $\beta$ in Equation~\ref{eq:beta-vae} leads to a posterior closer to the prior distribution at the expense of reconstruction quality.
Moreover, \citet{higgins2017beta} claim that increasing $\beta$ leads to a better latent space disentanglement, i.e., each dimension in the latent space is uniquely correlated with a single factor of variation.

As discussed in Section~\ref{subsec:feature-disentanglement}, the quality of feature space disentanglement can be assessed by measuring how fast reconstructions change as the latent space is traversed.
However, a disentangled feature space should be problematic for datasets with categorical factors of variation, such as dSprites, where features change fast by definition.
Even though CelebA has binary labels as well (e.g., \say{brown hair}), there still is enough variation within hair colors to allow a smooth translation between hair colors~\citep{higgins2017beta}.
For dSprites, there are only three distinct shapes (\say{square}, \say{ellipse}, and \say{heart}).
To achieve a good reconstruction quality on these shapes, the model would have to place them in separate areas of the latent space, violating both feature space disentanglement and a Gaussian posterior.

To analyze how \acp{VAE} behave for categorical factors of variation, a \ac{VAE} with input size $64\times 64$ (the size of dSprites images) and a ten-dimensional latent space was trained on dsprites with different reconstruction term weights: 10,000, 7,500, 6,250, 5,000, and 3,750 (see Section~\ref{subsubsec:vae_models}).
For the remainder of this section, these models are referred to as \say{10,000-\ac{VAE}}, \say{7,500-\ac{VAE}}, \say{6,250-\ac{VAE}}, \say{5,000-\ac{VAE}}, and \say{3,750-\ac{VAE}}\footnote{It was first empirically validated that a reconstruction term weight of 10,000 leads to good reconstruction. The other values were chosen by reducing the reconstruction loss term by 1,250 for each model with a larger initial difference (10,000 vs. 7,500).}.

\begin{figure}
    \centering
    \includegraphics[width=\textwidth]{images/latent_space_entanglement/vae_dsprites_lf_10000_dist.png}
    \caption[VAE Latent Space Distribution - dSprites]{Posterior distribution of \ac{VAE} with reconstruction term weight 10,000 on 73,728 dSprites images from the validation set in 100 bins}
    \label{fig:10000_vae_latent_space_distribution}
\end{figure}

Figure~\ref{fig:10000_vae_latent_space_distribution} shows the posterior distributions of dSprites validation images.
The ninth dimension is far less Gaussian than the other nine dimensions.
As \say{object shape} is the only categorical factor of variation, one could assume that the ninth dimension encodes \say{object shape}.
However, the graph shows seven peaks in the ninth dimension, while there are only three distinct shapes in the dataset.

\begin{figure}
    \centering
    \begin{subfigure}{\textwidth}
        \centering
        \includegraphics[width=\textwidth]{images/latent_space_entanglement/vae_dsprites_lf_10000_dist_shape_1.png}
        \caption{Latent space distribution of images with shape \textit{Square}}
    \end{subfigure}
    \begin{subfigure}{\textwidth}
        \centering
        \includegraphics[width=\textwidth]{images/latent_space_entanglement/vae_dsprites_lf_10000_dist_shape_2.png}
        \caption{Latent space distribution of images with shape \textit{Ellipse}}
    \end{subfigure}
    \begin{subfigure}{\textwidth}
        \centering
        \includegraphics[width=\textwidth]{images/latent_space_entanglement/vae_dsprites_lf_10000_dist_shape_3.png}
        \caption{Latent space distribution of images with shape \textit{Heart}}
    \end{subfigure}
    \caption[VAE Latent Space Distribution - dSprites Shapes]{Histogram of dSprites images with a certain shape from the validation set in 100 bins}
    \label{fig:10000_vae_latent_space_distribution_shapes}
\end{figure}

Figure~\ref{fig:10000_vae_latent_space_distribution_shapes} shows the distribution of images with a particular shape.
Even though there is some difference between the plots, the $\mu$-values in the ninth dimension still are assigned to different and distinct areas.
The \textit{shape} alone does not explain the sharp peaks in the ninth dimension.

\begin{figure}
    \centering
    \begin{subfigure}{\textwidth}
        \centering
        \includegraphics[width=\textwidth]{images/latent_space_entanglement/vae_dsprites_lf_10000_dist_scale_0_5.png}
        \caption{Latent space distribution of images with scale = 0.5}
    \end{subfigure}
    \begin{subfigure}{\textwidth}
        \centering
        \includegraphics[width=\textwidth]{images/latent_space_entanglement/vae_dsprites_lf_10000_dist_scale_0_6.png}
        \caption{Latent space distribution of images with scale = 0.6}
    \end{subfigure}
    \begin{subfigure}{\textwidth}
        \centering
        \includegraphics[width=\textwidth]{images/latent_space_entanglement/vae_dsprites_lf_10000_dist_scale_0_7.png}
        \caption{Latent space distribution of images with scale = 0.7}
    \end{subfigure}
    \begin{subfigure}{\textwidth}
        \centering
        \includegraphics[width=\textwidth]{images/latent_space_entanglement/vae_dsprites_lf_10000_dist_scale_0_8.png}
        \caption{Latent space distribution of images with scale = 0.8}
    \end{subfigure}
    \begin{subfigure}{\textwidth}
        \centering
        \includegraphics[width=\textwidth]{images/latent_space_entanglement/vae_dsprites_lf_10000_dist_scale_0_9.png}
        \caption{Latent space distribution of images with shape scale = 0.9}
    \end{subfigure}
    \begin{subfigure}{\textwidth}
        \centering
        \includegraphics[width=\textwidth]{images/latent_space_entanglement/vae_dsprites_lf_10000_dist_scale_1_0.png}
        \caption{Latent space distribution of images with shape scale = 1.0}
    \end{subfigure}
    \caption[VAE Latent Space Distribution - dSprites Scales]{Posterior distribution for dSprites images with a certain scale from the validation set in 100 bins}
    \label{fig:10000_vae_latent_space_distribution_scales}
\end{figure}

Consider Figure~\ref{fig:10000_vae_latent_space_distribution_scales}, showing distributions of images grouped by their scale.
Images in dSprites are scaled by six distinct factors evenly chosen from $[0.5; 1.0]$.
The scale values in the dSprites dataset are six distinct values in the range from 0.5 to 1.0.
A \ac{VAE} model with a too strong reconstruction term does not learn to interpolate between these values but assigns them distinct areas in the latent space.

Grouping images by their scale \textit{and} a shape, explains the distinct peaks in the ninth dimension (see Figure~\ref{fig:10000_vae_latent_space_distribution_scales_and_shapes}).
Each shape has a distinct peak in the ninth dimension.
However, the peaks of shapes \textit{Ellipse} and \textit{Square} superpose.
When plotting all three shapes together, a small peak (\textit{Heart}) and a big peak (\textit{Ellipse} and \textit{Square}) can be observed.

\begin{figure}
    \centering
    \begin{subfigure}{\textwidth}
        \centering
        \includegraphics[width=\textwidth]{images/latent_space_entanglement/vae_dsprites_lf_10000_dist_shape_1_scale_0_5.png}
        \caption{Latent space distribution of images with scale = 0.5 and shape \textit{Square}}
    \end{subfigure}
    \begin{subfigure}{\textwidth}
        \centering
        \includegraphics[width=\textwidth]{images/latent_space_entanglement/vae_dsprites_lf_10000_dist_shape_2_scale_0_5.png}
        \caption{Latent space distribution of images with scale = 0.5 and shape \textit{Ellipse}}
    \end{subfigure}
    \begin{subfigure}{\textwidth}
        \centering
        \includegraphics[width=\textwidth]{images/latent_space_entanglement/vae_dsprites_lf_10000_dist_shape_3_scale_0_5.png}
        \caption{Latent space distribution of images with scale = 0.5 and shape \textit{Heart}}
    \end{subfigure}
    \caption[VAE Latent Space Distribution - dSprites Scale and Shapes]{Posterior distribution for dSprites images with scale = 0.5 and different shapes from the validation set in 100 bins}
    \label{fig:10000_vae_latent_space_distribution_scales_and_shapes}
\end{figure}

Furthermore, the model seems to learn a hierarchical clustering in the ninth dimension.
\textit{Shape} appears to be a sub cluster of \textit{Scale}.
This sub-clustering leads to highly entangled latent space and a highly un-Gaussian distribution.
However, it allows us to violate the \ac{KL}-term in favor of the reconstruction term in just one instead of two dimensions.

\begin{figure}
    \centering
    \begin{subfigure}{\textwidth}
        \centering
        \includegraphics[width=\textwidth]{images/latent_space_entanglement/vae_10000_traverse_square_ellipse.png}
        \caption{Latent space traversal from \textit{Square} $\rightarrow$ \textit{Ellipse}}
        \label{subfig:10000_vae_latent_space_traversal_square_to_ellipse}
    \end{subfigure}
    \begin{subfigure}{\textwidth}
        \centering
        \includegraphics[width=\textwidth]{images/latent_space_entanglement/vae_10000_traverse_ellipse_heart.png}
        \caption{Latent space traversal from \textit{Ellipse} $\rightarrow$ \textit{Heart}}
    \end{subfigure}
    \begin{subfigure}{\textwidth}
        \centering
        \includegraphics[width=\textwidth]{images/latent_space_entanglement/vae_10000_traverse_heart_square.png}
        \caption{Latent space traversal from \textit{Heart} $\rightarrow$ \textit{Square}}
        \label{subfig:10000_vae_latent_space_traversal_heart_to_square}
    \end{subfigure}
    \caption[VAE Latent Space Traversal - dSprites]{Latent space traversal between latent space representations of images with certain shapes for 10,000-\ac{VAE}. Color-values were inverted for this plot.}
    \label{fig:10000_vae_latent_space_traversal_shape_to_shape}
\end{figure}

Figure~\ref{fig:10000_vae_latent_space_traversal_shape_to_shape} shows generated images for a latent space traversal between different shapes\footnote{The corresponding plots for the other model can be found in Appendix~\ref{sec:additional_plots_latent_space_entanglement}.}.
For the plot, the latent representations $\bm{z}_1, \bm{z}_2$ were obtained for two $\bm{x}_i$s with identical parameters except for the shape.
Figure~\ref{fig:10000_vae_latent_space_traversal_shape_to_shape} shows the reconstructions of 21 evenly spaced points on the line segment between $\bm{z}_1$ and $\bm{z}_2$.
It shows the high degree of latent space entanglement: sudden position changes (Figure~\ref{subfig:10000_vae_latent_space_traversal_square_to_ellipse}) or the sudden emergence of new objects (Figure~\ref{subfig:10000_vae_latent_space_traversal_heart_to_square}).

Obtaining $\bm{z}$ by fixing only the shape and the scale for an $\bm{x}_i$ and averaging over all other configurations was unsuccessful.
Based on the cluster analysis, not fixing the scale would easily lead to regions with low probability density.
It lead to empty images as it probably lead to $\bm{z}_i$s in regions of the latent space with low probability density.
Furthermore, for the latent space traversal shown in Figure~\ref{fig:10000_vae_latent_space_traversal_shape_to_shape}, it was possible always to find parameter combinations where the traversal lead to empty images in the middle of the line segment.

\begin{figure}
    \centering
    \begin{subfigure}{\textwidth}
        \centering
        \includegraphics[width=\textwidth]{images/latent_space_entanglement/vae_dsprites_lf_7500_dist.png}
        \caption{Reconstruction term weight 7,500}
    \end{subfigure}
    \begin{subfigure}{\textwidth}
        \centering
        \includegraphics[width=\textwidth]{images/latent_space_entanglement/vae_dsprites_lf_6250_dist.png}
        \caption{Reconstruction term weight 6,250}
    \end{subfigure}
    \begin{subfigure}{\textwidth}
        \centering
        \includegraphics[width=\textwidth]{images/latent_space_entanglement/vae_dsprites_lf_5000_dist.png}
        \caption{Reconstruction term weight 5,000}
    \end{subfigure}
    \begin{subfigure}{\textwidth}
        \centering
        \includegraphics[width=\textwidth]{images/latent_space_entanglement/vae_dsprites_lf_3750_dist.png}
        \caption{Reconstruction term weight 3,750}
    \end{subfigure}
    \caption[VAE Latent Space Distribution - Different Reconstruction Term Weights]{Posterior distribution of VAE with different reconstruction terms weights for 73,728 dSprites images from the validation set in 100 bins}
    \label{fig:7500_5000_vae_latent_space_distribution_scales_and_shapes}
\end{figure}

Figure~\ref{fig:7500_5000_vae_latent_space_distribution_scales_and_shapes} shows the latent space distribution of \acp{VAE} with a reduced reconstruction term weight, i.e., an increase in \ac{KL-divergence} term weight.
The histograms look more Gaussian, and the \ac{KL-divergence} on the validation set decreased from 22.744 (reconstruction loss weight: 10,000) to 22.271 (reconstruction loss weight: 7,500) or, respectively, 18.172 (reconstruction loss weight: 5,000).
Surprisingly, the difference between 10,000-\ac{VAE} and 7,500-\ac{VAE} is small on the validation data.

To account for latent space disentanglement, the \ac{PPL}\footnote{See Section~\ref{subsec:feature-disentanglement}.} for the latent spaces of the different models is computed.
The perceptual loss is obtained by a \ac{CNN}, trained to predict dSprites object categories\footnote{The model architecture can be found in Appendix~\ref{sec:appendix_feature_extraction_network_ppl_dsprites}, the features are extracted in \say{conv2d\_94}. The model was trained for one epoch using categorical crossentropy with the Adam optimizer with an initial learning rate of 0.1. The model achieved a top-1 accuracy of 0.8 on the validation set.}.
The \ac{PPL} was computed for 50 different random seeds.
Each of these \ac{PPL}-computations interpolated at 100 random steps (dependent on the random seed).

\begin{table}
    \centering
    \begin{tabular}{lrr}
        \toprule
        Model           & mean \ac{PPL} & standard deviation \\
        \midrule
        10,000-\ac{VAE} & 936.257       & 779.098            \\
        7,500-\ac{VAE}  & 2834.674      & 567.933            \\
        6,250-\ac{VAE}  & 4498.156      & 1091.255           \\
        5,000-\ac{VAE}  & 173.533       & 35.896             \\
        3,750-\ac{VAE}  & 258.326       & 49.117             \\
        \bottomrule
    \end{tabular}
    \caption{\acfp{PPL} for the different models latent spaces}
    \label{tab:ppl-dsprites}
\end{table}

Table~\ref{tab:ppl-dsprites} shows mean \ac{PPL} values and the standard deviations.
A two-sided Wilcoxon rank-sum test for the differences in the model's \acp{PPL} yielded $p$-values $< 10^{-29}$ in all cases.
Surprisingly, the 10,000-\ac{VAE}'s \ac{PPL} is lower than for 7,500-\ac{VAE} or 6,250-\ac{VAE}.

This can be explained by the vast regions of low probability density in the posterior distribution.
For 10,000-\ac{VAE}, it is likely to produce images that are almost entirely black for some random point in the latent space.
Only a few regions in the latent space produce reasonable images.
A neighboring point in the latent space will probably also produce a black image, leading to a low perceptual loss.
As the reconstruction term weight is lowered (7,500-\ac{VAE} and 6,250-\ac{VAE}), the probability of generating some image and, therefore, the \ac{PPL} increases.
For 5,000-\ac{VAE} and 3,750-\ac{VAE}, finally, the latent space is sufficiently disentangled.
Again, this leads to a low \ac{PPL}.

The standard deviation of the \acp{PPL} supports this argumentation.
Consider 6,250-\ac{VAE}.
Here, generating some image for a random point in the latent space is already quite likely.
Nevertheless, there are still \say{enough} regions producing black images.
For these black regions, the \ac{PPL} is low, just like for 10,000-\ac{VAE}.
However, for the points producing images, the \ac{PPL} is high, because the latent space still is entangled.
This causes a high standard deviation.

The previous observations have some implications for the application of the \ac{PPL}.
First, the latent space to be approximately Gaussian is necessary for the \ac{PPL} to produce interpretable results.
Second, if the latent space is Gaussian, the standard deviation should be observed.
A too high standard deviation is evidence that there are regions of low probability density in the latent space.
The \ac{PPL} standard deviation could be a measure of this.

\subsection{Latent Space Analysis}\label{subsec:model-generated-samples}

\subsubsection{Latent Space Embeddings}\label{subsubsec:latent_space_embeddings}

\paragraph{\textsc{Mnist}}

Figures~\ref{fig:vae_latent_space_mnist} and~\ref{fig:vlae_latent_space_mnist} show the latent spaces of \ac{VAE} and \ac{VLAE} trained on the \textsc{Mnist} dataset (\textsc{Mnist}-\ac{VAE} and \textsc{Mnist}-\ac{VLAE}, see Section~\ref{subsec:models}).
For \ac{VAE}, there is only one latent space while there are three for \ac{VLAE}.
The embeddings are colored by different means, employing information from Morpho-\textsc{Mnist} (see Section~\ref{subsubsec:morphomnist}) and the digit identity information provided by \textsc{Mnist} itself.

As discussed in Section~\ref{subsubsec:representation_learning}, the \acl{VLAE} aims at learning \say{hierarchical disentangled representation}~\citep{zhao2017learning}.
The lower embedding layers of such a model trained on \textsc{Mnist} (see Section~\ref{subsubsec:mnist}) encode features such as stroke width, digit width, and digit tilt, whereas the highest layer mainly learns digit identity~\citep{zhao2017learning}.

Incorporating the additional labels provided by Morpho-\textsc{Mnist} (see Section~\ref{subsubsec:morphomnist}) helps to analyze to what extent the lower layers learn which morphological features of \textsc{MNIST}.
The morphological attributes, however, are not equally distributed for all digits.
The mean stroke length and the digit width, for example, have a meager mean value for the digit \say{1} (see Figure~\ref{fig:morpho_mnist_distribution}).
Therefore, it should be possible to almost uniquely identify digit \say{1} identity by just considering the stroke length or the digit width.
Other attributes, such as stroke thickness, digit slant, and digit height, are more evenly distributed (see Figure~\ref{fig:morpho_mnist_distribution}).

Consider Figure~\ref{subfig:vlae_mnist_latent_space_z_1_slant} showing the embedding layer $\bm{z}_1$ colored by digit slant.
Figure~\ref{fig:morpho_mnist_distribution} shows that the mean of the attribute \textit{Slant} is quite evenly distributed.
The color gradient in Figure~\ref{subfig:vlae_mnist_latent_space_z_1_slant}, therefore, indicates that the VLAE learns the morphological attribute instead of just showing the class identity, encoded using another morphological attribute that correlates with class identity.

For Figure~\ref{subfig:vlae_mnist_latent_space_z_1_width}, the situation is different.
The noticeable dark-purple cluster in the top left correlates with dark-purple points in Figure~\ref{subfig:vlae_mnist_latent_space_z_1_identity} that encode image with label \say{1}.
For digit width, however, \say{1} is an outlier (see Figure~\ref{fig:morpho_mnist_distribution}), and a small digit width, therefore, is a reliable indicator for a digit identity of \say{1}.
Nonetheless, overall digit identity does not seem to be encoded strongly by $\bm{z}_1$ (see Figure~\ref{subfig:vlae_mnist_latent_space_z_1_identity}).

\begin{figure}
    \centering
    \begin{subfigure}{.32\textwidth}
        \includegraphics[width=\textwidth]{images/latent_spaces/mnist/vae/embeddings_mu_0.png}
        \caption{Latent space colored by digit slant}
        \label{subfig:vae_mnist_latent_space_slant}
    \end{subfigure}
    \hfill
    \begin{subfigure}{.32\textwidth}
        \includegraphics[width=\textwidth]{images/latent_spaces/mnist/vae/embeddings_mu_1.png}
        \caption{Latent space colored by digit thickness}
        \label{subfig:vae_mnist_latent_space_thickness}
    \end{subfigure}
    \hfill
    \begin{subfigure}{.32\textwidth}
        \includegraphics[width=\textwidth]{images/latent_spaces/mnist/vae/embeddings_mu_2.png}
        \caption{Latent space colored by digit area}
        \label{subfig:vae_mnist_latent_space_area}
    \end{subfigure}
    \hfill
    \begin{subfigure}{.24\textwidth}
        \includegraphics[width=\textwidth]{images/latent_spaces/mnist/vae/embeddings_mu_3.png}
        \caption{Latent space colored by digit length}
        \label{subfig:vae_mnist_latent_space_length}
    \end{subfigure}
    \hfill
    \begin{subfigure}{.24\textwidth}
        \includegraphics[width=\textwidth]{images/latent_spaces/mnist/vae/embeddings_mu_4.png}
        \caption{Latent space colored by digit width}
        \label{subfig:vae_mnist_latent_space_width}
    \end{subfigure}
    \hfill
    \begin{subfigure}{.24\textwidth}
        \includegraphics[width=\textwidth]{images/latent_spaces/mnist/vae/embeddings_mu_5.png}
        \caption{Latent space colored by digit height}
        \label{subfig:vae_mnist_latent_space_height}
    \end{subfigure}
    \hfill
    \begin{subfigure}{.24\textwidth}
        \includegraphics[width=\textwidth]{images/latent_spaces/mnist/vae/embeddings_mu_6.png}
        \caption{Latent space colored by digit identity}
        \label{subfig:vae_mnist_latent_space_identity}
    \end{subfigure}
    \caption[\ac{VAE} Latent Space on \textsc{Mnist}]{Latent space colored by different means of \ac{VAE} with $dim_z=2$ trained on \textsc{Mnist} dataset}
    \label{fig:vae_latent_space_mnist}
\end{figure}

\begin{landscape}
    \begin{figure}
        \centering
        \foreach \i in {1,2,3}{
        \begin{subfigure}{.19\textwidth}
            \includegraphics[width=\textwidth]{images/latent_spaces/mnist/vlae/embeddings_mu_\i_0.png}
            \caption{$z_{\i}$: digit slant}
            \label{subfig:vlae_mnist_latent_space_z_\i_slant}
        \end{subfigure}
        \hfill
        \begin{subfigure}{.19\textwidth}
            \includegraphics[width=\textwidth]{images/latent_spaces/mnist/vlae/embeddings_mu_\i_1.png}
            \caption{$z_{\i}$: digit thickness}
            \label{subfig:vlae_mnist_latent_space_z_\i_thickness}
        \end{subfigure}
        \hfill
        \begin{subfigure}{.19\textwidth}
            \includegraphics[width=\textwidth]{images/latent_spaces/mnist/vlae/embeddings_mu_\i_2.png}
            \caption{$z_{\i}$: digit area}
            \label{subfig:vlae_mnist_latent_space_z_\i_area}
        \end{subfigure}
        \hfill
        \begin{subfigure}{.19\textwidth}
            \includegraphics[width=\textwidth]{images/latent_spaces/mnist/vlae/embeddings_mu_\i_3.png}
            \caption{$z_{\i}$: digit length}
            \label{subfig:vlae_mnist_latent_space_z_\i_length}
        \end{subfigure}
        \hfill
        \begin{subfigure}{.19\textwidth}
            \includegraphics[width=\textwidth]{images/latent_spaces/mnist/vlae/embeddings_mu_\i_4.png}
            \caption{$z_{\i}$: digit width}
            \label{subfig:vlae_mnist_latent_space_z_\i_width}
        \end{subfigure}
        \hfill
        \begin{subfigure}{.19\textwidth}
            \includegraphics[width=\textwidth]{images/latent_spaces/mnist/vlae/embeddings_mu_\i_5.png}
            \caption{$z_{\i}$: digit height}
            \label{subfig:vlae_mnist_latent_space_z_\i_height}
        \end{subfigure}
        \hfill
        \begin{subfigure}{.19\textwidth}
            \includegraphics[width=\textwidth]{images/latent_spaces/mnist/vlae/embeddings_mu_\i_6.png}
            \caption{$z_{\i}$: digit identity}
            \label{subfig:vlae_mnist_latent_space_z_\i_identity}
        \end{subfigure}}
        \caption[\ac{VLAE} Latent Space on \textsc{Mnist}]{Latent space colored by different means of \ac{VLAE} with $dim_z=2$ trained on \textsc{Mnist} dataset}
        \label{fig:vlae_latent_space_mnist}
    \end{figure}
\end{landscape}

Noticeable, digit identity seems to be learned in $z_2$ and $z_3$ (see Figures~\ref{subfig:vlae_mnist_latent_space_z_2_identity} and \ref{subfig:vlae_mnist_latent_space_z_3_identity}).
However, learning one factor of variation on two layers would be a contradiction to Figure~\ref{subfig:vlae_mnist_latent_space_traversal} and even more Figure~\ref{subfig:vlae_gan_mnist_latent_space_traversal}, showing that digit identity mainly is learned in the last embedding layer.
Albeit $z_2$ has some influence on digit identity, this effect is not nearly as prominent as for $z_3$ even though the clustering in Figure~\ref{subfig:vlae_mnist_latent_space_z_2_identity} seems to be almost as good as in Figure~\ref{subfig:vlae_mnist_latent_space_z_3_identity}.
It seems that the digit identity learned in $z_2$ only supports the model generation but has a far less strong influence than $z_3$.

All in all, \ac{VLAE} does not seem to learn entirely separable representations.
However, the less powerful network below $z_1$ indeed seems to learn lower-level representations that also are employed when generating new images (see Figures~\ref{subfig:vlae_mnist_latent_space_traversal} and \ref{subfig:vlae_gan_mnist_latent_space_traversal}).

For \ac{VAE}, digit identity defines the shape of the latent space (Figure~\ref{subfig:vae_mnist_latent_space_identity}), leading to more prominent \say{main clusters.}
As \ac{VAE} only has one latent space, it employs subspaces to learn the lower-level factors of variation, such as slant within the main clusters (Figure~\ref{subfig:vae_mnist_latent_space_slant}).

The \ac{VAE} seems to learn an efficient encoding of most of the factors of variation as revealed by this kind of visualization even though it has a narrower information bottleneck with a two-dimensional latent space compared to the three two-dimensional latent spaces of \ac{VLAE}.

\paragraph{CelebA}

\begin{wrapfigure}[26]{R}{0.4\textwidth}
    \centering
    \includegraphics[height=.39\textheight]{images/latent_spaces/celeba/attribute_distribution.png}
    \caption[CelebA: Feature Distribution]{Distribution of the Binary Factors of Variation in the CelebA dataset. The green bar shows the fraction of the images where the attribute is present, the red bar where it is not present.}
    \label{fig:celeba_features_distribution}
\end{wrapfigure}

\begin{figure}
    \centering
    \begin{subfigure}{.49\textwidth}
        \includegraphics[width=\textwidth]{images/latent_spaces/celeba/vlae/vlae_celeba_Black_Hair.png}
        \caption{Black hair present (green) or not (red)}
        \label{subfig:latent_space_celeba_vale_colored_black_hair}
    \end{subfigure}
    \hfill
    \begin{subfigure}{.49\textwidth}
        \includegraphics[width=\textwidth]{images/latent_spaces/celeba/vlae/vlae_celeba_Blond_Hair.png}
        \caption{Blond hair present (green) or not (red)}
        \label{subfig:latent_space_celeba_vale_colored_blond_hair}
    \end{subfigure}
    \begin{subfigure}{.49\textwidth}
        \includegraphics[width=\textwidth]{images/latent_spaces/celeba/vlae/vlae_celeba_Wearing_Hat.png}
        \caption{Wearing a hat (green) or not (red)}
        \label{subfig:latent_space_celeba_vale_colored_waring_hat}
    \end{subfigure}
    \hfill
    \begin{subfigure}{.49\textwidth}
        \includegraphics[width=\textwidth]{images/latent_spaces/celeba/vlae/vlae_celeba_Wearing_Earrings.png}
        \caption{Wearing earrings (green) or not (red)}
        \label{subfig:latent_space_celeba_vale_colored_wearing_earrings}
    \end{subfigure}
    \caption[\ac{VLAE} Latent Space for CelebA, Curated Features]{\ac{t-SNE}-reduced Latent Space of a \ac{VLAE} trained on CelebA, colored by curated Factors of Variation (present or not present)}
    \label{fig:latent_space_celeba_vale_colored}
\end{figure}

The CelebA dataset provides binary labels for factors of variation.
Figure~\ref{fig:celeba_features_distribution} shows the distribution of the different features in the dataset.
Figure~\ref{fig:latent_space_celeba_vale_colored} shows the \ac{t-SNE}-reduced latent space of a \ac{VLAE} (CelebA-\ac{VLAE}, see Section~\ref{subsubsec:vlae_models}).
A green dot in the latent space indicates that the feature is present for the data point, a red point indicates that it is not present\footnote{The plots for all models and all factors of variation can be found in Appendix~\ref{subsection:appendix_celeba_latent_space}/}.

The latent space plots allow to analyze which features are learned in which layers.
For example, hair color is mainly learned in the first and second layers (see Figures~\ref{subfig:latent_space_celeba_vale_colored_black_hair} and \ref{subfig:latent_space_celeba_vale_colored_blond_hair}) because green and red dots build clusters within these layers.
Other features do not seem to be learned at all (see Figure~\ref{subfig:latent_space_celeba_vale_colored_wearing_earrings}).

However, since CelebA provides only binary attributes, the latent space analysis is less insightful than for dSprites and Morpho-\textsc{Mnist}.
Furthermore, the models learn factors of variation that are not present in the labels (see Section~\ref{subsubsec:latent_space_traversals}).

\paragraph{dSprites}

The latent space embeddings for the dSprites dataset allow a more profound comparison of \ac{VAE} and \ac{VLAE}.
Here, the dSprites-\ac{VAE}-dim6 model (see Section~\ref{subsubsec:vae_models}) was given a six-dimensional embedding space, whereas the dSprites-\ac{VLAE}-dim2 (see Section~\ref{subsubsec:vlae_models}) model has three two-dimensional embedding spaces.
The \ac{VAE} and \ac{VLAE} approximately have the same model capacity under the assumption that \ac{VLAE} uses lower embedding layers to learn lower-level features and higher embedding layers to only learn higher-level features and that the dataset can be split in this way.
The \ac{VLAE} latent spaces were chosen two-dimensional to allow for a direct visualization of the latent space without dimensionality reduction.
As the \ac{VAE} model employs a higher-dimensional latent space, the embeddings are visualized using \ac{t-SNE} embeddings (see Figure~\ref{fig:vae_latent_space_dsprites}).

One advantage of dSprites over \textsc{Mnist} is that all factors of variation by design are independent and represented equally often, allowing a more straightforward analysis.

\begin{figure}
    \centering
    \begin{adjustbox}{valign=T}
        \begin{subfigure}{.19\textwidth}
            \includegraphics[width=\textwidth]{images/latent_spaces/dsprites/vae/embeddings_mu_0.png}
            \caption{Latent space colored by object shape}
        \end{subfigure}
    \end{adjustbox}
    \hfill
    \begin{adjustbox}{valign=T}
        \begin{subfigure}{.19\textwidth}
            \includegraphics[width=\textwidth]{images/latent_spaces/dsprites/vae/embeddings_mu_1.png}
            \caption{Latent space colored by object scale}
            \label{subfig:vae_embedding_dsprites_scale}
        \end{subfigure}
    \end{adjustbox}
    \hfill
    \begin{adjustbox}{valign=T}
        \begin{subfigure}{.19\textwidth}
            \includegraphics[width=\textwidth]{images/latent_spaces/dsprites/vae/embeddings_mu_2.png}
            \caption{Latent space colored by object orientation}
            \label{subfig:vae_embedding_dsprites_orientation}
        \end{subfigure}
    \end{adjustbox}
    \hfill
    \begin{adjustbox}{valign=T}
        \begin{subfigure}{.19\textwidth}
            \includegraphics[width=\textwidth]{images/latent_spaces/dsprites/vae/embeddings_mu_3.png}
            \caption{Latent space colored by object $x$-position}
        \end{subfigure}
    \end{adjustbox}
    \hfill
    \begin{adjustbox}{valign=T}
        \begin{subfigure}{.19\textwidth}
            \includegraphics[width=\textwidth]{images/latent_spaces/dsprites/vae/embeddings_mu_4.png}
            \caption{Latent space colored by object $y$-position}
        \end{subfigure}
    \end{adjustbox}
    \caption[\ac{VAE} Latent Space on dsprites]{\ac{t-SNE} latent space embeddings colored by different means of \ac{VAE} with $dim_z=6$ trained on dsprites dataset}
    \label{fig:vae_latent_space_dsprites}
\end{figure}

\begin{figure}
    \centering
    \foreach \i in {1,2,3}{
    \begin{subfigure}{.19\textwidth}
        \includegraphics[width=\textwidth]{images/latent_spaces/dsprites/vlae/embeddings_mu_\i_0.png}
        \caption{Latent space $z_{\i}$ colored by object shape}
        \label{subfig:vlae_embedding_z\i_dsprites_shape}
    \end{subfigure}
    \hfill
    \begin{subfigure}{.19\textwidth}
        \includegraphics[width=\textwidth]{images/latent_spaces/dsprites/vlae/embeddings_mu_\i_1.png}
        \caption{Latent space $z_{\i}$ colored by object scale}
        \label{subfig:vlae_embedding_z\i_dsprites_scale}
    \end{subfigure}
    \hfill
    \begin{subfigure}{.19\textwidth}
        \includegraphics[width=\textwidth]{images/latent_spaces/dsprites/vlae/embeddings_mu_\i_2.png}
        \caption{Latent space $z_{\i}$ colored by object orientation}
        \label{subfig:vlae_embedding_z\i_dsprites_orientation}
    \end{subfigure}
    \hfill
    \begin{subfigure}{.19\textwidth}
        \includegraphics[width=\textwidth]{images/latent_spaces/dsprites/vlae/embeddings_mu_\i_3.png}
        \caption{Latent space $z_{\i}$ colored by object $x$-position}
        \label{subfig:vlae_embedding_z\i_dsprites_x_pos}
    \end{subfigure}
    \hfill
    \begin{subfigure}{.19\textwidth}
        \includegraphics[width=\textwidth]{images/latent_spaces/dsprites/vlae/embeddings_mu_\i_4.png}
        \caption{Latent space $z_{\i}$ colored by object $y$-position}
        \label{subfig:vlae_embedding_z\i_dsprites_y_pos}
    \end{subfigure}
    }
    \caption[\ac{VLAE} Latent Space on dsprites]{Latent space colored by different means of \ac{VLAE} with $dim_{z_i}=6$ trained on dsprites dataset}
    \label{fig:vlae_latent_space_dsprites}
\end{figure}

Again, the \ac{VAE} builds main clusters and seems to embed the lower-level features into main clusters.
Especially the object scale seems to be a lower-level feature (see Figure~\ref{subfig:vae_embedding_dsprites_scale}).
Only object-orientation does not seem to be learned at all (see Figure~\ref{subfig:vae_embedding_dsprites_orientation}).

\ac{VLAE} performance on dsprites (see Figure~\ref{fig:vlae_latent_space_dsprites}) is worse compared to \ac{VAE}.
Again, the model learns main clusters, especially for object position (see Figures~\ref{subfig:vlae_embedding_z3_dsprites_x_pos} and~\ref{subfig:vlae_embedding_z3_dsprites_y_pos}).
Furthermore, the model learns the object scale (see Figure~\ref{subfig:vlae_embedding_z3_dsprites_scale} and \ref{subfig:vlae_embedding_z2_dsprites_scale}).
Otherwise, the model does not seem to learn other factors of variation.
Furthermore, the representations are not particularly disentangled.
Object position and scale are learned in all layers, but best in $z_3$.

After training 200 epochs, the \ac{VAE} achieved a loss of 21.638 (reconstruction loss: 6.375, \ac{KL}-loss: 15.264).
In contrast, the \ac{VLAE} achieved a loss of 47.860 (reconstruction loss: 32.813, \ac{KL}-loss: 15.047).

All in all, using a \ac{VAE} with a higher-dimensional space seems superior to the \ac{VLAE} in terms of encoding efficiency.

\subsubsection{Latent Space Traversals}\label{subsubsec:latent_space_traversals}

\subparagraph{\textsc{Mnist}}

A latent space traversal is one way to see if and how a model learns a general representation of the dataset.
Figures~\ref{fig:mnist_latent_space_traversal_vae} and \ref{fig:mnist_latent_space_traversal_vlae} show the latent space traversal of \textsc{Mnist}-\ac{VAE}\footnote{See Section~\ref{subsubsec:vae_models}.}, \textsc{Mnist}-\ac{VAE}-\ac{GAN}\footnote{See Section~\ref{subsubsec:vae_gan_models}.}, \textsc{Mnist}-\ac{VLAE}\footnote{See Section~\ref{subsubsec:vlae_models}.}, and \textsc{Mnist}-\ac{VLAE}-\ac{GAN}\footnote{See Section~\ref{subsubsec:vlae_gan_models}.}.
For the models with only one latent layer (\textsc{Mnist}-\ac{VAE} and \textsc{Mnist}-\ac{VAE}-\ac{GAN}), the plot is generated by traversing the latent space in equal steps from $z_i = -3$ to $z_i = 3$ in both dimensions.
For the hierarchical models (\textsc{Mnist}-\ac{VLAE} and \textsc{Mnist}-\ac{VLAE}-\ac{GAN}), one $z$-layer was fixed and traversed in the same way.
The values of the other $z$-layers were obtained by sampling from a uniform distribution over $[-3; 3]$.

First, most models properly employ the latent space - the traversal shows almost no non-representative generated images, with a few exceptions for \textsc{Mnist}-\ac{VLAE} and \textsc{Mnist}-\ac{VLAE}-\ac{GAN}.
Noteworthy, these non-representative images mainly occur on the borders where the variance is high.
The models were trained such that the latent space is standard normal, the latent space traversal, however, goes from -3 to 3, in areas with low probability density.
Unrepresentative generations in such areas, therefore, is no model failure.

\begin{figure}
    \centering
    \begin{subfigure}{.45\textwidth}
        \centering
        % include first image
        \includegraphics[width=\textwidth]{images/latent_space_traversals/vae_mnist.png}
        \caption{\ac{VAE} latent space traversal from $z_i=-3$ to $z_i=3$ in both $z$ dimensions}
    \end{subfigure}
    \hfill
    \begin{subfigure}{.45\textwidth}
        \centering
        % include second image
        \includegraphics[width=\textwidth]{images/latent_space_traversals/vae_gan_mnist.png}
        \caption{\ac{VAE}-\ac{GAN} latent space traversal from $z_i=-3$ to $z_i=3$ in both $z$ dimensions}
    \end{subfigure}
    \caption[\ac{VAE} Models on \textsc{Mnist} - Latent Space Traversal]{Latent space traversal for \ac{VAE} models on \textsc{Mnist}. The original images have been inverted for the purpose of this figure.}
    \label{fig:mnist_latent_space_traversal_vae}
\end{figure}
\begin{figure}
    \centering
    \begin{subfigure}{\textwidth}
        \centering
        % include second image
        \includegraphics[width=\textwidth]{images/latent_space_traversals/vlae_mnist.png}
        \caption{\ac{VLAE} latent space traversal from $z_i=-3$ to $z_i=-3$ in both dimensions for the respective $z$-dimension. The other $z$ dimensions are sampled uniformly over $[-3; 3]$}
        \label{subfig:vlae_mnist_latent_space_traversal}
    \end{subfigure}
    \begin{subfigure}{\textwidth}
        \centering
        % include second image
        \includegraphics[width=\textwidth]{images/latent_space_traversals/vlae_gan_mnist.png}
        \caption{\ac{VLAE}-\ac{GAN} latent space traversal from $z_i=-3$ to $z_i=3$ in both dimensions for the respective $z$-dimension. The other $z$ dimensions are sampled uniformly over $[-3; 3]$}
        \label{subfig:vlae_gan_mnist_latent_space_traversal}
    \end{subfigure}
    \caption[\ac{VLAE} Models on \textsc{Mnist} - Latent Space Traversal]{Latent space traversal for \ac{VLAE} models on \textsc{Mnist}. The original images have been inverted for the purpose of this figure.}
    \label{fig:mnist_latent_space_traversal_vlae}
\end{figure}

\begin{figure}
    \centering
    \begin{subfigure}{.48\textwidth}
        % include second image
        \includegraphics[width=\textwidth]{images/latent_space_traversals/vlae_mnist_morpho_latent_space_values_area.png}
        \caption{area}
    \end{subfigure}
    \hfill
    \begin{subfigure}{.48\textwidth}
        % include second image
        \includegraphics[width=\textwidth]{images/latent_space_traversals/vlae_mnist_morpho_latent_space_values_height.png}
        \caption{height}
    \end{subfigure}
    \begin{subfigure}{.48\textwidth}
        % include second image
        \includegraphics[width=\textwidth]{images/latent_space_traversals/vlae_mnist_morpho_latent_space_values_identity.png}
        \caption{identity}
        \label{subfig:vlae_mnist_morpho_latent_space_values_identity}
    \end{subfigure}
    \hfill
    \begin{subfigure}{.48\textwidth}
        % include second image
        \includegraphics[width=\textwidth]{images/latent_space_traversals/vlae_mnist_morpho_latent_space_values_length.png}
        \caption{length}
    \end{subfigure}
    \begin{subfigure}{.48\textwidth}
        % include second image
        \includegraphics[width=\textwidth]{images/latent_space_traversals/vlae_mnist_morpho_latent_space_values_slant.png}
        \caption{slant}
        \label{subfig:vlae_mnist_morpho_latent_space_values_slant}
    \end{subfigure}
    \hfill
    \begin{subfigure}{.48\textwidth}
        % include second image
        \includegraphics[width=\textwidth]{images/latent_space_traversals/vlae_mnist_morpho_latent_space_values_thickness.png}
        \caption{thickness}
        \label{subfig:vlae_mnist_morpho_latent_space_values_thickness}
    \end{subfigure}
    \begin{subfigure}{.48\textwidth}
        % include second image
        \includegraphics[width=\textwidth]{images/latent_space_traversals/vlae_mnist_morpho_latent_space_values_width.png}
        \caption{width}
    \end{subfigure}
    \caption{Mean latent space values for \ac{VLAE} on \textsc{Mnist} when fixing different factors of variation from Morpho-\textsc{Mnist}}
    \label{fig:vlae_mnist_morpho_latent_space_values}
\end{figure}

Morpho-\textsc{Mnist} allows for another kind of analysis.
Consider Figure~\ref{fig:vlae_mnist_morpho_latent_space_values}\footnote{The corresponding figures for the other models can be found in Appendix~\ref{sec:appendix_plots_latent_space_traversals}}.
Each sub-figure shows the latent space values for \textsc{Mnist} images, predicted by the \textsc{Mnist}-\ac{VLAE} when fixing one Morpho-\textsc{Mnist} attribute and averaging over the others.
The value range of each Morpho-\textsc{Mnist} attribute was divided into 50 evenly-sized bins, and the \textsc{Mnist} images were assigned to these bins based on the value of the corresponding attribute.
Single bins can contain no values if no \textsc{Mnist} image has an attribute value in the corresponding range.
The $\bm{z}$-predictions within one bin are averaged.
Each column in a sub-figure in Figure~\ref{fig:vlae_mnist_morpho_latent_space_values} corresponds to one layer of \ac{VLAE} (the first column corresponds to the first layer, etc.).
The second row in each sub-figure shows the mean latent space position for each bin.
The bins are ordered, increasing color values correspond to increasing attribute values as indicated by the respective color bar.
The first row shows the latent space position for each dimension separately.
The $x$-axis corresponds to the (ordered) bins.

If one layer does not learn a particular attribute, its value should not change much as the attribute values are varied.
For example, this can be observed in Figure~\ref{subfig:vlae_mnist_morpho_latent_space_values_thickness} (layers two and three).
Even though the latent space values are non-stationary for high thickness values, this can be attributed to the few images of a high thickness (see Figure~\ref{fig:morpho_mnist_distribution}).
Apart from that, only layer one seems to encode thickness as it is the only layer showing a trend for this attribute.
This is supported by Figure~\ref{subfig:vlae_mnist_latent_space_traversal}, where thickness increases along the trajectory in Figure~\ref{subfig:vlae_mnist_morpho_latent_space_values_thickness}.
Another example is the slant (Figure~\ref{subfig:vlae_mnist_morpho_latent_space_values_slant}).
The slant-trajectory is almost orthogonal to the thickness-trajectory (see also Figure~\ref{subfig:vlae_mnist_latent_space_traversal}).

However, many attributes do not seem to be learned in one layer.
For example, the values of layers two and three are highly non-stationary for digit identity (Figure~\ref{subfig:vlae_mnist_morpho_latent_space_values_identity}).
This indicates that digit identity is jointly learned in both layers.
Figure~\ref{subfig:vlae_mnist_latent_space_traversal} supports this.
Traversals in both layers seem to influence digit identity, but there is still much variation within subregions.
The same holds for other attributes.

In conclusion, \textsc{Mnist}-\ac{VLAE}, in some cases, succeeds in learning separable representations but also fails in other cases.
A low variation of values within one layer when changing one factor of variation indicates that this layer does not learn this factor of variation.
This is not obvious because the layer could learn this attribute and be overruled by the higher layers during the reconstructions.
However, this does not seem to be the case, and the \ac{VLAE} uses the latent space efficiently, as observed by jointly studying Figure~\ref{fig:vlae_mnist_morpho_latent_space_values} and Figure~\ref{subfig:vlae_mnist_latent_space_traversal}.

\subparagraph{CelebA}

\begin{figure}
    \centering
    \includegraphics[width=\textwidth]{images/latent_space_traversals/vlae_gan_celeba.png}
    \caption[\ac{VLAE}-\ac{GAN} on CelebA: Latent Space Exploration]{Latent space exploration of \ac{VLAE}-\ac{GAN} with $z_{dim_i}=2$ on CelebA.}
    \label{fig:celeba_latent_space_traversal}
\end{figure}

Figure~\ref{fig:celeba_latent_space_traversal} shows the latent space exploration of CelebA-\ac{VLAE}-\ac{GAN} model\footnote{See Section~\ref{subsubsec:vlae_gan_models}.} with input size $128\times 128$ on the CelebA dataset.
The model was generated by evenly interpolating in $[-3; 3]$ in the respective layer and by sampling from a uniform distribution over $[-3; 3]$ for the other layers.
Similar to \textsc{Mnist}, the model learns different factors of variation on different layers.
The first layer mainly learns skin color, layer two hair color, and layer three pose and background color.
Importantly, the model learns only a few factors of variation due to the small latent space dimensionality.

\begin{figure}
    \centering
    \begin{subfigure}{\textwidth}
        % include second image
        \includegraphics[width=\textwidth]{images/latent_space_traversals/vae_celeba_black_to_blond.png}
        \caption{Black to Blond Hair}
        \label{subfig:black_to_blond}
    \end{subfigure}
    \begin{subfigure}{\textwidth}
        % include second image
        \includegraphics[width=\textwidth]{images/latent_space_traversals/vae_celeba_man_to_woman.png}
        \caption{Female to Male}
        \label{subfig:female_to_male}
    \end{subfigure}
    \caption[Interpolating between black and blond hair, man and woman]{Interpolating between latent factors of variation in a \ac{VAE} latent space, trained on CelebA}
    \label{fig:vae_celeba_black_to_blond_man_to_woman}
\end{figure}

As discussed in Section~\ref{subsubsec:representation_learning}, \textit{feature consistency} is an important property of \acp{VAE}.
It was empirically evaluated if the \ac{VAE} models have the same property.
Figure~\ref{fig:vae_celeba_black_to_blond_man_to_woman} shows the transition between black and blond hair, and female and male for a \ac{VAE} trained on ImageNet.

To generate the plots, the mean vectors of the different factors of variation (\textit{male}, \textit{black hair}, etc.) were obtained by predicting the posterior for corresponding images from CelebA.
Then, for Figure~\ref{subfig:black_to_blond}, a random latent vector $\bm{v}_1$ with \say{black hair} was chosen from the dataset.
The transition was then generated by choosing different values $\alpha \in [0, 2]$ in the operation $\bm{v}_1 - \bm{v}(\text{black hair}) + \alpha\bm{v}(\text{blond hair})$.
For Figure~\ref{subfig:female_to_male}, a random image was generated by sampling $\bm{v}_2\sim \mathcal{N}(\bm{0}, \bm{I})$.
The transition was then generated by $\bm{v}_2 + \alpha\bm{v}(\text{blond hair})$ with $\alpha \in [0, 4]$.

Even though larger values of $\alpha$ are required to obtain meaningful translations, the latent space has the discussed semantic property.
The same holds for CelebA-\ac{VAE}-\ac{GAN}.

\subparagraph{dsprites}

Section~\ref{subsec:latent-space-entanglement-and-categorical-factors-of-variation} discusses the latent space of dSprites-\acp{VAE}\footnote{See Section~\ref{subsubsec:vae_gan_models}.}.
One conclusion was that \acp{VAE} on dsprites do not learn a smooth transition of different scales.
A transition between different shapes, however, is possible.

\begin{figure}
    \centering
    \foreach \i in {1,..., 7}{
    \begin{subfigure}{\textwidth}
        % include second image
        \includegraphics[width=\textwidth]{images/latent_space_traversals/vae_dsprites_10000_rotation_\i.png}
    \end{subfigure}}
    \caption[10,000-\ac{VAE} - Rotation traversal]{Latent spaces traversal between different rotation values for 10,000-\ac{VAE} on the dsprites dataset}
    \label{fig:vae_dsprites_rotation_vae_10000}
\end{figure}

% TODO: Add links to models

Figure~\ref{fig:vae_dsprites_rotation_vae_10000} shows the latent space exploration between different rotation values of 10,000-\ac{VAE}\footnote{\say{dSprites-\ac{VAE}}, see Section~\ref{subsubsec:vae_models}} with a reconstruction term weight of 10,000 on the dsprites dataset (the plots for the other models can be found in Appendix~\ref{sec:appendix_plots_latent_space_traversals}).
The 10,000-\ac{VAE} learns a transition between different rotation values.
This also holds for the other models (7,500-, 6,250-, 5000-, 3750-\ac{VAE}, see Appendix~\ref{sec:appendix_plots_latent_space_traversals}).

However, how are the different rotation angles represented in the latent space?

\begin{figure}
    \centering
    \begin{subfigure}{.3\textwidth}
        % include second image
        \includegraphics[width=\textwidth]{images/latent_space_traversals/vae_dsprites_orientation_latent_space.png}
        \caption{\textit{Ellipse} - only $\frac{1}{2}$ rotation is shown}
    \end{subfigure}
    \hfill
    \begin{subfigure}{.3\textwidth}
        % include second image
        \includegraphics[width=\textwidth]{images/latent_space_traversals/vae_dsprites_orientation_latent_space_heart.png}
        \caption{\textit{Heart} - only $\frac{1}{2}$ rotation is shown}
    \end{subfigure}
    \hfill
    \begin{subfigure}{.3\textwidth}
        % include second image
        \includegraphics[width=\textwidth]{images/latent_space_traversals/vae_dsprites_orientation_latent_space_square.png}
        \caption{\textit{Square} - only $\frac{1}{4}$ rotation is shown}
    \end{subfigure}
    \caption[10,000-\ac{VAE} - Rotation latent space]{\ac{PCA}-transformed latent space positions of different dsprites shapes with a fixed position, averaged over scales and a 10,000-\ac{VAE} where only rotation is changed between objects. Increasing color values correspond to an increase in rotation. }
    \label{fig:vae_dsprites_rotation_latent_space_vae_10000}
\end{figure}

Figure~\ref{fig:vae_dsprites_rotation_latent_space_vae_10000} shows the latent space position of an ellipse with a fixed position, averaged over different scales for different rotations.
For this plot, the ten-dimensional latent space was reduced to a two-dimensional using \ac{PCA} on the vector values.
Only half of the rotations are shown (i.e., rotations in $[0;\pi]$).
Rotations in $[\pi; 2\pi]$ fill the circle a second time.
The behavior does not qualitatively change for the other models (7,500-, 6,250-, 5000-, 3750-\ac{VAE}), neither for different shapes.
Rotations are learned naturally;
A circle in the (reduced) latent space corresponds to a rotation of the object.
For \textit{Square}, the shape is the same after rotating by $\frac{1}{4}\pi$.
The model captures this property - the circle is traveled once after a $\frac{1}{4}\pi$-rotation (in contrast to \textit{Ellipse}, where a $\frac{1}{2}\pi$-rotation is required).

Something similar can be observed for the position.
\begin{figure}
    \centering
    \begin{subfigure}{.48\textwidth}
        % include second image
        \includegraphics[width=\textwidth]{images/latent_space_traversals/vae_10000_dsprites_latent_space_values_x_position.png}
        \caption{Traversal of the reduced latent space for different $x$-positions.}
        \label{subfig:vae_dsprites_x_pos_latent_space_route}
    \end{subfigure}
    \begin{subfigure}{.48\textwidth}
        % include second image
        \includegraphics[width=\textwidth]{images/latent_space_traversals/vae_10000_dsprites_latent_space_values_y_position.png}
        \caption{Traversal of the reduced latent space for different $y$-positions.}
        \label{subfig:vae_dsprites_y_pos_latent_space_route}
    \end{subfigure}
    \caption[\ac{VAE} on dsprites: Latent Space Values]{Latent space of 10,000-\ac{VAE} trained on dsprites. Different values are either for different $x$-positions or for different $y$-position. The other position is fixed to 1.0. It is averaged over all other parameters.}
    \label{fig:vae_dsprites_latent_space_x_position}
\end{figure}

\begin{figure}
    \centering
    \includegraphics[width=\textwidth]{images/latent_space_traversals/vae_dsprites_7500_position.png}
    \caption{Linear interpolation between \textit{top-left} and \textit{bottom-right} 7,500-\ac{VAE} latent space representations on dsprites leads to areas of low probability density.}
    \label{fig:vae_7500_dsprites_position_interpolation}
\end{figure}

Figure~\ref{fig:vae_dsprites_latent_space_x_position} shows the path in a reduced latent space\footnote{Plots for the other models (7,500-, 6,250-, 5,000, 3750-\ac{VAE}, and \ac{VAE}-\ac{GAN}) can be found in Appendix~\ref{sec:appendix_plots_latent_space_traversals}.)}.
Each arrow is the difference between two successive $x$-, or $y$-positions, mapped into the two-dimensional space.
The path is curved, surprising as linear interpolations in the latent space are known to be successful for natural attributes such as hair color~(\citep{radford2016deep} and CelebA in this section).
Linear interpolation for the position, however, leads to regions of low probability density (see Figure~\ref{fig:vae_7500_dsprites_position_interpolation}).
Figure~\ref{subfig:vae_dsprites_x_pos_dim_values} shows the values for the different dimensions in the non-reduced latent space.
Some values almost resemble a sine-curve.
This behavior is qualitatively the same for the other models (7,500-, 6,250-, 5000-, 3750-\ac{VAE}).

Like rotation for which this has been reported previously~\citep{chen2018isolating}, \acp{VAE} seem to encode position in a periodic manner.

Regarding the human visual system, a high-level model of the ventral stream should not capture object position, scale, or rotation at all.
Capturing orientation and position of a whole object is instead a property of the dorsal stream.
However, by definition of the loss function\footnote{This holds for the pixel-wise and the adversarial loss.}, \acp{VAE} are trained to encode these properties.
One way to make \acp{VAE} agnostic of these factors is to center objects in the picture as in the CelebA dataset.
However, even for CelebA, the models learn head rotation as one factor of variation.

Regarding vision, there is no reason to assume that \acp{VAE} are a better model of the ventral than of the dorsal stream.
Moreover, \acp{VAE} seem to learn positional factors of variation differently from non-positional ones.
It has been shown that, for factors of variation such as \say{hair color} or \say{gender} (for the CelebA dataset), a simple linear traversal in the latent space is sufficient to interpolate between these factors.
Positional factors of variation, however, seem to be learned differently.
Here, a linear interpolation is misleading because a highly curved interpolation would be required.

This difference in handling positional and non-positional attributes could be related to the two-stream hypothesis where such attributes are also treated differently (see Section~\ref{subsubsec:visual-cortex}).
In any case it has to be considered in the work with \acp{VAE}.

For the dSprites-\acp{VLAE} model, a similar analysis allows us to identify which layers encode which factor of variation\footnote{The plots for the corresponding dSprites-\ac{VLAE}-\ac{GAN} model can be found in Appendix~\ref{sec:appendix_plots_latent_space_traversals}.}.
\begin{figure}
    \centering
    \begin{subfigure}{.48\textwidth}
        \centering
        \includegraphics[width=\textwidth]{images/latent_space_traversals/vlae_dsprites_left_latent_space_values.png}
        \caption{Varying $x$-position}
        \label{subfig:vlae_dsprites_latent_space_values_x}
    \end{subfigure}
    \hfill
    \begin{subfigure}{.48\textwidth}
        \centering
        \includegraphics[width=\textwidth]{images/latent_space_traversals/vlae_dsprites_bottom_latent_space_values.png}
        \caption{Varying $y$-position}
        \label{subfig:vlae_dsprites_latent_space_values_y}
    \end{subfigure}
    \vfill
    \begin{subfigure}{.48\textwidth}
        \centering
        \includegraphics[width=\textwidth]{images/latent_space_traversals/vlae_dsprites_orientation_latent_space_values.png}
        \caption{Varying orientation}
        \label{subfig:vlae_dsprites_latent_space_values_orientation}
    \end{subfigure}
    \hfill
    \begin{subfigure}{.48\textwidth}
        \centering
        \includegraphics[width=\textwidth]{images/latent_space_traversals/vlae_dsprites_scale_latent_space_values.png}
        \caption{Varying scale}
        \label{subfig:vlae_dsprites_latent_space_values_scale}
    \end{subfigure}
    \vfill
    \begin{subfigure}{.48\textwidth}
        \centering
        \includegraphics[width=\textwidth]{images/latent_space_traversals/vlae_dsprites_shape_latent_space_values.png}
        \caption{Varying shape}
        \label{subfig:vlae_dsprites_latent_space_values_shape}
    \end{subfigure}
    \caption[\ac{VLAE} on dsprites: Latent Space Values]{Values of different dimensions and layers in the \ac{VLAE} latent space for different factor of variation values (first row in each subplot), and position in a \ac{PCA}-reduced latent space (second row in each subplot). The model was trained on dsprites. The left column corresponds to the first embedding layer, the right one to the third. \ac{PCA} was performed separately for each factor of variation and latent space layer. Different values correspond to different values for the respective factor of variation. For orientation and scale, the position is fixed to 0.0 in both directions, shape is fixed to \textit{Square}. For $x$-and $y$-postition, the other position is fixed to 1.0 and shape is fixed to \textit{Square}.}
    \label{fig:vlae_dsprites_latent_space_values}
\end{figure}
Figure~\ref{fig:vlae_dsprites_latent_space_values} shows the $\bm{z}$ values of different layers and dimensions for different factors of variation of dSprites-\ac{VLAE}.
The third layer seems to encode most of the factors of variation: $x$-, and $y$-position, scale, and shape.
However, the first and second layers also encode orientation (Figures~\ref{subfig:vlae_dsprites_latent_space_values_x} and ~\ref{subfig:vlae_dsprites_latent_space_values_y}) even though less strong.
The representation, therefore, is not separated;
The different layers do not independently encode different factors of variation even though these are, by definition, independent in the dsprites dataset.
Independence, in the case of \textsc{Mnist}, is further discussed in Section~\ref{subsec:independence-of-vlae-embeddings}.

The first and second layers jointly learn orientation, similarly to the dSprites-\ac{VAE}.
The scale, and the $x$- and $y$-positions are mainly learned in the third layer.
Again, the latent space trajectories for the $x$- and $y$-positions show a curved path, even though less circular compared to dSprites-\ac{VAE}.
Importantly, the \ac{VLAE} has three four-dimensional latent spaces.
However, the reconstructions are worse compared to the ten-dimensional-latent-space dSprites-\ac{VAE}.
dSprites-\acp{VLAE}, therefore, do not seem to use the hierarchical latent spaces efficiently under all circumstances.

Qualitatively, dSprites-\ac{VAE}-\ac{GAN} and dSprites-\ac{VLAE}-\ac{GAN} behave similarly for the considerations in this section.
They, therefore, are not discussed.

By analyzing the latent spaces of different models (\ac{VAE} and \ac{VLAE} models) on different datasets (dsprites and \textsc{Mnist} with Morpho-\textsc{Mnist}), it has been shown that \ac{VAE}- and \ac{VLAE}-models partially learn curved latent space trajectories for ordered latent space attributes (Figures~\ref{fig:vlae_mnist_morpho_latent_space_values},~\ref{fig:vae_dsprites_latent_space_x_position}, and~\ref{fig:vlae_dsprites_latent_space_values}).
It has empirically been validated that a simple linear interpolation between different orientation values in the latent space of a \ac{VAE} trained on dsprites does not capture the learned \say{orientation-trajectory}.
The latent space is highly entangled for positional attributes, even if the reconstruction term is extremely low (see Appendix~\ref{sec:appendix_plots_latent_space_traversals}).
Therefore, it is argued that linear interpolations sometimes do not capture the learned trajectory.
Interpolating linearly can lead to wrong conclusions regarding the model performance.

% TODO: Write about latent space entangledment

\subsection{Latent Space Separability on Generated Images}\label{subsec:independence-of-vlae-embeddings}

The VLAE learns embeddings on different levels.
For \textsc{Mnist}, \citet{zhao2017learning} used three two-dimensional layers to learn image semantics of different granularity.
They claim that their model can learn disentangled hierarchical features.
Figure~\ref{subfig:vlae_mnist_latent_space_traversal} shows reconstructions of this model when systematically exploring one dimension and randomly choosing the others.
Evidently, the model can learn disentangled representations to some extent.

For example, $z_1$ seems to mainly encode the digit thickness, whereas $z_3$ seems to encode digit identity.
For $z_2$, however, it seems also to influence the digit identity.
Therefore, it is not apparent how disentangled the representations are.
\begin{breakablealgorithm}
    \caption{Generating Layer Representative Samples by Averaging Out Other Embedding Layers}\label{alg:layer_representative_samples}
    \begin{algorithmic}[1]
        \Function{LayerRepresentativeSamples}{numSamples,numApproximations}
            \State $j \gets 0$
            \State $\mathcal{L}\gets \varnothing$
            \While{$i < \text{numSamples}$}
                \State $\bm{v} \gets \bm{v} \sim \mathcal{N}(\bm{0}, \bm{I})$\label{line:fixing_v}
                \ForAll{$j \in \{1,2,3\}$}
                    \State $\bm{s}_j \gets$ \Call{LayerRepresentativeSample}{$\bm{v}$, numApproximations, $j$}
        \EndFor
        \State $\mathcal{L} \gets \mathcal{L} \cup \{\{\bm{s}_1, \bm{s}_2, \bm{s}_3\}\}$

        \EndWhile
        \State \Return $\mathcal{L}$
        \EndFunction

        \Function{LayerRepresentativeSample}{fixedDimensionValue, numApproximations, dimensionIndex}
        \State $\mathcal{D} \gets \{1,2,3\}$
        \State $\alpha \gets \text{fixedDimensionValue}$
        \State $\beta \gets (D \setminus \text{dimensionIndex})_1$
        \State $\gamma \gets (D \setminus \text{dimensionIndex})_2$
        \State $\bm{z}_{\alpha} \gets \bm{a} \sim \mathcal{N}(\bm{0}, \bm{I})$
        \State $\mathcal{L}\gets \varnothing$
        \State $i \gets 0$
        \While{$i < \text{numApproximations}$}
        \State $\bm{z}_{\beta}^i \gets \bm{b}_i \sim \mathcal{N}(\bm{0}, \bm{I})$
        \State $\bm{z}_{\gamma}^i \gets \bm{c}_i \sim \mathcal{N}(\bm{0}, \bm{I})$
        \State $\mathcal{L} \gets \mathcal{L} \cup \{$ \Call{VLAE-Decoder}{$\bm{z}_{\alpha}, \bm{z}_{\beta}^j, \bm{z}_{\gamma}^j$} $\}$
        \State $i \gets i + 1$
        \EndWhile
        \State \Return $\frac{1}{|\mathcal{L}|}\sum_j \mathcal{L}_j$
        \EndFunction
    \end{algorithmic}
\end{breakablealgorithm}

Consider Algorithm~\ref{alg:layer_representative_samples}.
The function \textsc{VLAE-Decoder} calls the decoder of the \ac{VLAE}, i.e., $p_\theta(\bm{x} | \bm{z}_1, \bm{z}_2, \bm{z}_3)$.
Calling the function \textsc{LayerRepresentativeSamples} returns an ordered set $\mathcal{L}$ of \say{layer representative samples.}
Each sample $i$ contains three items, say $\bm{x}_1^i, \bm{x}_2^i, \bm{x}_3^i$ that were created by fixing a value $\bm{v}$ in Line~\ref{line:fixing_v} of Algorithm \ref{alg:layer_representative_samples}.
What is meant by the term \say{layer representative samples} is that for example $\bm{x}_1^i$ is approximately drawn from the marginal distribution
\begin{align}
    \bm{x}_1^i \sim p_\theta(\bm{v} | \bm{z}_1) = \int_{\bm{z}_2} \int_{\bm{z}_3} p_\theta(\bm{v} | \bm{z}_1, \bm{z}_2, \bm{z}_3) d\bm{z}_2 d\bm{z}_3.
\end{align}
Now, if the embedding layers learn disentangled hierarchical representations, $p_\theta(\bm{x} | \bm{z}_1)$, $p_\theta(\bm{x} | \bm{z}_2)$, and $p_\theta(\bm{x} | \bm{z}_3)$ should be pairwise statistically independent.
Specifically,
\begin{align}
    p_\theta(\bm{x} | \bm{z}_i) \not \propto p_\theta(\bm{x} | \bm{z}_j) \quad \forall (i,j):i\neq j. \label{eq:notprop}
\end{align}

However, this is not true.
Choosing a value $\bm{z}_1 = \varphi$ such that $p_\theta(\bm{x} | \bm{z}_1 = \varphi)$ also leads to a high value of $p_\theta(\bm{x} | \bm{z}_2 = \varphi)$, leading to a violation of Equation \ref{eq:notprop}.

\begin{figure}
    \centering
    \includegraphics[width=\textwidth]{images/notprop/mnist/vlae/ccs_0_1_vlae.png}
    \caption[\textsc{Mnist}-\ac{VLAE} - Pixel intensity correlation]{Correlation of pixel intensities when fixing $\bm{z}_1 = \bm{z}_2=\varphi$ for \textsc{Mnist}-\ac{VLAE}. Each box represents one of $28\times 28$ \textsc{Mnist} pixels. The $x$-axis of each box encodes the mean pixel intensity of the $\bm{z}_1$-representative sample. The $y$-axis encodes the mean pixel intensity of the of the $\bm{z}_2$-representative sample. Dots within boxes belong to the same $\varphi$ for both, $\bm{z}_1$ and $\bm{z}_2$. }
    \label{fig:notprop}
\end{figure}

\begin{figure}
    \centering
    \begin{subfigure}{.3\textwidth}
        \includegraphics[width=\textwidth]{images/notprop/dsprites/vlae/dim_1_2.png}
        \caption{\textsc{Mnist}-\ac{VLAE} - Layer 1 vs. Layer 2}
    \end{subfigure}
    \hfill
    \begin{subfigure}{.3\textwidth}
        \includegraphics[width=\textwidth]{images/notprop/dsprites/vlae/dim_1_3.png}
        \caption{\textsc{Mnist}-\ac{VLAE} - Layer 1 vs. Layer 3}
    \end{subfigure}
    \hfill
    \begin{subfigure}{.3\textwidth}
        \includegraphics[width=\textwidth]{images/notprop/dsprites/vlae/dim_2_3.png}
        \caption{\textsc{Mnist}-\ac{VLAE} - Layer 2 vs. Layer 3}
    \end{subfigure}
    \begin{subfigure}{.3\textwidth}
        \includegraphics[width=\textwidth]{images/notprop/dsprites/vlae_gan/dim_1_2.png}
        \caption{\textsc{Mnist}-\ac{VLAE}-\ac{GAN} - Layer 1 vs. Layer 2}
    \end{subfigure}
    \hfill
    \begin{subfigure}{.3\textwidth}
        \includegraphics[width=\textwidth]{images/notprop/dsprites/vlae_gan/dim_1_3.png}
        \caption{\textsc{Mnist}-\ac{VLAE}-\ac{GAN} - Layer 1 vs. Layer 3}
    \end{subfigure}
    \hfill
    \begin{subfigure}{.3\textwidth}
        \includegraphics[width=\textwidth]{images/notprop/dsprites/vlae_gan/dim_2_3.png}
        \caption{\textsc{Mnist}-\ac{VLAE}-\ac{GAN} - Layer 2 vs. Layer 3}
    \end{subfigure}
    \caption[\textsc{Mnist}-\ac{VLAE} and \textsc{Mnist}-\ac{VLAE}-\ac{GAN} - Pixel intensity correlation]{Histogram of correlations of pixel-wise intensities when fixing different pairs of dimensions for \textsc{Mnist}-\ac{VLAE} and \textsc{Mnist}-\ac{VLAE}-\ac{GAN}.
    Each value in the histogram is the Pearson correlation coefficient for one pixel of a generated dsprites image (compare Figure~\ref{fig:notprop} showing the correlations in the case of \textsc{Mnist}).
    The histogram is colored by if the correlation is significant ($H_0$: The absolute correlation is equal to the absolute correlation of samples with zero correlation) at a 95\% confidence level.
    The $p$-values have be corrected by the Benjamini/Hochberg method.}
    \label{fig:mnist_vlae_notprop}
\end{figure}

\subsubsection{Mnist}

Consider Figure~\ref{fig:notprop}.
It shows results of samples $(\bm{x}_1^1,\bm{x}_2^1),\dots,(\bm{x}_1^{100},\bm{x}_{100}00)$ that were generated by Algorithm~\ref{alg:layer_representative_samples}.
Thus, the parameter \say{numSamples} is chosen as 100, and the parameter \say{numApproximations} is chosen as 300.
Each $\bm{x}_i^j$ is one generated \textsc{Mnist} image of size $28\times 28$ pixels.
Each box in Figure~\ref{fig:notprop} corresponds to one of these pixels.
The box index corresponds to the pixel in the \textsc{Mnist} image.
The $x$-values of dots in the same box then correspond to $\bm{x}_1^1\big|_{(3,1)}, \dots, \bm{x}_1^{100}\big|_{(3,1)}$, i.e., the pixel intensities of one specific pixel (here: third row, first column (3,1)) over all 100 samples for $\bm{x}_1$, i.e., the sample generated by fixing $\bm{z}_1$.
Analogously, the $y$-values of dots in the same box correspond to $\bm{x}_2^1\big|_{(3,1)}, \dots, \bm{x}_2^{100}\big|_{(3,1)}$.
Each dot corresponds to one fixed $\varphi$-value.

If changing the value of $\bm{z}_1$ was independent of changing the value of $\bm{z}_2$, the boxes should show no trend.
This is true for outer boxes.
They correspond to pixel values that always close to zero.
Therefore the values are in the bottom left corner, and no correlation can be observed.
For center pixels, however, the plot shows something different.
They show an interesting correlation pattern of negative and positive correlations.

Negative correlations for a pixel with index $(i,j)$ indicate that $p_\theta(\bm{x}\big|_{(i,j)} | \bm{z}_1 = \varphi) \propto \frac{1}{p_\theta(\bm{x}\big|_{(i,j)} | \bm{z}_1 = \varphi)}$, i.e., choosing a value $\varphi$ that leads to a high $(i,j)$-pixel intensity for the $\bm{z}_1$-representative sample leads to a low intensity of the same pixel of the corresponding $\bm{z}_1$-representative sample.
This, however, is a violation of equation~\ref{eq:notprop}.
Importantly, the correlation only persists for individual pixels.

The qualitative behavior is similar for all \ac{VLAE} models.
All plots can be found in Appendix~\ref{sec:additional-plots-for-section_independence}.

Figure~\ref{fig:mnist_vlae_notprop} shows a histogram of Pearson correlation coefficients for the pixel-wise intensities (compare Figure~\ref{fig:notprop})
The correlations are colored by whether the (corrected) $p$-values are $< 0.05$.
Apparently, the \ac{GAN}-models learn representations more independently in the different layers.
This could have two reasons.
First, the \ac{GAN}-models could disregard the lower layers more strongly than the non-\ac{GAN} models, using effectively only the third layer.
Second, the \ac{GAN}-models could in fact learn more independent representations in terms of image generation.

It is probably not the case that the \ac{GAN}-models disregard lower layers, because they seem to incorporate lower-layer information for the generation of new images (compare Figures~\ref{subfig:vlae_gan_mnist_latent_space_traversal} and \ref{fig:appendix_vlae_gan_mnist_latent_space_morpho}).
Therefore, it is assumed that the discriminative loss leads to more independent generated images.

\subsubsection{dsprites}

\begin{figure}
    \centering
    \begin{subfigure}{.3\textwidth}
        \includegraphics[width=\textwidth]{images/notprop/dsprites/vlae/dim_1_2.png}
        \caption{dSprites-\ac{VLAE} - Layer 1 vs. Layer 2}
    \end{subfigure}
    \hfill
    \begin{subfigure}{.3\textwidth}
        \includegraphics[width=\textwidth]{images/notprop/dsprites/vlae/dim_1_3.png}
        \caption{dSprites-\ac{VLAE} - Layer 1 vs. Layer 3}
    \end{subfigure}
    \hfill
    \begin{subfigure}{.3\textwidth}
        \includegraphics[width=\textwidth]{images/notprop/dsprites/vlae/dim_2_3.png}
        \caption{dSprites-\ac{VLAE} - Layer 2 vs. Layer 3}
    \end{subfigure}
    \begin{subfigure}{.3\textwidth}
        \includegraphics[width=\textwidth]{images/notprop/dsprites/vlae_gan/dim_1_2.png}
        \caption{dSprites-\ac{VLAE}-\ac{GAN} - Layer 1 vs. Layer 2}
    \end{subfigure}
    \hfill
    \begin{subfigure}{.3\textwidth}
        \includegraphics[width=\textwidth]{images/notprop/dsprites/vlae_gan/dim_1_3.png}
        \caption{dSprites-\ac{VLAE}-\ac{GAN} - Layer 1 vs. Layer 3}
    \end{subfigure}
    \hfill
    \begin{subfigure}{.3\textwidth}
        \includegraphics[width=\textwidth]{images/notprop/dsprites/vlae_gan/dim_2_3.png}
        \caption{dSprites-\ac{VLAE}-\ac{GAN} - Layer 2 vs. Layer 3}
    \end{subfigure}
    \caption[dSprites-\ac{VLAE} and dSprites-\ac{VLAE}-\ac{GAN} - Pixel intensity correlation]{Histogram of correlations of pixel-wise intensities when fixing different pairs of dimensions for dSprites-\ac{VLAE} and dSprites-\ac{VLAE}-\ac{GAN}.
    Each value in the histogram is the Pearson correlation coefficient for one pixel of a generated dsprites image (compare Figure~\ref{fig:notprop} showing the correlations in the case of \textsc{Mnist}).
    The histogram is colored by if the correlation is significant ($H_0$: The absolute correlation is equal to the absolute correlation of samples with zero correlation) at a 95\% confidence level.
    The $p$-values have be corrected by the Benjamini/Hochberg method.}
    \label{fig:dsprites_vlae_notprop}
\end{figure}

Consider Figure~\ref{fig:dsprites_vlae_notprop} showing again the histograms of pixel-wise intensity correlations.
The Figure supports the reasoning of the previous paragraph, i.e., that the \ac{GAN}-generated images are more independent in terms of pixel-wise intensity correlation.

\subsection{Pixel-wise Distribution of Generated Images}\label{subsubsec:pixel_wise_statistics}

\acp{GAN} (see Section~\ref{subsubsec:representation_learning}) are trained by simultaneously training a \textit{generator} to create new samples and a \textit{discriminator} to discriminate between real and generated samples.
\acp{VAE} are another generative model that is not forced in the same way to create indistinguishable samples.
Instead, a reconstruction loss is used to force the reconstruction to be close to the real sample in terms of the difference in the pixel values.
Simultaneously, the \ac{KL}-loss and the reparametrization trick force the model to place similar samples close to another in a continuous Gaussian embedding space.
Therefore, drawing from the Gaussian embedding space should allow us to generate new samples similar to real samples.
However, the question remains how indistinguishable these generates samples are from actual samples.

Different statistical analyses were performed to address this question, revealing that generated samples can be correctly distinguished from actual samples.

\subsubsection{\textsc{Mnist}}\label{subsubsec:pixel_wise_distribution_mnist}

\begin{figure}
    \centering
    \includegraphics[width=.8\textwidth]{images/generated_vs_true/mnist/vlae_gan_kde.png}
    \caption[\textsc{Mnist}-\ac{VLAE}-GAN - Latent Space Distribution]{Histogram of mean $z$-values for different layers and dimensions of \textsc{Mnist}-\ac{VLAE}-\ac{GAN} (blue), the result of the \ac{KDE} (green), and a standard normal distribution (orange). Additional plots can be found in Appendix~\ref{sec:appendix_pixel_wise_statistics}.}
    \label{fig:vlae_gan_kde}
\end{figure}


\begin{figure}
    \centering
    \begin{subfigure}{0.4\textwidth}
        \centering
        \includegraphics[width=\textwidth]{images/generated_vs_true/mnist/mnist_vs_models_mean.png}
        \caption{Histogram of mean pixel intensities for different models}
        \label{subfig:mean_generated_vs_true}
    \end{subfigure}
    \hfill
    \begin{subfigure}{0.4\textwidth}
        \centering
        \includegraphics[width=\textwidth]{images/generated_vs_true/mnist/mnist_vs_models_sd.png}
        \caption{Histogram of pixel intensity standard deviations for different models}
        \label{subfig:sd_generated_vs_true}
    \end{subfigure}
    \hfill
    \begin{subfigure}{0.4\textwidth}
        \centering
        \includegraphics[width=\textwidth]{images/generated_vs_true/mnist/mnist_vs_models_skew.png}
        \caption{Histogram of pixel intensity skewness for different models}
        \label{subfig:skew_generated_vs_true}
    \end{subfigure}
    \hfill
    \begin{subfigure}{0.4\textwidth}
        \centering
        \includegraphics[width=\textwidth]{images/generated_vs_true/mnist/mnist_vs_models_kurt.png}
        \caption{Histogram of pixel intensity kurtosis for different models}
        \label{subfig:kurt_generated_vs_true}
    \end{subfigure}
    \caption[Models on \textsc{Mnist}: Pixel-wise distributions]{Pixel-wise distributions of different models and moments for the \textsc{Mnist} validation data set.}
    \label{fig:mean_generated_vs_true}
\end{figure}

\begin{figure}
    \centering
    \begin{subfigure}{0.4\textwidth}
        \centering
        \includegraphics[width=\textwidth]{images/generated_vs_true/mnist/mnist_vs_models_mean_gauss_post.png}
        \caption{Histogram of mean pixel intensities for different models}
        \label{subfig:mean_generated_vs_true_gauss_post}
    \end{subfigure}
    \hfill
    \begin{subfigure}{0.4\textwidth}
        \centering
        \includegraphics[width=\textwidth]{images/generated_vs_true/mnist/mnist_vs_models_sd_gauss_post.png}
        \caption{Histogram of pixel intensity standard deviations for different models}
        \label{subfig:sd_generated_vs_true_gauss_post}
    \end{subfigure}
    \hfill
    \begin{subfigure}{0.4\textwidth}
        \centering
        \includegraphics[width=\textwidth]{images/generated_vs_true/mnist/mnist_vs_models_skew_gauss_post.png}
        \caption{Histogram of pixel intensity skewness for different models}
        \label{subfig:skew_generated_vs_true_gauss_post}
    \end{subfigure}
    \hfill
    \begin{subfigure}{0.4\textwidth}
        \centering
        \includegraphics[width=\textwidth]{images/generated_vs_true/mnist/mnist_vs_models_kurt_gauss_post.png}
        \caption{Histogram of pixel intensity kurtosis for different models}
        \label{subfig:kurt_generated_vs_true_gauss_post}
    \end{subfigure}
    \caption[Models on \textsc{Mnist}: Pixel-wise distributions - Gaussian Posterior]{Pixel-wise distributions of different models and moments for the \textsc{Mnist} validation data set with a standard normal posterior.}
    \label{fig:mean_generated_vs_true_gauss_post}
\end{figure}

The following procedure was applied to generate samples for different models.
One part of the \ac{VAE} loss function (and of the other models) is the \ac{KL}-term, forcing the models to match a standard multivariate normal distribution.
The learned distribution, however, is not perfectly Gaussian because of the other loss function terms.
Therefore, the model's encoders first predicted the mean values in $z$-space for the 10,000 validation images of the \textsc{Mnist} dataset.
Subsequently, for each dimension, \ac{KDE} was performed to estimate the \ac{PDF} of the latent space for \textsc{Mnist}-\ac{VAE} and \textsc{Mnist}-\ac{VAE}-\ac{GAN}.
Since the learned distribution, by definition, is covariance-free, the \ac{KDE} was performed for each dimension separately.
The estimated \ac{PDF} was then used to generate 1,000 new images to perform the statistical analyses.
Figure~\ref{fig:vlae_gan_kde} shows the estimated \ac{KDE} for \textsc{Mnist}-\ac{VLAE}-\ac{GAN}, plots for the other models can be found in Appendix~\ref{subsec:appendix_pixel_wise_statistics_mnist}.

For \textsc{Mnist}-\ac{VLAE} and \textsc{Mnist}-\ac{VLAE}-\ac{GAN}, it cannot be sampled independently from the different latent spaces (see Section~\ref{subsec:independence-of-vlae-embeddings}).
For these models, the predicted mean values directly were used to reconstruct 10,000 images.

The \textsc{Mnist} test set of 10,000 images was compared to a 1,000 generated samples from \textsc{Mnist}-\ac{VAE}, \textsc{Mnist}-\ac{VLAE}, \textsc{Mnist}-\ac{VAE}-\ac{GAN}, and \textsc{Mnist}-\ac{VLAE}-\ac{GAN} according to the estimated posterior.
First, the mean pixel values, i.e.,~the mean over all $28\times 28$ pixel values, were compared (see Figure~\ref{fig:mean_generated_vs_true}).
The plot overlays the histograms of mean pixel values for the five conditions: \textsc{Mnist}, \textsc{Mnist}-\ac{VAE}, textsc{Mnist}-\ac{VAE}-\ac{GAN}, \textsc{Mnist}-\ac{VLAE}, and \textsc{Mnist}-\ac{VLAE}-\ac{GAN}.
The other plots in Figure~\ref{fig:mean_generated_vs_true} were created accordingly but for higher moments of the pixel value distributions.

Consider Figure~\ref{fig:mean_generated_vs_true_gauss_post} showing the same distribution but for the standard normal prior.
The pixel-wise statistics are quite different compared to Figure~\ref{fig:mean_generated_vs_true} and resemble the true distribution worse.
Therefore, it is argued that the above procedure is crucial when evaluating a model.
Sampling from the prior, in general, leads to worse model performance in resembling the data distribution.

Analyzing Figure~\ref{fig:mean_generated_vs_true} leads to the assumption that all models learn the pixel distributions to some extent.
The overlap of the histograms is high for all models.
\begin{table}
    \begin{tabular}{lrrrr}
        \toprule
        Model              & $p$-value (mean)    & $p$-value (sd)      & $p$-value (skewness) & $p$-value (kurtosis) \\
        \midrule
        \ac{VAE}           & $6.5\cdot 10^{-15}$ & 0.0                 & $3.7\cdot 10^{-257}$ & $1.4\cdot 10^{-70}$  \\
        \ac{VLAE}          & 0.439               & $4.1\cdot 10^{-98}$ & $1.1\cdot 10^{-8}$   & .0                   \\
        \ac{VAE}-\ac{GAN}  & $6.7\cdot 10^{-75}$ & 0.0                 & $9.3\cdot 10^{-48}$  & $2.2\cdot 10^{-63}$  \\
        \ac{VLAE}-\ac{GAN} & 0.0                 & $4.0\cdot 10^{-6}$  & 0.023                & 0.227                \\
        \bottomrule
    \end{tabular}
    \caption{$p$-values of a Mann-Whitney U test. Each cell tests the hypothesis that the respective moments for the respective model are equal to the values for the \textsc{Mnist} test images. For each cell, the sample size was $2\cdot 10,000$.}
    \label{tab:vae-vlae-mnist}
\end{table}
Table~\ref{tab:vae-vlae-mnist} shows the results of a two-sided Mann-Whitney $U$ test for the samples of moments of the pixel distributions.
The results lead to two conclusions: 1) The \ac{VLAE}-models capture the statistics of the pixel distribution better compared to the \ac{VAE}-models.
2) All models do not capture the true pixel distribution (rejection of $H_0$: \say{The distributions are the same.}).

To verify that none of the models generates indistinguishable samples, a discriminator network was trained to distinguish generated samples from true \textsc{Mnist} test images\footnote{The configuration of the discriminator network can be found in Appendix~\ref{sec:listing_discriminator_network}. The network was trained using binary cross entropy for one epoch using the Adam optimizer.}.
The discriminator network shows an accuracy of 1.0 for distinguishing for all models, i.e.,~it is perfectly able to predict generated from true samples for all models.

Comparing the pixel-wise statistics between generated (according to the \ac{KDE} procedure explained above) and reconstructed samples furthermore showed that there is a significant difference ($p < 0.05$, two-sided Mann-Whitney $U$ test) even though the difference between the $\bm{z}$ distribution of encoded validation images and $\bm{z}$s sampled from the estimated posterior is not significant ($p < 0.05$, two-sided Mann-Whitney $U$ test).
The reason for this is assumed to lie in samples for which the encoder predicts very low values $\log \sigma^2$ that have been observed during training.
Enforcing a lower bound for $\log \sigma^2$ in the encoder could prevent this.

\subsubsection{dsprites}

\begin{figure}
    \centering
    \begin{subfigure}{0.4\textwidth}
        \centering
        \includegraphics[width=\textwidth]{images/generated_vs_true/dsprites/dsprites_vs_models_mean.png}
        \caption{Histogram of mean pixel intensities for different models}
        \label{subfig:mean_generated_vs_true_dsprites}
    \end{subfigure}
    \hfill
    \begin{subfigure}{0.4\textwidth}
        \centering
        \includegraphics[width=\textwidth]{images/generated_vs_true/dsprites/dsprites_vs_models_sd.png}
        \caption{Histogram of pixel intensity standard deviations for different models}
        \label{subfig:sd_generated_vs_true_dsprites}
    \end{subfigure}
    \hfill
    \begin{subfigure}{0.4\textwidth}
        \centering
        \includegraphics[width=\textwidth]{images/generated_vs_true/dsprites/dsprites_vs_models_skew.png}
        \caption{Histogram of pixel intensity skewness for different models}
        \label{subfig:skew_generated_vs_true_dsprites}
    \end{subfigure}
    \hfill
    \begin{subfigure}{0.4\textwidth}
        \centering
        \includegraphics[width=\textwidth]{images/generated_vs_true/dsprites/dsprites_vs_models_kurt.png}
        \caption{Histogram of pixel intensity kurtosis for different models}
        \label{subfig:kurt_generated_vs_true_dsprites}
    \end{subfigure}
    \caption[Models on dsprites: Pixel-wise distributions]{Pixel-wise distributions of different models and moments for the dsprites validation data set.}
    \label{fig:mean_generated_vs_true_dsprites}
\end{figure}

\begin{table}
    \begin{tabular}{lrrrr}
        \toprule
        Model              & $p$-value (mean) & $p$-value (sd) & $p$-value (skewness) & $p$-value (kurtosis) \\
        \midrule
        \ac{VAE}           & $0.0$            & 0.0            & $3.6\cdot 10^{-22}$  & $6.0\cdot 10^{-24}$  \\
        \ac{VLAE}          & 0.0              & $0.0$          & $0.091$              & 0.005                \\
        \ac{VAE}-\ac{GAN}  & 0.0              & 0.0            & 0.0                  & 0.0                  \\
        \ac{VLAE}-\ac{GAN} & 0.0              & $0.0$          & 0.005                & 0.061                \\
        \bottomrule
    \end{tabular}
    \caption{$p$-values of a Mann-Whitney U test. Each cell tests the hypothesis that the respective moments for the respective model are equal to the values for the dsprites test images. For each cell, the sample size was $2\cdot 1,000$.}
    \label{tab:vae-vlae-dsprites}
\end{table}

Figure~\ref{fig:mean_generated_vs_true_dsprites} shows the pixel-wise distribution for the dsprites dataset.
The \ac{VLAE}-models (\ac{VLAE} and \ac{VLAE}-\ac{GAN}) employ a four-dimensional, the \ac{VAE} (\ac{VAE} and \ac{VAE}-\ac{GAN}) a ten-dimensional latent space.
Again, the \ac{VLAE}-models capture the true distribution better.
However, all distributions differ significantly from the true distribution (see Table~\ref{tab:vae-vlae-dsprites}).
The \ac{KDE} distributions can be found in Appendix~\ref{subsec:appendix_pixel_wise_statistics_dsprites}.

\subsection{Class-Distribution of Generated Images}\label{subsec:class-distribution-of-generated-images}

Do the \ac{VAE} and \ac{VLAE} models generate each number with the same probability as a number's probability in the training set as required by latent space disentanglement (see Section~\ref{subsec:feature-disentanglement})?
To answer this question, the following procedure was applied.

First, let each model generate a large number of images by drawing the latent space variable(s) $z$ from their estimated posterior (see Section~\ref{subsubsec:pixel_wise_statistics} for details).
Second, let a classifier\footnote{The classifier has shown to perform sufficiently well, see above.} classify each generated image.
This procedure was applied to \textsc{Mnist} only.
It is the only labeled dataset where the \acp{VAE} and \acp{VLAE} models have shown a promising performance and for the availability of labels.

\begin{figure}
    \centering
    \includegraphics[width=\textwidth]{images/generated_vs_true/mnist/class_distr.png}
    \caption{Distribution of generated class labels for different models.}
    \label{fig:generated_class_distribution}
\end{figure}

Figure~\ref{fig:generated_class_distribution} shows the class-distribution of generated images.
Even though it is sampled from the (estimated) posterior, the class distribution is quite uneven.
This aligns with Figures~\ref{fig:mnist_latent_space_traversal_vae} and \ref{fig:mnist_latent_space_traversal_vlae}.
The reason why the model maps some digits to very small subspaces, however, remains unclear.
If it was because there is less variation for a particular digit, i.e.,~the digit three is written very similarly in all cases, then all models should map threes to small subspaces.

However, \textsc{Mnist}-\ac{VAE} reproduces few threes whereas \textsc{Mnist}-\ac{VAE}-\ac{GAN} produces few fours.
Overall, the \ac{VLAE} models generate a more even distribution of generated digits.

\begin{figure}
    \centering
    \includegraphics[width=\textwidth]{images/generated_vs_true/mnist/morpho_distr.png}
    \caption{Distribution of Morpho- \textsc{Mnist} attributes for different models.}
    \label{fig:generated_morpho_distribution}
\end{figure}

Figure~\ref{fig:generated_morpho_distribution} shows the distribution of morphological attributes for generated digits.
The Figure is accordingly to Figure~\ref{fig:morpho_mnist_distribution}.
The bottom row in Figure~\ref{fig:morpho_mnist_distribution} is equivalent to the top row in Figure~\ref{fig:generated_morpho_distribution}.

The \ac{VLAE} models learn a smooth distribution for most attributes that approximately resembles the true distribution.
However, they fail in generating the same proportion of digits with a small width (i.e., the digit one).
The \ac{VAE} model (but not \ac{VAE}-\ac{GAN}) is more successful in that regard.
However, the \ac{VAE} distributions are less smooth compared to the \ac{VLAE} models.

\subsection{Feature Map Stripes}\label{subsec:feature-map-stripes}

\begin{figure}
    \centering
    \begin{subfigure}{0.3\textwidth}
        \centering
        \includegraphics[width=\textwidth]{images/stripes/original.jpg}
        \caption{The original image.}
        \label{subfig:stripes_original}
    \end{subfigure}
    \hfill
    \begin{subfigure}{0.3\textwidth}
        \centering
        \includegraphics[width=\textwidth]{images/stripes/leaky_re_lu_5.png}
        \caption{The feature maps after LeakyReLU 5}
        \label{subfig:lakyrelu5}
    \end{subfigure}
    \hfill
    \begin{subfigure}{0.3\textwidth}
        \centering
        \includegraphics[width=\textwidth]{images/stripes/max_pooling2d_3.png}
        \caption{The feature maps after max pooling of LeakyReLU 5}
        \label{subfig:maxpool}
    \end{subfigure}
    \caption[Feature map stripes]{The original image and the feature maps after passing the image through the network until after the specified layer. The stripe artifacts can be observed in many feature maps in Subfigure~\ref{subfig:lakyrelu5}. They vanish after max-pooling (Subfigure~\ref{subfig:maxpool}).}
    \label{fig:stripes}
\end{figure}


One observation made during the analysis of the networks was the emergence of striped artifacts in the network's feature maps (Figure~\ref{fig:stripes}).
These stripes were observed in the AlexNetVAE on ImageNet.
Te stripes are either always horizontal or vertical for one network type.
If they are vertical, they can appear on the left or right sides for the same network.
If they are horizontal, they appear either on the top or on the bottom of the same network.
The exact reason the networks show this behavior was not found, but it is assumed to be a combination of the following considerations.

\paragraph{Kernel Size}
The stripes were only observed for large kernel sizes.
AlexNet and AlexNet \ac{VAE} use $11\times 11$ kernels in the first layer.
Using such large kernels with zero-padding (see next paragraph) causes the convolution operation to incorporate many zero-terms.
For feature map pixels at borders and especially in corners, only a few non-zero values are contributing to the convolution.
This results in overall low feature map values for these pixels.

\paragraph{Padding}
The stripes occur due to the zero-padding in the network and the resulting contrast observed by the convolutional filters.
Take Figure~\ref{fig:stripes}.
Here, the stripes appear most evident on the bottom left of the image.
The stripes indicate less active regions in the feature map: The feature map has an activity of around zero everywhere except for the location of the stripes.
Here, the activity is strongly negative.

A comparison with the original image (Figure~\ref{subfig:stripes_original}) shows that for the left side of the image, the contrast is highest on the bottom if the image is zero-padded\footnote{Zero-padding can be understood as adding black pixels around the image.}.
The contrast is high on the bottom of the image, the right side, and the right side of the top of the image.
However, for this network, the stripes seem to occur for a sharp shift of black on the left to white on the right.

\begin{figure}
    \centering
    \foreach \n in {0,...,11}{
    \begin{subfigure}{0.05\textwidth}
        \frame{\includegraphics[width=\textwidth]{images/stripes/test_images/original\n.jpg}}
        \caption{}
        \label{subfig:test_images_stripes\n}
    \end{subfigure}
    \hfill
    }
    \caption[Feature Map Stripes - Test Images]{The test images used to analyze the networks behavior.}
    \label{fig:test_images_stripes}
\end{figure}

To better understand the behavior, the network was applied to a set of artificial test images (Figure~\ref{fig:test_images_stripes}).

\begin{figure}
    \centering
    \begin{subfigure}{0.45\textwidth}
        \centering
        \includegraphics[width=\textwidth]{images/stripes/test_img_9/leaky_re_lu_5.png}
        \caption{The original image.}
        \label{subfig:stipes_test_img_leakyrelu5}
    \end{subfigure}
    \hfill
    \begin{subfigure}{0.45\textwidth}
        \centering
        \includegraphics[width=\textwidth]{images/stripes/test_img_9/max_pooling2d_3.png}
        \caption{The feature maps after LeakyReLU 5}
        \label{subfig:stripes_test_img_maxpool3}
    \end{subfigure}
    \caption[Feature Map Stripes on Test Images]{The feature maps with respect to test image~\ref{subfig:test_images_stripes8}.}
    \label{fig:stripes_test_img}
\end{figure}

Subfigure~\ref{subfig:test_images_stripes8} and~\ref{subfig:test_images_stripes11} turned out to be most insightful.
Figure~\ref{fig:stripes_test_img} shows the feature map after the last LeakyReLU (LeakyReLU 5) activation of the network given Image~\ref{subfig:test_images_stripes8} (Subfigure~\ref{subfig:stipes_test_img_leakyrelu5}) as well as the feature map after max-pooling of this feature map (Subfigure~\ref{subfig:stripes_test_img_maxpool3}).
Many things can be observed in Subfigure~\ref{subfig:stipes_test_img_leakyrelu5}.
Firstly, the feature map in the first column of the fifth row resembles the input stimulus itself.
The fact that this feature map shows most of the activity (especially after max-pooling, see Subfigure~\ref{subfig:stripes_test_img_maxpool3}) has been observed for natural images as well, however not the resemblance of the input stimulus.
Except for this feature map, stripes emerge either on the bottom of the left side or the bottom of the right side of the feature map.
For the bottom of the left side, this is where the sharp black-white contrast is.
The bottom of the right side is more complicated.
Here the test image was black, and the \say{contrast} is a black-black contrast - or no contrast.
However, this is only true for the first feature map\footnote{The first feature maps are not shown here.}.
The following feature maps, again, are zero-padded.
However, due to the bias term in the convolutions, these might be non-zero in the bottom-right and top-left square of the image, thus leading to a contrast.
This explains why the network can be sensitive towards these black-black contrasts in the input image.

Noteworthy, if the bias term in the convolutions is removed, the black-black contrast sensitivity vanishes because the network is not able anymore to add a constant to the black pixel values on the bottom-right or top-left of the image.
For this purpose, the batch normalization has to be removed, too, since it uses a bias term.
This, however, does not qualitatively change the network's behavior on real images.

\paragraph{Network Depth}
The network depth is assumed to play a role in the emergence of feature map stripes.
As explained above, the emergence of feature maps is an amplifying process.
Low-developed feature map stripes in lower layers lead to highly developed feature map stripes in higher layers.

This effect was, to the best of the author's belief, not reported previously.
It seems to have no significant influence on network performance.
However, it could depending on the implementation, lead to a loss of precision as feature maps contain very different float values.
Floating-point representations then need a larger exponent and have less storage for the mantissa.

\paragraph{Dataset}
The stripes were present for the ImageNet dataset but not for CelebA.
For CelebA, AlexNetVAE was able to reproduce and generate good images.
For ImageNet, however, the network produced and generated blurry images that only remotely are related to the training data.
It is therefore assumed that training on ImageNet leads to a high reconstruction error and, as a result of this, to large gradients for the whole training time.
Together with the previous considerations, such high gradients are assumed to lead to the feature map stripes.
They probably lead to more extreme weight configurations than for non-failure modes (e.g., CelebA).

\subsection{Pixelwise vs. Adversarial Loss}\label{subsec:pixelwise-vs.-adversarial-loss}
This thesis employed two different methods to train the decoder to generate natural looking images, namely the pixel-wise or the generative loss.
Furthermore, the encoder of the generative models was trained such that the hidden-layer activity of the discriminator is similar for generated and real images (in addition to the KL-term).

Advantages, disadvantages, and use-cases for both loss functions are discussed in the following.

First of all, both loss functions do not seem to be biologically plausible as they are not founded in Hebbian learning.
For the encoder loss, it could be argued that it is trained to elicit a \say{neural response} in the discriminator, similar to the one for real images.
This \say{activity matching} seems to be somewhat related to Hebbian learning as it explicitly considers the amount of activity on the level of single neurons.
However, discriminator and encoder itself are trained using backpropagation.

\citet{larsen2015autoencoding} state that the pixel-wise loss can lead to very high values even for small translation (see Section~\ref{subsubsec:representation_learning}).
Intuitively seeming like a disadvantage, this is only true for images with a high frequency and a notable variance of pixel values.
A black-and-white image consisting of alternating black and white rows and the same image shifted by one pixel in the $y$-direction would result in a maximum pixel-wise loss as the two images are orthogonal.
However, natural images are usually less susceptible to a sizeable pixel-wise loss for small translation because there are fewer regions of high contrast.
In the context of natural images, the pixel-wise loss function, for example, enables the model to detect if an object is placed in another corner of the image as the loss would increase for such high translations.
However, if the aim is to generate more realistic images, the generative loss should be used since the pixel-wise loss leads to blurry reconstructions.

Another consideration is training stability.
Due to the adversarial loss, training \acp{GAN} has many difficulties (mode collapse, error oscillation, see Section~\ref{subsubsec:gans}), often resulting in failure modes.
Training regular \acp{VAE}, in contrast, is far more stable.

The choice of the loss function might also play an essential role in studying a model's representation with human \ac{IT} representations.
Compared to the pixel-wise loss, the generative loss allows for a more holistic image assessment.
This might play an important role in the emergence of a models' latent representation, i.e., might lead to a more realistic latent representation in terms of human IT activations.
This aspect should be investigated in future work.

\subsection{Model Limitations}\label{subsec:model-limitations}
All networks investigated in the course of this thesis are \acp{CNN}.
Network design has to be chosen for each network type (see Section~\ref{subsec:models}).
Some of these hyperparameters are context-dependent.
A network operating on \textsc{Mnist}, for example, receives inputs of size $28\times 28$ pixels.
Such networks must not be as deep as a network operating on ImageNet, as fewer layers are sufficient for the network to capture the image in its entirety.
Other hyperparameters are chosen based on previous research.
Some configurations are known to be better for a specific task than others (see Section~\ref{subsec:models}).

Another consideration in the context of computational neuroscience is the biological plausibility.
This thesis aims at answering how far \acp{VAE} are reasonable models of the human visual cortex.
To do this, the model structure must allow comparisons with empirical data for the parts under investigation.

Specifically, lower model layers are compared with earlier, higher model layers with later regions in the visual cortex.
Furthermore, the models are chosen such that similar inputs translate to similar encodings.
The biological plausibility of the encoding should be further investigated in future research.

However, due to practical reasons such as the availability of training data or computational resources, the model disregards the human body's actualities, a more refined model should incorporate.

Other than in the models used in this thesis, the human eye receives a stream of visual stimuli.
Perceiving movements of animate objects could play an important role in distinguishing between animate and inanimate objects.
The dissimilarity between semantic representations of animate and inanimate objects has been shown in \citet{khaligh2014deep}.
Recurrent models such as LSTMs could extend the models developed in this thesis to sequences of images.

This thesis only considered image data to form semantic representations, even though the human mind perceives the world through more senses.
Future research could aim at investigating if and by how far semantic representations can be improved by using additional sources of information, such as sound.

\subsection{Top-Down Connections}\label{subsec:feedback-connections-of-the-lateral-geniculate-nucleus}
As discussed in Section~\ref{subsubsec:visual-cortex}, the \ac{LGN} and \ac{V4} receive input from top-down connections.
The top-down connections into \ac{V4} are assumed to enable attentional mechanisms~\citep{roe2012toward}.

Similarly, the \ac{LVAE}-model (see Section~\ref{subsubsec:representation_learning}) employs top-down connections.
In that regard, it is biologically more plausible compared to the original \ac{VAE} models.
\citet{sonderby2016ladder} claim that the top-down pass helps improving the low-level feature representation because it enables the model to incorporate the higher-level context.
Even though \citet{zhao2017learning} discuss disadvantages of this model, top-down connections enabling attentional mechanisms could play an important role designing more biologically plausible \ac{VAE} models.

Future work should investigate if such top-down connections are allowing visual regions operating on a lower semantic level to incorporate the high-level context to disambiguate the lower-level representations.

\subsection{Possible applications of VLAE-GAN model}\label{subsec:possible-applications-of-vlae-gan-model}
\citet{vanrullen2019reconstructing} learn a mapping between test subject's fMRI responses when shown images from the CelebA dataset and the latent space of a \ac{VAE}.
They discuss which brain regions have most influence in the mapping, showing that the occipital lobe contributes most and the temporal and frontoparietal Cortex less.

The \ac{VLAE}(-\ac{GAN}) used in the course of this thesis uses three latent spaces compared to one latent space for a regular \ac{VAE}(-\ac{GAN}).
Furthermore it is designed such that the latent spaces encode features of different granularities.

Conducting an experiment that is similar to \citet{vanrullen2019reconstructing}, but maps information from ealier regions of the visual cortex onto lower layers of the model, and higher regions onto higher layers could help answering whether the lower-level representations are biologically plausible.
However, a prerequisite for such a model is to find a \ac{VAE}-like model explaining cortical activity.

\subsection{Comparison to the Inferior Temporal Cortex}\label{subsec:representational-dissimilarity-matrices}
\citet{khaligh2014deep} have shown that higher layers of \acp{CNN} trained in a supervised manner explain cortical activity.
They also showed that this is not true for a variety of unsupervised models, however not for \acp{VAE}-models.
It cannot be ruled out that \acp{VAE}, even though being unrelated to lower regions of the visual system (see Section~\ref{subsec:results_visual_features_in_variational_autoencoders}) explain cortical activity in the higher layers or the latent space.
Future work should compare \ac{IT} fMRI data to higher layers of \acp{VAE} to investigate this.

\subsection{Semantic Representations}\label{subsec:semantic-representations-results}
Other than unsupervised models, supervised models are known to explain cortical activity~\citep{khaligh2014deep,cadieu2014deep}.
Regarding semantic representations, hidden layer activations of supervised \acp{CNN} are a promising candidate.

Nevertheless, semantic representations at least should fulfill the requirements discussed in Section~\ref{subsec:semantic-representations}.
Hidden layer activations of supervised \acp{CNN} have already shown to fulfill the requirement of biological plausibility to some extent.
Furthermore, the \ac{RDM} study indicates that similar stimuli activate similar subpopulations of \say{neurons} in the network.

However, it remains unclear whether these subpopulations consist of neighboring units.
This is presumably not the case as the network has no intent to group units that are active for a particular stimulus, close one to another.

\acfp{VAE}, in contrast, have the property to map similar stimuli to neighboring areas of the latent space.
A \ac{VAE} as part of a supervised \ac{CNN}\footnote{Potentially even a sequential one, see Section~\ref{subsec:sequential-data}}, trained to reconstruct hidden layer activations could therefore be a good candidate model for semantic representations.
It is suspected to obtain the \say{explain-cortical-activity} property while also grouping similar stimuli onto neighboring areas in the latent space.

This approach, however, still would not employ sparse representations.

\subsection{Sequential Data}\label{subsec:sequential-data}

All models that were trained in the course of this thesis were non-sequential.
Even though supervised models trained on static images show relatedness to the visual system~\citep{khaligh2014deep,cadieu2014deep,krizhevsky2012imagenet}, the same does not seem to hold for unsupervised models.
However, some unsupervised models trained on sequential data have shown to learn Gabor wavelets~\citep{berkes2005slow,palm2012prediction}.
This could indicate that unsupervised models need sequential data to form semantic representations.
This line of reasoning is further supported by the fact that humans perceive sequentially perceive the world.

A \ac{VAE} or \ac{VLAE}-model extended to sequential data could bring new insights into the role of \acp{CNN} as models of the visual system.

\section{Discussion and Future Work}\label{sec:discussion}

\subsection{Possible applications of VLAE-GAN model}
\citet{vanrullen2019reconstructing} learn a mapping between test subject's fMRI responses when shown images from the CelebA dataset and the latent space of a \ac{VAE}.
They discuss which brain regions have most influence in the mapping, showing that the occipital lobe contributes most and the temporal and frontoparietal Cortex less.

The \ac{VLAE}(-\ac{GAN}) used in the course of this thesis uses three latent spaces compared to one latent space for a regular \ac{VAE}(-\ac{GAN}).
Furthermore it is designed such that the latent spaces encode features of different granularities.

\textbf{DRAFT FROM HERE:}
This model allows to learn three mappings from different lobes to different latent spaces of the VLAE-GAN decoder.
Maybe this is a better model and allows deeper analyses compared to the normal VAE decoder.


\subsection{Disentangled Representations of VLAE}
\textbf{Discuss the result from section \ref{subsec:morpho-mnist-on-vlae}. For example, digit identity is encoded in} $\bm{z}_2$ and $\bm{z}_3$ \textbf{. That is another indicator to why VLAE does not learn fully independent representation (Ref \ref{subsec:independence-of-vlae-embeddings})}

\subsection{Feedback Connections of the \acl{LGN}}\label{subsec:feedback-connections-of-the-lateral-geniculate-nucleus}
\textbf{DRAFT!}
As discussed in Section~\ref{subsubsec:visual-cortex}, the \ac{LGN} receives strong feedback connections from the \ac{V1}.
Consider the \ac{LVAE}-model in Section~\ref{subsubsec:representation_learning} and the discussion concerning the top-down pass.
It is stated that, compared to the Vanilla VAE, the \ac{LVAE} model is more biologically plausible because top-down connections resemble the feedback connections in the ventral and primary visual pathway.
\citet{sonderby2016ladder} claim that the top-down pass helps improving the low-level feature representation because it enables the model to incorporate the higher-level context.

Maybe the feedback connections fulfil this very function, i.e.\ allowing visual regions operating on a lower semantic level to incorporate the high-level context to disambiguate the lower-level representations.

\subsection{Criticism on \citet{zhao2017learning}}\label{subseq:criticism_vlae}
\textbf{DRAFT!}
\citet{zhao2017learning} compared their model to a HVAE and showed the difference on a plot.
However, the plots are not really comparable.


\subsection{Visual Features in Variational Autoencoders}\label{subsec:discussion_visual_features_in_variational_autoencoders}
\textbf{DRAFT!}
\begin{itemize}
\item Reconstructions of AlexNet-VAE don't look natural at all, this might be because ImageNet, unlike CelebA, is not properly aligned and contains images from a large variety of classes.
This might be another hint towards the assumption that \acp{VAE} are no suitable model for semantic representations
\item How does it come that supervised models seem to be a good representation of \ac{IT} cortical representation~\citep{khaligh2014deep}? Maybe the brain learns in a \say{supervised} manner in the sense that if you have never seen a horse, you need someone to tell you that that is a horse in order to make sense out of it.
However, you might, at least, be able to recognize that it is an animal.
Also, if someone tells you \say{This is a horse.}, you'll be able to recognize horses from different orientations and in different body positions without further training.
This is quite different from how \acp{CNN} capture animals and is a strong hint to that supervised \acp{CNN} alone a no sufficient model for cortical IT representations.
\end{itemize}

\subsection{MNIST vs. VAE and VLAE Embeddings}
\textbf{DRAFT!}
\textbf{Discuss in how far the models are comparable because they have different architectures.}

\subsection{Comparison to \ac{IT}}
\textbf{DRAFT!}
Talk about how to compare network activites to IT activities.

\subsection{Pixelwise vs. Generative Loss}
This thesis employed two different methods to train the decoder towards generating natural looking images, namely the pixel-wise or the generative loss.
The encoder of the generative models furthermore was trained such that the hidden-layer activity of the discriminator is similar for generated and real images (in addition to the KL-term).

Advantages, disadvantages, and use-cases for both loss functions are discussed in the following.

First of all, both loss functions do not seem to be biologically plausible as they are not founded in Hebbian learning.
For the encoder loss, it could be argued that it is trained such that it elicits a \say{neural response} in the discriminator, similar to the one for real images.
This \say{activity matching} seems to be at least somewhat related to Hebbian learning as it explicitly considers the amount of activity on the level on single neurons.
However, discriminator and encoder itself are trained using backpropagation.

\citet{larsen2015autoencoding} state that the pixel-wise loss can lead to very high values even for small translation (see Section~\ref{subsubsec:representation_learning}).
Intuitively seeming like a disadvantage, this is only true for images with a high frequency and a large variance of pixel values.
A black-and-white image consisting of alternating black and white rows and the same image shifted by one pixel in $y$-direction in fact would result in a maximum pixel-wise loss as the two images are orthogonal to one another.
Natural images, however, usually are less susceptible to a large pixel-wise loss for small translation because there are less regions of high contrast.
In the context of natural images, the pixel-wise loss function for example enables the model to detect if an object is placed in another corner of the image as the loss would incrase for such high translations.
If, however, the aim is to generate more realistic images, the generative loss should be used since the pixel-wise loss leads to blurry reconstructions.
Also, \acp{GAN} empirically are not agnostic of an objects position (\textbf{REF: GAN-VLAE on dsprites}).

Another consideration is training stability.
Due to the adversarial loss, training \acp{GAN} has many difficulties (mode collapse, error oscillation\textbf{REF}), often resulting in failure modes.
Training regular \acp{VAE}, in contrast, is far more stable.

The choice of the loss function also might play an important role when studying a model’s representation with human \ac{IT} representations.
Compared to the pixel-wise loss, the generative loss allows for a more holistic image assessment.
This might play an important role in the emergence of a models’s latent representation, i.e. might lead to a more realistic latent representation in terms of human IT activations.
This aspect should be investigated in future work.

\subsection{Failure on ImageNet}

\subsection{Model Limitations}
All networks investigated in the course of this thesis are \acp{CNN}.
A network design has to be chosen for each network type (see Section~\ref{subsec:models}).
Some of these hyperparameters are context dependent.
A network operating on \textsc{Mnist}, for example, receives inputs of size $28\times 28$ pixels.
Such a networks must not be as deep as a network operating on ImageNet, as fewer layers are sufficient for the network to capture the image in it's entirety.
Other hyperparameters are chosen based on previous research.
Some configurations are known to be better for a certain task than others (see Section~\ref{subsec:models}).

Another consideration in the context of computational neuroscience is the biological plausibility.
This thesis aims at anwering in how far \acp{VAE} are reasonable models of the human visual cortex.
In order to do this, the model structure must allow comparisons with empirical data for the parts under investigation.

Specifically, lower model layers are compared with earlier, higher model layers with later regions in the visual cortex.
Furthermore, the models are chosen such that similar inputs translate to similar encodings.
The biological plausibility of the encoding should be further investigated in future research (\textbf{REF TO RELATED SECTION}).

However, due to practial reasons such as the availability of training data or computational resoureces, the model disregards actualities of the human body, a more refined model should incorporate.

Other than in the models used in this thesis, the human eye receives a stream of visual stimuli.
Perceiving movements of animate objects could play an important role in distinguishing between animate and inanimate objects.
The dissimilarity between semantic representations of animate and inanimate objects has been shown in \textbf{REF}.
Recurrent models such as LSTMs could allow to extend the models developed in the course of this thesis to sequences of images.

This thesis only considered image data to form semantic representations even though the human mind perceives the world through more senses.
Future research could aim at investigating if and by how far semantic representations can be improved by using additional sources of information, for example sound.

\section{Conclusion}\label{sec:conclusion}

\newpage
\printbibliography

\newpage
\appendix
\section{Additional Plots for Section~\ref{subsec:independence-of-vlae-embeddings}}\label{sec:additional-plots-for-section_independence}
\begin{figure}[H]
\centering
\includegraphics[width=\textwidth]{images/appendix_plots/notprop_1_3.png}
\caption{Proportionality of pixel intensities when fixing $\bm{z}_1 = \bm{z}_3=\varphi$. Created accordingly to Figure~\ref{fig:notprop}.}
\label{fig:notprop_1_3}
\end{figure}

\begin{figure}[H]
\centering
\includegraphics[width=\textwidth]{images/appendix_plots/notprop_2_3.png}
\caption{Proportionality of pixel intensities when fixing $\bm{z}_2 = \bm{z}_3=\varphi$. Created accordingly to Figure~\ref{fig:notprop}.}
\label{fig:notprop_2_3}
\end{figure}

\section{Additional Plots for Section~\ref{subsec:morpho-mnist-on-vlae}}\label{sec:additional-plots-for-section_morpho_mnist}
\foreach \i in {1,...,3}{
\begin{figure}[H]
\centering
\foreach \j in {0,...,6}{
\begin{subfigure}{0.3\textwidth}
\centering
\includegraphics[width=\textwidth]{images/vlae_embeddings/embeddings_mu_\i_\j.png}
\label{subfig:vlae_morpho_z\i_\j}
\end{subfigure}
\hfill
}
\caption{Embeddings on VLAE layer $\bm{z}_{\i}$ colored by means of MorphoMNIST}
\label{fig:vlae_morpho_z\i}
\end{figure}
}
\pagebreak
\begin{landscape}
\section{\ac{VLAE} Networks used in Experiments on Sparsity (Section~\label{subsec:effective-network-capacity})}\label{sec:listings_sparsity_networks}
\begin{lstlisting}[caption={\textsc{Mnist}-\ac{VLAE}-factor-1 Encoder},captionpos=b,basicstyle=\tiny, label={lst:sparsity-vlae-encoder-28-fm1}]
__________________________________________________________________________________________________
Layer (type)                    Output Shape         Param #     Connected to
==================================================================================================
input_1 (InputLayer)            (None, 28, 28, 1)    0
__________________________________________________________________________________________________
inference_0_conv2d_0 (Conv2D)   (None, 14, 14, 64)   1664        input_1[0][0]
__________________________________________________________________________________________________
inference_0_relu_0 (ReLU)       (None, 14, 14, 64)   0           inference_0_conv2d_0[0][0]
__________________________________________________________________________________________________
inference_1_conv2d_0 (Conv2D)   (None, 7, 7, 64)     36928       inference_0_relu_0[0][0]
__________________________________________________________________________________________________
inference_1_relu_0 (ReLU)       (None, 7, 7, 64)     0           inference_1_conv2d_0[0][0]
__________________________________________________________________________________________________
ladder_2_conv2d_0 (Conv2D)      (None, 7, 7, 64)     36928       inference_1_relu_0[0][0]
__________________________________________________________________________________________________
ladder_2_relu_0 (ReLU)          (None, 7, 7, 64)     0           ladder_2_conv2d_0[0][0]
__________________________________________________________________________________________________
ladder_0_conv2d_0 (Conv2D)      (None, 14, 14, 64)   1664        input_1[0][0]
__________________________________________________________________________________________________
ladder_1_conv2d_0 (Conv2D)      (None, 7, 7, 64)     102464      inference_0_relu_0[0][0]
__________________________________________________________________________________________________
ladder_2_conv2d_1 (Conv2D)      (None, 7, 7, 64)     36928       ladder_2_relu_0[0][0]
__________________________________________________________________________________________________
ladder_0_relu_0 (ReLU)          (None, 14, 14, 64)   0           ladder_0_conv2d_0[0][0]
__________________________________________________________________________________________________
ladder_1_relu_0 (ReLU)          (None, 7, 7, 64)     0           ladder_1_conv2d_0[0][0]
__________________________________________________________________________________________________
ladder_2_relu_1 (ReLU)          (None, 7, 7, 64)     0           ladder_2_conv2d_1[0][0]
__________________________________________________________________________________________________
ladder_0_flatten (Flatten)      (None, 12544)        0           ladder_0_relu_0[0][0]
__________________________________________________________________________________________________
ladder_1_flatten (Flatten)      (None, 3136)         0           ladder_1_relu_0[0][0]
__________________________________________________________________________________________________
ladder_2_flatten (Flatten)      (None, 3136)         0           ladder_2_relu_1[0][0]
__________________________________________________________________________________________________
mu_1 (Dense)                    (None, 2)            25090       ladder_0_flatten[0][0]
__________________________________________________________________________________________________
log_var_1 (Dense)               (None, 2)            25090       ladder_0_flatten[0][0]
__________________________________________________________________________________________________
mu_2 (Dense)                    (None, 2)            6274        ladder_1_flatten[0][0]
__________________________________________________________________________________________________
log_var_2 (Dense)               (None, 2)            6274        ladder_1_flatten[0][0]
__________________________________________________________________________________________________
mu_3 (Dense)                    (None, 2)            6274        ladder_2_flatten[0][0]
__________________________________________________________________________________________________
log_var_3 (Dense)               (None, 2)            6274        ladder_2_flatten[0][0]
__________________________________________________________________________________________________
z_1_latent (Lambda)             (None, 2)            0           mu_1[0][0]
log_var_1[0][0]
__________________________________________________________________________________________________
z_2_latent (Lambda)             (None, 2)            0           mu_2[0][0]
log_var_2[0][0]
__________________________________________________________________________________________________
z_3_latent (Lambda)             (None, 2)            0           mu_3[0][0]
log_var_3[0][0]
==================================================================================================
Total params: 291,852
Trainable params: 291,852
Non-trainable params: 0
__________________________________________________________________________________________________
\end{lstlisting}
\pagebreak
\input{listings/results_sparsity/fm1-decoder.tex}
\pagebreak
\begin{lstlisting}[caption={\ac{VLAE} network encoder used for network sparsity experiments with half the number of feature maps},captionpos=b,basicstyle=\tiny, label={lst:sparsity-vlae-encoder-28-fm2}]
__________________________________________________________________________________________________
Layer (type)                    Output Shape         Param #     Connected to
==================================================================================================
input_1 (InputLayer)            (None, 28, 28, 1)    0
__________________________________________________________________________________________________
inference_0_conv2d_0 (Conv2D)   (None, 14, 14, 32)   832         input_1[0][0]
__________________________________________________________________________________________________
inference_0_relu_0 (ReLU)       (None, 14, 14, 32)   0           inference_0_conv2d_0[0][0]
__________________________________________________________________________________________________
inference_1_conv2d_0 (Conv2D)   (None, 7, 7, 32)     9248        inference_0_relu_0[0][0]
__________________________________________________________________________________________________
inference_1_relu_0 (ReLU)       (None, 7, 7, 32)     0           inference_1_conv2d_0[0][0]
__________________________________________________________________________________________________
ladder_2_conv2d_0 (Conv2D)      (None, 7, 7, 32)     9248        inference_1_relu_0[0][0]
__________________________________________________________________________________________________
ladder_2_relu_0 (ReLU)          (None, 7, 7, 32)     0           ladder_2_conv2d_0[0][0]
__________________________________________________________________________________________________
ladder_0_conv2d_0 (Conv2D)      (None, 14, 14, 32)   832         input_1[0][0]
__________________________________________________________________________________________________
ladder_1_conv2d_0 (Conv2D)      (None, 7, 7, 32)     25632       inference_0_relu_0[0][0]
__________________________________________________________________________________________________
ladder_2_conv2d_1 (Conv2D)      (None, 7, 7, 32)     9248        ladder_2_relu_0[0][0]
__________________________________________________________________________________________________
ladder_0_relu_0 (ReLU)          (None, 14, 14, 32)   0           ladder_0_conv2d_0[0][0]
__________________________________________________________________________________________________
ladder_1_relu_0 (ReLU)          (None, 7, 7, 32)     0           ladder_1_conv2d_0[0][0]
__________________________________________________________________________________________________
ladder_2_relu_1 (ReLU)          (None, 7, 7, 32)     0           ladder_2_conv2d_1[0][0]
__________________________________________________________________________________________________
ladder_0_flatten (Flatten)      (None, 6272)         0           ladder_0_relu_0[0][0]
__________________________________________________________________________________________________
ladder_1_flatten (Flatten)      (None, 1568)         0           ladder_1_relu_0[0][0]
__________________________________________________________________________________________________
ladder_2_flatten (Flatten)      (None, 1568)         0           ladder_2_relu_1[0][0]
__________________________________________________________________________________________________
mu_1 (Dense)                    (None, 2)            12546       ladder_0_flatten[0][0]
__________________________________________________________________________________________________
log_var_1 (Dense)               (None, 2)            12546       ladder_0_flatten[0][0]
__________________________________________________________________________________________________
mu_2 (Dense)                    (None, 2)            3138        ladder_1_flatten[0][0]
__________________________________________________________________________________________________
log_var_2 (Dense)               (None, 2)            3138        ladder_1_flatten[0][0]
__________________________________________________________________________________________________
mu_3 (Dense)                    (None, 2)            3138        ladder_2_flatten[0][0]
__________________________________________________________________________________________________
log_var_3 (Dense)               (None, 2)            3138        ladder_2_flatten[0][0]
__________________________________________________________________________________________________
z_1_latent (Lambda)             (None, 2)            0           mu_1[0][0]
log_var_1[0][0]
__________________________________________________________________________________________________
z_2_latent (Lambda)             (None, 2)            0           mu_2[0][0]
log_var_2[0][0]
__________________________________________________________________________________________________
z_3_latent (Lambda)             (None, 2)            0           mu_3[0][0]
log_var_3[0][0]
==================================================================================================
Total params: 92,684
Trainable params: 92,684
Non-trainable params: 0
__________________________________________________________________________________________________
\end{lstlisting}
\pagebreak
\begin{lstlisting}[caption={\textsc{Mnist}-\ac{VLAE}-factor-2 Decoder},captionpos=b,basicstyle=\tiny, label={lst:sparsity-vlae-decoder-28-fm2}]
__________________________________________________________________________________________________
Layer (type)                    Output Shape         Param #     Connected to
==================================================================================================
z_3 (InputLayer)                (None, 2)            0
__________________________________________________________________________________________________
generative_2_dense_0 (Dense)    (None, 512)          1536        z_3[0][0]
__________________________________________________________________________________________________
generative_2_relu_0 (ReLU)      (None, 512)          0           generative_2_dense_0[0][0]
__________________________________________________________________________________________________
generative_2_dense_1 (Dense)    (None, 512)          262656      generative_2_relu_0[0][0]
__________________________________________________________________________________________________
generative_2_relu_1 (ReLU)      (None, 512)          0           generative_2_dense_1[0][0]
__________________________________________________________________________________________________
z_2 (InputLayer)                (None, 2)            0
__________________________________________________________________________________________________
concatenate_2_and_1 (Concatenat (None, 514)          0           generative_2_relu_1[0][0]
z_2[0][0]
__________________________________________________________________________________________________
generative_1_dense_0 (Dense)    (None, 512)          263680      concatenate_2_and_1[0][0]
__________________________________________________________________________________________________
generative_1_relu_0 (ReLU)      (None, 512)          0           generative_1_dense_0[0][0]
__________________________________________________________________________________________________
generative_1_dense_1 (Dense)    (None, 512)          262656      generative_1_relu_0[0][0]
__________________________________________________________________________________________________
generative_1_relu_1 (ReLU)      (None, 512)          0           generative_1_dense_1[0][0]
__________________________________________________________________________________________________
z_1 (InputLayer)                (None, 2)            0
__________________________________________________________________________________________________
concatenate_1_and_0 (Concatenat (None, 514)          0           generative_1_relu_1[0][0]
z_1[0][0]
__________________________________________________________________________________________________
generative_0_dense_0 (Dense)    (None, 1568)         807520      concatenate_1_and_0[0][0]
__________________________________________________________________________________________________
generative_0_relu_0 (ReLU)      (None, 1568)         0           generative_0_dense_0[0][0]
__________________________________________________________________________________________________
generative_0_reshape_0 (Reshape (None, 7, 7, 32)     0           generative_0_relu_0[0][0]
__________________________________________________________________________________________________
generative_0_conv2d_transpose_0 (None, 14, 14, 32)   25632       generative_0_reshape_0[0][0]
__________________________________________________________________________________________________
generative_0_leaky_relu_transpo (None, 14, 14, 32)   0           generative_0_conv2d_transpose_0[0
__________________________________________________________________________________________________
generative_0_conv2d_transpose_1 (None, 28, 28, 32)   9248        generative_0_leaky_relu_transpose
__________________________________________________________________________________________________
generative_0_leaky_relu_transpo (None, 28, 28, 32)   0           generative_0_conv2d_transpose_1[0
__________________________________________________________________________________________________
generative_0_conv2d_transpose_2 (None, 28, 28, 1)    801         generative_0_leaky_relu_transpose
__________________________________________________________________________________________________
generative_0_sigmoid_0 (Activat (None, 28, 28, 1)    0           generative_0_conv2d_transpose_2[0
==================================================================================================
Total params: 1,633,729
Trainable params: 1,633,729
Non-trainable params: 0
__________________________________________________________________________________________________
\end{lstlisting}
\pagebreak
\begin{lstlisting}[caption={\textsc{Mnist}-VLAE-factor-3 Encoder},captionpos=b,basicstyle=\tiny, label={lst:sparsity-vlae-encoder-28-fm3}]
__________________________________________________________________________________________________
Layer (type)                    Output Shape         Param #     Connected to
==================================================================================================
input_1 (InputLayer)            (None, 28, 28, 1)    0
__________________________________________________________________________________________________
inference_0_conv2d_0 (Conv2D)   (None, 14, 14, 22)   572         input_1[0][0]
__________________________________________________________________________________________________
inference_0_relu_0 (ReLU)       (None, 14, 14, 22)   0           inference_0_conv2d_0[0][0]
__________________________________________________________________________________________________
inference_1_conv2d_0 (Conv2D)   (None, 7, 7, 22)     4378        inference_0_relu_0[0][0]
__________________________________________________________________________________________________
inference_1_relu_0 (ReLU)       (None, 7, 7, 22)     0           inference_1_conv2d_0[0][0]
__________________________________________________________________________________________________
ladder_2_conv2d_0 (Conv2D)      (None, 7, 7, 22)     4378        inference_1_relu_0[0][0]
__________________________________________________________________________________________________
ladder_2_relu_0 (ReLU)          (None, 7, 7, 22)     0           ladder_2_conv2d_0[0][0]
__________________________________________________________________________________________________
ladder_0_conv2d_0 (Conv2D)      (None, 14, 14, 22)   572         input_1[0][0]
__________________________________________________________________________________________________
ladder_1_conv2d_0 (Conv2D)      (None, 7, 7, 22)     12122       inference_0_relu_0[0][0]
__________________________________________________________________________________________________
ladder_2_conv2d_1 (Conv2D)      (None, 7, 7, 22)     4378        ladder_2_relu_0[0][0]
__________________________________________________________________________________________________
ladder_0_relu_0 (ReLU)          (None, 14, 14, 22)   0           ladder_0_conv2d_0[0][0]
__________________________________________________________________________________________________
ladder_1_relu_0 (ReLU)          (None, 7, 7, 22)     0           ladder_1_conv2d_0[0][0]
__________________________________________________________________________________________________
ladder_2_relu_1 (ReLU)          (None, 7, 7, 22)     0           ladder_2_conv2d_1[0][0]
__________________________________________________________________________________________________
ladder_0_flatten (Flatten)      (None, 4312)         0           ladder_0_relu_0[0][0]
__________________________________________________________________________________________________
ladder_1_flatten (Flatten)      (None, 1078)         0           ladder_1_relu_0[0][0]
__________________________________________________________________________________________________
ladder_2_flatten (Flatten)      (None, 1078)         0           ladder_2_relu_1[0][0]
__________________________________________________________________________________________________
mu_1 (Dense)                    (None, 2)            8626        ladder_0_flatten[0][0]
__________________________________________________________________________________________________
log_var_1 (Dense)               (None, 2)            8626        ladder_0_flatten[0][0]
__________________________________________________________________________________________________
mu_2 (Dense)                    (None, 2)            2158        ladder_1_flatten[0][0]
__________________________________________________________________________________________________
log_var_2 (Dense)               (None, 2)            2158        ladder_1_flatten[0][0]
__________________________________________________________________________________________________
mu_3 (Dense)                    (None, 2)            2158        ladder_2_flatten[0][0]
__________________________________________________________________________________________________
log_var_3 (Dense)               (None, 2)            2158        ladder_2_flatten[0][0]
__________________________________________________________________________________________________
z_1_latent (Lambda)             (None, 2)            0           mu_1[0][0]
log_var_1[0][0]
__________________________________________________________________________________________________
z_2_latent (Lambda)             (None, 2)            0           mu_2[0][0]
log_var_2[0][0]
__________________________________________________________________________________________________
z_3_latent (Lambda)             (None, 2)            0           mu_3[0][0]
log_var_3[0][0]
==================================================================================================
Total params: 52,284
Trainable params: 52,284
Non-trainable params: 0
__________________________________________________________________________________________________
\end{lstlisting}
\pagebreak
\begin{lstlisting}[caption={\ac{VLAE} network decoder used for network sparsity experiments with one third the number of feature maps},captionpos=b,basicstyle=\tiny, label={lst:sparsity-vlae-decoder-28-fm3}]
__________________________________________________________________________________________________
Layer (type)                    Output Shape         Param #     Connected to
==================================================================================================
z_3 (InputLayer)                (None, 2)            0
__________________________________________________________________________________________________
generative_2_dense_0 (Dense)    (None, 342)          1026        z_3[0][0]
__________________________________________________________________________________________________
generative_2_relu_0 (ReLU)      (None, 342)          0           generative_2_dense_0[0][0]
__________________________________________________________________________________________________
generative_2_dense_1 (Dense)    (None, 342)          117306      generative_2_relu_0[0][0]
__________________________________________________________________________________________________
generative_2_relu_1 (ReLU)      (None, 342)          0           generative_2_dense_1[0][0]
__________________________________________________________________________________________________
z_2 (InputLayer)                (None, 2)            0
__________________________________________________________________________________________________
concatenate_2_and_1 (Concatenat (None, 344)          0           generative_2_relu_1[0][0]
z_2[0][0]
__________________________________________________________________________________________________
generative_1_dense_0 (Dense)    (None, 342)          117990      concatenate_2_and_1[0][0]
__________________________________________________________________________________________________
generative_1_relu_0 (ReLU)      (None, 342)          0           generative_1_dense_0[0][0]
__________________________________________________________________________________________________
generative_1_dense_1 (Dense)    (None, 342)          117306      generative_1_relu_0[0][0]
__________________________________________________________________________________________________
generative_1_relu_1 (ReLU)      (None, 342)          0           generative_1_dense_1[0][0]
__________________________________________________________________________________________________
z_1 (InputLayer)                (None, 2)            0
__________________________________________________________________________________________________
concatenate_1_and_0 (Concatenat (None, 344)          0           generative_1_relu_1[0][0]
z_1[0][0]
__________________________________________________________________________________________________
generative_0_dense_0 (Dense)    (None, 1078)         371910      concatenate_1_and_0[0][0]
__________________________________________________________________________________________________
generative_0_relu_0 (ReLU)      (None, 1078)         0           generative_0_dense_0[0][0]
__________________________________________________________________________________________________
generative_0_reshape_0 (Reshape (None, 7, 7, 22)     0           generative_0_relu_0[0][0]
__________________________________________________________________________________________________
generative_0_conv2d_transpose_0 (None, 14, 14, 22)   12122       generative_0_reshape_0[0][0]
__________________________________________________________________________________________________
generative_0_leaky_relu_transpo (None, 14, 14, 22)   0           generative_0_conv2d_transpose_0[0
__________________________________________________________________________________________________
generative_0_conv2d_transpose_1 (None, 28, 28, 22)   4378        generative_0_leaky_relu_transpose
__________________________________________________________________________________________________
generative_0_leaky_relu_transpo (None, 28, 28, 22)   0           generative_0_conv2d_transpose_1[0
__________________________________________________________________________________________________
generative_0_conv2d_transpose_2 (None, 28, 28, 1)    551         generative_0_leaky_relu_transpose
__________________________________________________________________________________________________
generative_0_sigmoid_0 (Activat (None, 28, 28, 1)    0           generative_0_conv2d_transpose_2[0
==================================================================================================
Total params: 742,589
Trainable params: 742,589
Non-trainable params: 0
__________________________________________________________________________________________________
\end{lstlisting}
\end{landscape}

\pagebreak

\section{Discriminator Network used in Section \ref{subsec:vae-generated-samples-vs-true-samples}}\label{sec:listing_discriminator_network}
\begin{lstlisting}[caption={The discriminator network used to distinguish generated \ac{VAE}/\ac{VLAE} samples from true \textsc{MNIST} images.},captionpos=b]
_________________________________________________________________
Layer (type)                 Output Shape Param #
=================================================================
input_2 (InputLayer)         (None, 28, 28, 1)         0
_________________________________________________________________
conv2d_4 (Conv2D)            (None, 24, 24, 20)        520
_________________________________________________________________
leaky_re_lu_4 (LeakyReLU)    (None, 24, 24, 20)        0
_________________________________________________________________
batch_normalization_23 (Batc (None, 24, 24, 20)        80
_________________________________________________________________
conv2d_5 (Conv2D)            (None, 22, 22, 20)        3620
_________________________________________________________________
leaky_re_lu_5 (LeakyReLU)    (None, 22, 22, 20)        0
_________________________________________________________________
batch_normalization_24 (Batc (None, 22, 22, 20)        80
_________________________________________________________________
flatten_5 (Flatten)          (None, 9680)              0
_________________________________________________________________
dense_15 (Dense)             (None, 100)               968100
_________________________________________________________________
leaky_re_lu_6 (LeakyReLU)    (None, 100)               0
_________________________________________________________________
batch_normalization_25 (Batc (None, 100)               400
_________________________________________________________________
dense_16 (Dense)             (None, 1)                 101
=================================================================
Total params: 972,901
Trainable params: 972,621
Non-trainable params: 280
_________________________________________________________________
\end{lstlisting}


\newpage
\pagenumbering{Roman}
\setcounter{page}{\thesavepage}
\section*{Acronyms}
\begin{acronym}[TDMA]
\acro{AdaIN}{Adaptive Instance Normalization}
\acro{ALAE}{Adversarial Latent Autoencoder}
\acro{VAE}{Variational Autoencoder}
\acro{CNN}{Convolutional Neural Network}
\acro{IT}{Inferior Temporal Cortex}
\acro{ReLU}{Rectified Linear Unit}
\acro{LeakyReLU}{Leaky Rectified Linear Unit}
\acro{ILSVRC2017}{Large Scale Visual Recognition Challenge 2017}
\acro{MSE}{Mean Squared Error}
\acro{RDM}{Representational Dissimilarity Matrix}
\acro{KL-divergence}{Kullback-Leibler divergence}
\acro{ELBO}{Evidence Lower Bound}
\acro{PDF}{Probability Density Function}
\acro{GAN}{Generative Adversarial Network}
\acro{VLAE}{Variational Ladder Autoencoder}
\acro{LVAE}{Ladder Variational Autoencoder}
\acro{CNS}{Central Nervous System}
\acro{V1}{Primary Visual Cortex}
\acro{LGN}{Lateral Geniculate Nucleus}
\acro{TEO}{Temporo-Occipital Area}
\end{acronym}
\newpage
\listoffigures
\newpage
\listoftables
\newpage
\section*{Erklärung}

Ich erkläre, dass das Thema dieser Arbeit nicht identisch ist mit dem Thema einer von mir bereits für eine andere Prüfung eingereichte Arbeit.\par
Ich erkläre weiterhin, dass ich die Arbeit nicht bereits an einer anderen Hochschule zur Erlangung eines akademischen Grades eingereicht habe.\par
\vspace{2cm}
Ich versichere, dass ich die Arbeit selbstständig verfasst und keine anderen als die angegebenen Quellen benutzt habe. Die Stellen der Arbeit, die anderen Werken dem Wortlaut oder dem Sinn nach entnommen sind, habe ich unter Angabe der Quellen der Entlehnung kenntlich gemacht, Dies gilt sinngemäß auch für gelieferte Zeichnungen, Zkizzen, bildliche Darstellungen und dergleichen.

\vfill

\hspace{2cm} Ort, Datum \hfill Unterschrift \hspace{2cm}


\end{document}
