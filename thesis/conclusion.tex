This thesis discussed if \acfp{VAE} are candidate models of the visual cortex, potentially allowing to obtain semantic representations of the input.
Only little evidence has been found that \acp{VAE} and other related models have this property, suggesting that these models at least need to be modified to be good models of the visual system.
A more pessimistic conclusion is that unsupervised models per se are unfit models of the visual system.

This, however, leads to the question why supervised explain \ac{IT} activity better than unsupervised models.

One assumption is that the brain also mainly learns in a supervised manner and that supervision is always required to build realistic models of the brain.
This assumption seems to be too naive for a variety of reasons.

First, the models discussed in this thesis are too unrelated to the brain to allow such a conclusion.
The role of top-down connections as well as the role of sequential data has not been investigated thoroughly.
Furthermore, there are unsupervised models leading to Gabor wavelets~\citep{berkes2005slow}.
This is another hint that unsupervised models in general might be suitable models but need more refinement.

This thesis ruled out a family of models (\acp{VAE} and \acp{VLAE} in particular) as good models of the visual system.
But also, this thesis discussed ways to extend these models to better model the visual cortex and the emergence of semantic representations, suggesting multiple approaches for future research.